\documentclass[11pt]{article}
\usepackage[utf8]{inputenc}
\usepackage[T1]{fontenc}
\pagestyle{plain}
\usepackage{times,geometry,graphicx,color,floatflt,setspace}
\usepackage[hyphens]{url}
\usepackage{wrapfig}
\geometry{margin=1.5cm}
%\onehalfspacing
\begin{document}
\title{Poetry Articles (45k words)}
\author{Tim Love}
\date{\today}

\section{Introduction}
This collection of articles should suit first year literature students and poets serious about the teachniques of their art. It's technical in parts, but the techniques used are common enough to be understood, I hope. To make each section more readable in isolation, one or two blocks of text have been repeated.

If you disagree strongly with any of the following, you might not get much from this book.
\begin{itemize}

\item    Poetry is a wide genre - if someone tells me that they like Music I wouldn't know whether to play them Abba, Garage or Chopin. Even if they say they like Jazz, there's still a wide range. Similarly, the word "Poetry" is used to describe many types of text that (beyond the media used) have little in common. In particular their aesthetics may differ. A single poem can contain aspects of many of these subgenres, including prose genres.
\item      Poems can foreground typography (see ``Notation in poetry and music'', ``Poetry punctuation'', ``Line-breaks''), sound (see ``Choosing between sound and sense''), discontinuity (see ``Juxtaposing text''), spelling (see ``Strange Forms''), shifts of tone and register (see ``Poetry, voice, and discourse analysis'') or none of these (in which case if it's short it may well be described as prose - see ``Adapting short texts for the market''). These features can interact with content and each other.
\item Reading a text is an act performed within a context that affects the method of reading. While reading, the reader might develop expectations and assumptions that can be thought of as a dialogue between the reader and the text (or poet)  
\item Poetry appreciation is affected by reputations and fashion, factors that are hard to disentangle from other, supposedly more literary features. Poetry is read by humans with different aesthetic profiles. ``Attention, Agility and Poetic Effects'' looks at the demands made on poetry readers.
\item      Poetry (like music but unlike a painting) is absorbed serially, patterns having to be built up by the reader. ``Ingarden and the Sense of Resolution'' considers what factors promote (or inhibit) this process.
 \item     Novelty is attractive but relative. Readers who have not been exposed to loquacious people with mental problems, or to various avant-garde practices may not recognise how standard those methods are (though the poems may still be good). See ``Literature, depersonalisation and derealisation''
 \item     Difficulty can be superficially attractive. See the ``Difficulty'' section


\item Correspondences with the World (or at least other texts) are deal with in ``The Real World'' section. The reader-poet relationship is dealt with in ``Linguistic psychopathology of poets and strangers''.

\item      Poetry writing is influenced by the poet's desire to be liked, or read. Or a mix. It might be therapy. See ``The games poets play'' and'' Psychology, Psychiatry and Writers Groups''.
\item      Intentionally or otherwise, the poet may use old devices on the reader (see ``Charlatanism, Poetry and Art'' and ''The Great Poetry Hoax'')
\end{itemize}

\subsection*{Poetry structure}

Inevitably, I'll sometimes focus on features (e.g. rhyme) in isolation. However, much of the material here concerns how these features are connected and prioritised. Sometimes these features are treated as if they're organised as layers - sounds/letters at the bottom, then words, etc. This metaphor leads naturally to the idea of palimpsests - the lower layers showing though. Traditionally, poetic works are structured in this way.

However, this bottom-up approach isn't how readers are likely to process the work. Certainly they won't exhaust all the possibilities in the low levels before progressing to higher ones. Initially they may well ``read for meaning'', as if it were prose. Only if unsatisfied might they resort to other means. 

Perhaps a more fruitful metaphor is to think of processes with feedback. In reality, perhaps the best analogy is more of a multi-level maze, where different people enter in different places, zooming in on detail and pulling away to get context. To some extent the readers' trajectory is under the poet's control. 

Some texts according to the writers are best read in a certain way, and advertize themselves as such (being line-broken and in a poetry magazine is a strong hint about the recommended reader strategy).

\section{Difficulty}
First I look at some aspects of a poem that may evoke a sense of difficulty. Then I consider practical approaches when faced with poems that one finds difficult - adjusting one's expectations is commonly effective.

\section{Features}
The list of features dealt with here is far from comprehensive. Many text-books
will deal with Rhyme, etc far better than I could do. 

\section{Integration}
Features are often designed to work together - Rhyming words are often followed by a line-break, for example. Identifying and prioritizing features is a complex task. Familiarity with the poet's other work may help.

Sometimes the poet deliberating withholds information so that integration can only happen at the end.


\section{The Real World}

How might be poem fare when released into a hostile world? Do everyday terms like truth and precision have a place in poetry? 

\section{Psychology}

Here I delve deeper into the psychology of reading.


\end{document}
