\documentclass[11pt]{article}
\usepackage[utf8]{inputenc}
\usepackage[T1]{fontenc}
\pagestyle{plain}
\usepackage{times,geometry,graphicx,color,floatflt,setspace}
\usepackage[hyphens]{url}
\usepackage{wrapfig}
\geometry{margin=1.5cm}
%\onehalfspacing
\begin{document}
\title{Poetry Articles (45k words)}
\author{Tim Love}
\date{\today}
\maketitle
\tableofcontents

\newpage
\section{Introduction}
This collection of articles should suit first year literature students and poets serious about the techniques of their art. To make each section more readable in isolation, one or two blocks of text have been repeated.

This book is based on the following assumptions
\begin{itemize}

\item    Poetry is a wide genre - if someone tells me that they like Music I wouldn't know whether to play them Abba, Garage or Chopin. Even if they say they like Jazz, there's still a wide range. Similarly, the word "Poetry" is used to describe many types of text that (beyond the media used) have little in common. In particular their aesthetics may differ. A single poem can contain aspects of many of these subgenres, including prose genres.
\item Poems can foreground typography (see ``Notation in poetry and music'', ``Poetry punctuation'', ``Line-breaks''), sound (see ``Choosing between sound and sense''), discontinuity (see ``Juxtaposing text''), spelling (see ``Strange Forms''), shifts of tone and register (see ``Poetry, voice, and discourse analysis'') or none of these (in which case if it's short it may well be described as prose - see ``Adapting short texts for the market''). These features can interact with content and each other. Poetry (like music but unlike a painting) is absorbed serially, patterns having to be built up by the reader. ``Ingarden and the Sense of Resolution'' considers what factors promote (or inhibit) this process.

\item Poetry appreciation is affected by reputations and fashion, factors that are hard to disentangle from other, supposedly more literary features. Reading a text is an act performed within a context that affects the method of reading. While reading, the reader might develop expectations and assumptions that can be thought of as a dialogue between the reader and the text (or poet).  Poetry is read by humans with different aesthetic profiles. ``Attention, Agility and Poetic Effects'' looks at the demands made on poetry readers.

\end{itemize}

\subsection*{Poetry structure}

Inevitably, I'll sometimes focus on features (e.g. rhyme) in isolation. However, much of the material here concerns how these features are connected and prioritised. Sometimes these features are treated as if they're organised as layers - sounds/letters at the bottom, then words, etc.  Geoffrey Hill wrote that "Simultaneous composition on several planes at once is the law of artistic creation, and wherein, in fact, lies its difficulty", (``A Postscript on Modernist Poetics''). This metaphor leads naturally to the idea of palimpsests - the lower layers showing though. Traditionally, poetic works are structured in this way.

This bottom-up approach isn't how readers are likely to process the work. Certainly they won't exhaust all the possibilities in the low levels before progressing to higher ones. Initially they may well ``read for meaning'', as if it were prose. Only if unsatisfied might they resort to other means. 

In reality, perhaps the best analogy is more of a multi-level maze, where different people enter in different places, repeatedly zooming in on detail and pulling back to get context. To some extent the readers' trajectory is under the poet's control. 

Some texts according to the writers are best read in a certain way, and advertize themselves as such (being line-broken and in a poetry magazine is a strong hint about the recommended reader strategy).

\newpage
\section{Difficulty}
First I look at some aspects of a poem that may evoke a sense of difficulty. Then I consider practical approaches when faced with poems that one finds difficult - adjusting one's expectations is commonly effective.

\subsection{Poetry and communication}

The two nouns in the title sometimes thread through fractious discussions. Both
terms are rather slippery, and depend in turn on notions like Meaning,
Language and Understanding, discussion of which can easily hijack the
debate. Also once failure of communication becomes an issue so does blame,
arrogance, incompetence, and fraudulence. Here I'm going to restrict myself
to mentioning just a few points, some borrowed from my other articles.

\subsubsection*{Language} 

Language is used for many purposes other than information transfer. It's
sometimes a replacement for pictures in order to record. People talk to
themselves, cry out in pain, pray, chant, seduce, and tell jokes. Language can perform
the role that nit-picking performs in a troop of apes. It can establish and reinforce power hierarchies.  


Does poetry have as wide a range? I don't see why not. For instance,
it's used to help people remember how many days are in each month, and to release emotion when a princess dies. Much of the
time it's only read by the poet, if that (John Stuart Mill wrote that
"eloquence is heard; poetry is overheard"). That other people read the poem
may be incidental - after all, a person playing a round of golf needn't be
watched for the activity to be worthwhile.

\subsubsection*{Understanding} 

I heard an educationalist suggest that the word "understand"
should be banned in the educational world because it has too many meanings.
Can you play tennis? Do you \textit{understand} tennis (the quantum mechanical reasons why balls bounce)? Can you add numbers?
Do you \textit{understand} 1+1=2? (remember that Bertrand Russell in Principia
Mathematica took a few hundred pages to establish it). Do you understand Mondrian? Can you hence explain a Mondrian to someone who
dislikes him? Do you understand melody or how much your lover
misses you? Some poems might start as if they can be understood the way a
novel can, but end up like a Mondrian.

 Wittgenstein wrote that "We speak of                                                              
understanding a sentence in the sense in which it can be replaced by                                  
another which says the same; but also in the sense in which it cannot be                              
replaced by any other". How can one test understanding? And when
understanding merges into appreciation, how can one know if someone likes a
work?

\subsubsection*{Meaning and poetry} 

For some people a poem needs to be paraphrasable to have a meaning, and the poet knows what the meaning is. Others feel that neither the reader nor the author need
"understand" the text for it to be considered effective 
\begin{itemize}
\item "Genuine poetry can communicate before it is understood", T. S.
Eliot 
\item "The chief use of the 'meaning' of a poem, in the ordinary                             
sense, may be ... to satisfy one habit of the reader, to keep his mind                                
diverted and quiet, while the poem does its work upon him" T.S. Eliot
\item "Four major thinkers, Darwin, Marx, Frazer and Freud, gave grounds for                            
the belief that the artist often does not know what he is doing", William
Empson

\end{itemize}


\subsubsection*{Communication}

If understanding/appreciation is hard to assess, it's harder still to assess a communication's success. 



It's common for readers to believe that a writer knows what s/he means and puts it into words so that others can learn what s/he knows. The simplest signal-processing-based diagram to represent this is
\begin{verbatim}
   author - text - reader
\end{verbatim}

or
\begin{verbatim}
   author - sound - listener 
\end{verbatim} 

In the text-mediated version
especially, there is no direct contact between author and audience. At best
the message will arrive undamaged, but noise in the communications channel is always a
risk.

This is the model appropriate for learning facts from an encyclopedia, but poetry often plays a different language game.

 The simplistic model above can be developed to model more complex situations in several ways. 
\begin{itemize}

\item The message is usually in a code of sorts (a protocol or language).
Conversion to and from this code may be inexact or at least difficult 
\item The diagram is "New Criticism" in its purest form. In practise the
author works within a context (era, country, gender, lifestyle) and
the reader within another. The reader may or may not be aware of the
author's context, which can lead to misunderstandings 
\item Sometimes
extra stages are added, the author creating a \textit{narrator} who's
producing words for an implied \textit{addressee} 
\item The situation where
the reading happens is significant, and should be included in the schema. Found poetry (and the Art equivalent) works because of this

\item "Even if a poet is pragmatically                                                                  
dedicated to transmitting a message, the temporal delay involved in                                   
preparing an artifact (poem as message) plunges the activity into a                                   
perceptual realm distinct from the intersubjective circuit of a                                       
communications environment", Jed Rasula, "Syncopations"
\end{itemize}


The signal processing metaphor breaks down for poems where the "noise" (the words, the medium) is part of the message. Rather than signal processing, game theory has been used to model the poet/reader relationship
\begin{itemize}
\item  "the reading process can be                                                                      
represented as a one-sided bargaining process of imperfect information [with]                         
mutual interdependence (reflexivity), fixed order of play, one-sidedness of                           
the communicative process, possibility of limited pre-play communication                              
(e.g. by means of publishing, advertising, generic conventions), inability                            
to make side-payments or binding agreements", "The Role of Game Theory in                             
Literature Studies", Peter Swirski in "Empirical Approaches to Literature" 
\end{itemize}


Seeing all this scope for errors it's easy to forget that most of the
time communication works just fine when it's intended. Poetry presents (often deliberately)
greater problems than usual, exploiting these problems - "If what has happened in the one person were                                                                                       
communicated directly to the other, all art would collapse, all the effects                           
of art would disappear", Valery, 

\subsubsection*{Deliberate difficulty} 

Sometimes authors are trying to communicate something difficult in the easiest way possible, or they feel that an initial difficulty makes eventual understanding more likely, or they believe that a poem's "meaning" is constructed by the reader (the poet's intentions being irrelevant), or the text is a product of the society and so society generates and consumes meaning, the poet being a mere conduit. Authors may want to produce work that should be approached partly like a Rorschach blot, a riddle, or a melody ("The poet is not there to share a poetic communication, but to stimulate an imaginative speculation on the nature of reality", Barbara Guest). Even if an author's aim is to conventionally communicate, there are several reasons why they might make texts in some way "difficult". 
 \begin{itemize} 
\item Recent work by
Oppenheimer has shown that texts in a font that was hard to read were
better understood by students 
\item "the problem of the poet, if he is                             
to produce work which forces his readers to experience real perception, is                            
how to make recognition difficult and perception inevitable. The poem                                 
should give an immediate impression of having a 'message' function, in                                
order to achieve unity, but not more than an impression need be provided at                           
the most accessible 'levels' of the poem", "Poetic Truth", Skelton
\item "One encounters in any ordinary day far more real difficulty than one                             
confronts in the most 'intellectual' piece of work. Why is it believed that                           
poetry, prose, painting, music should be less than we are?", Geoffrey
Hill 
\item "The relationship between an artist and reality is always an                            
oblique one, and indeed there is no good art which is not consciously                                 
oblique. If you respect the reality of the world, you know that you can                               
only approach that reality by indirect means", Richard Wilbur
\item "The general public ... has set up a criterion of its own, one by which                           
every form of contemporary art is condemned. This criterion is, in the case                           
of music, melody; in the case of painting, representation; in the case of                             
poetry, clarity. In each case one simple aspect is made the test of a                                 
complicated whole, becomes a sort of loyalty oath for the work of art. ...                            
instead of having to perceive, to enter, and to interpret those new worlds                            
which new works of art are, the public can notice at a glance whether or          
not these pay lip-service to its own 'principles'", Randall Jarrell
\item In poetry, unlike in other forms of discourse, obscurity might be an aesthetic principle; indeed, poetic discourse enjoys a special privilege: it may run counter to the fundamental requirement of language, namely communicability, and may infringe some of the basic rules of language. ... It is able to depart from the requirements of coherence, cohesion and consistency with ideas expressed in the text, or indeed with external knowledge. It does not establish any information known both to the originator and to the recipient that would ensure a grasp of the information that follows (see Clark and Clark, 1977). It will frequently depart from the literal sense of the words that it uses and endow them with new meanings. And despite all this, simply because it is a poem, it will be perceived as a significant text (Iris Yaron-Leconte)
\item For the person who reads a poem, obscurity is one of the elements that create 'magic'. Unlike in the case of non-poetic obscure texts, the fact that understanding is deferred is part of the aesthetics of obscurity and this in itself is thus linked to the experience that the poet seeks to create for the reader (Iris Yaron-Leconte)

\end{itemize}

\subsubsection*{John Ashbury and communication} 

In an \url{http://blogs.spectator.co.uk/books/2013/02/interview-with-a-writer-john-ashbery/} interview in The Spectator in Feb 2013, John Ashbury made some points that seem amenable to the signal-processing metaphor
\begin{itemize}
\item "\textit{After listening to a piece of music we often feel a sense of satisfaction and understanding. Poetry aims for this as well, but it's limited by what the words mean, whereas in music, the message is exact and intelligible but without being paraphrasable like language}"
\item "\textit{On the one hand I have always felt the most important thing that a writer should do is to write something that people will understand. But I also want to write poetry that expresses my usually tangled thoughts without condescending to a reader. }"
\end{itemize}

I'm not convinced that "in music, the message is exact and intelligible". I can't even think of how one could in general test whether this is so (other than with program music). However, I can appreciate that sometimes the literal meaning of individual words can interfere with the poem's intended meaning. The second comment is one that I imagine most writers would agree with, but of course it depends who the people are and in what sense they understand. Elsewhere, Ashbery has said "\textit{I'm ... mildly distressed at not being able to give a satisfactory account of my work because in certain moods this inability seems like a limit to my powers of invention. After all, if I can invent poetry, why can't I invent the meaning?}", so I'm unsure quite what he means.


\subsubsection*{Rothko and communication} 

It's interesting to consider Mark Rothko's views on how his Art communicated. He wrote that "\textit{The fact that people break down and cry when confronted with          
my pictures shows that I can communicate those basic human emotions ... the                           
people who weep before my pictures are having the same religious experience                           
I had when painting them. And if you say you are moved only by their color                            
relationships then you miss the point. I am interested in expressing the big emotions - tragedy, ecstasy, doom}".


In this phase he seemed to
feel that there were right and wrong ways to approach his work. I'm not
convinced by the rigour of his tests on his communication successes, though his statements about technique show that the "signal processing" metaphor can be applied to non-figurative work. He was aware of some reasons for misunderstandings -

\begin{itemize}
\item  "\textit{Since
my pictures are large, colorful, and unframed, and since museum walls are
usually immense and formidable, there is the danger that the pictures
relate themselves as decorative areas to the walls}" 
\item  "\textit{We favor the
simple expression of the complex thought. We are for the large shape
because it has the impact of the unequivocal. We wish to reassert the
picture plane. We are for flat forms because they destroy illusion and
reveal truth}" 
\end{itemize}

\newpage\subsection{Pragmatic approaches to difficult texts}
\begin{table}[htbp]
\centering
\begin{tabular}{|l|l|l|}\hline
 & \textbf{like} &\textbf{ don't like} \\\hline
\textbf{understand} & my good stuff& my bad stuff\\\hline
\textbf{don't understand} &Bach & ???\\\hline
\end{tabular}
\end{table}
"I don't get it" is a reaction I have when reading some works. I may still like them, so I could fill in the first column of the table on the right. I think there are pieces that I understand and (hence) don't like, but is it fair to dislike a piece that you don't understand? Would one dislike a piece merely because it's in a language you don't understand? And yet, in other situations "I don't like" versus "I don't understand" seems too artificial a distinction, a politically-correct attempt to blame oneself rather than the writer - is disliking always the result of failed (inappropriate) understanding? I doubt it. 


Amongst the options when facing something you don't get are
\begin{itemize}
\item Trying to understand it
\item Accepting that you won't be able to understand it - a blindspot
\item Believing that the work is bad \textit{because} you don't understand it, though it may still be interesting
\item Pretending to like it (fortunately, you can say you like a piece without needed to convince people that you understand it)
\end{itemize}

I'll look at those options in more detail here, bringing together some articles I've posted elsewhere, mostly dealing with poetry.

\subsubsection*{Trying to understand}

I don't suspend disbelief very willingly. I like to stay close to the text. If there's something I don't understand I don't like pretending it's not there, skimming over it until I find something I do understand. When evaluating a poem I don't edit away the inconvenient mysteries. I'm prepared to blame the poet, even call their bluff. Consequently I struggle with some poetry, and read books that attempt to explain it to me. Amongst the books that analyse poems are
\begin{itemize}

\item 52 ways of looking at a poem (Ruth Padel). \textit{Contemporary poems explained for the benefit of intelligent laypeople. The material derives from newspaper articles. I like it.} 

\item The poem and the journey (Ruth Padel). \textit{More of the same. Poems by Prynne etc.}
\item Nearly Too Much: The Poetry of J.H. Prynne (N.H.Reeve and Richard Kerridge). \textit{The deep end. See also \url{http://www.arduity.com/poets/prynne/prynnedifficulty.html} J H Prynne and difficulty from Arduity}.
\end{itemize}

Where these sometimes fail for me is even when they can decode a difficult phrase, they don't explain why a simpler phrase wasn't used instead, or why a more obvious interpretation is discounted.

I also read theory and articles, mostly to shake me out of my habits - 
\begin{itemize}
\item Close calls with nonsense (Stephen Burt). \textit{An unpreachy look at the factors and fashions involved with recent North American poetry.}
\item How to write a poem (John Redmond) \textit{A book for beginners that provides building blocks more in keeping with contemporary poetry - a Jori Graham poem is successfully discussed}
\item Literary Theory: A Guide for the Perplexed (Mary Klages)
\item next word, better word (Stephen Dobyns). \textit{Includes in-depth analysis of poetry and prose, showing how some older methods of analysis still have a place.}
\item Problems and poetics of the nonaristotelian novel (Leonard Orr). \textit{A reminder that harmony and unity aren't unquestionable delights.} 
\item \url{http://www.arduity.com/} Arduity: Clarifying difficult poetry \textit{A site with articles that help to ease the pain of supposedly difficult poetry}
\end{itemize}


Then there's the poetry itself. Sometimes I just give up.

%\begin{figure}[htbp]
\begin{wrapfigure}{r}{0.3\textwidth}
\centering
\includegraphics{rabbit.jpg}
\end{wrapfigure}
%\end{figure}
I suspect some of my troubles are caused by my lack of awareness of factors that affected the poet, though becoming aware of these factors doesn't always solve everything

\begin{itemize}
\item Maybe there are unknown aims that compromise my view of the poem. If I only see this drawing as a rabbit looking left I might criticise the execution, not realising that it's a duck looking right too. If I then notice the duck and point out that the duck's not very good either, the artist might respond by saying that accuracy of either image isn't the point. And they'd be right, but if  accuracy doesn't matter one way or the other, the artist might just as well be more accurate in order to placate people who judge by measuring the realism. Or is the artist's technique lacking? (it's my drawing, and mine certainly is). A poem, like a picture, can do more than demonstrate an idea - it can also fulfil other aims. The criticisms of the piece might still be valid even if the critic missed the "main point" - why should the main point be the only one?
\item Maybe the poem's constrained by a form that's hard to notice (it's an acrostic, or an N+7 piece, for example). 

\item Maybe the poem's a reaction to something - the poet's previous style, or a prevailing fashion. This might explain the poem (and its historical or personal importance), but doesn't justify its contemporary value as a poem. An old poem rebelling against end-rhyme loses much of its force nowadays. Besides, there are good and less good ways of reacting, however worthy the cause. 

\end{itemize}

\subsubsection*{Blindspots}
\begin{itemize}
\item \textit{"You know I can't stand Shakespeare's plays, but yours are even worse" -
    Tolstoy to Chekhov}

\item \textit{"Many accepted authors simply do not exist for me. Bertolt Brecht,
    William Faulkner, Albert Camus, many others, mean absolutely nothing to
    me" - Nabokov}
\end{itemize}

The "I don't like it" vs "I don't get it" distinction is hard to assess. Which predispositions and situations make people say one rather than the other, and what motives and consequences might there be? It depends on the context.
\begin{itemize}
\item Social situation - In the tutor/pupil situation perhaps it's easier for the
  tutor to say "I don't like it" and the pupil to say "I don't get it". In
  judges' reports you might not often see lines quoted that the judge admits to
  not understanding. Of course contestants are happy to accept that judges have
  preferences, but if they have unadvertised biases it's more awkward. Perhaps
  judges should be more upfront about their blindspots beforehand, or return
  the entry fees of poems that they feel unqualified to judge :-). As a ploy they might be prepared to say that a particular poem is "good of its type" but then dismiss that type for reasons they don't explain. 
\item The Artist - it's easier to admit to blindspots about some artists than
  others. Stockhausen, Larkin, Prynne, and Olds are fair game. However a
  dislike or incomprehension of Neruda might be viewed with more
  suspicion. Perhaps there are poets for whom the only acceptable reason for
  disliking them is that you don't understand them - to know them is to love
  them, though they may be difficult to get to know. Shakespeare? Geoffrey Hill?

\end{itemize}


In practise the distinction might just be another way of saying something else
- whether you'd bother re-reading the work, for example. Maybe "I don't get it"
can mean "I don't like it but famous people do, so I'm inadequate". 



There are some poets' work I find easy to like but hard to love (Glyn Maxwell
maybe). There are poems I'd rather read about than read (Les Murray's
maybe). If I understand and like a poem I may not agree with it (it may be
a Political poem, for example) but that's a different matter. 



\subsubsection*{Attitudes to Blindspots}


 I heard Philip Hensher (novelist and reviewer) being interviewed recently,
 saying that he didn't get Ian McEwan's work and hence didn't review
 it. I think he said he
 was happy to accept that 
people had blind spots - big ones even. But perhaps he doesn't really like
 McEwan's novels. Such meta-judgements are going to be error-prone though.
I'm not keen on Olson. I'd go so far to say that I think he's more important than good, that I don't have a blindspot as far as he's concerned. Whereas I think there's more to Heaney than meets my eye. 



 Moreso than judges, magazine editors can afford to have blindspots - they're  what give their
  magazines character. Practising writers perhaps have the most license. Even so there may be repercussions.
Nabokov said - "[music] I regret to say, affects me merely as an arbitrary succession of more or less irritating sounds". With the benefit of medical advances we'd tend to label this as a medical condition, a handicap. But Nabokov's admission didn't affect his work's reception.
Larkin's dislike of modern jazz is treated more as a personality defect, though
  some people read aesthetic limitations into it. 
I prefer even "Frankie goes to Hollywood" to Mozart. Ok, so I like some JS Bach, Barber's Adagio, Bartok's quartets, but surely my statement exposes a lack of taste.




\textit{de gustibus non disputandum} - but
career poets had better not advertise too many of their poetry blindspots if they want to judge competitions, or if they don't want to discourage people coming to their workshops.


\subsubsection*{Covering up blindspots}

Whether the blindspot's related to emotion, empathy or intellect there may
  be remedies. Firstly there may be an underlying perception problem - if
  you're totally colourblind you're not going to be able to respond to blue or
  even "blue". Readers won't see syllabics unless they count the syllables,
  and some readers don't listen to the  sound of the words. Sometimes these
  perceptual deficits are due to inattention and can be remedied.



But the problems may lie deeper. Psychology tests nowadays can reveal all kinds of individual quirks in our
visual and language processing. The effects show up in contrived laboratory
conditions. In everyday life we manage to compensate for them - e.g. the face-blind
pay more attention to gait, clothes, etc. It's not surprising that poetry would
reveal individual differences in apprehension. I think my poetry appreciation
is a patchwork of blindspots - from poem to poem or even from line to line. I approach texts with a mishmash of innate and learnt behaviours, but usually act as if the unevenness is all in the text.




Each of our eyes compensates for the blindspot of the other in most situations. How can one compensate for aesthetic blindspots? Give a computer enough examples of so-called good and bad art (in a limited field) and pattern-matching software can often judge future examples pretty well (though it may not be able to give explanations). You can train yourself in the same way - working by analogy and general principles. In poetry, where there's a wide range of tastes anyway, it's not too hard to bluff one's way through one genre or facet of poetry, especially if you've acquired credibility in other genres. Indeed, opinions by newcomers and outsiders might prove valuable. If I were to judge Mr World I might well make a less controversial decision than if I were judging Miss World - fewer hormones and idiosyncrasies get in the way, and I'd use more general principles and cliches/archetypes.


\subsubsection*{Learning from bad writing}

I read small-press literary magazines, online writing forums and go to writers meetings. Not all that I read or see at those venues is publishable. Bad or not, I think there's much to be learnt from it. Equally I think one can sometimes learn much about a well-known writer by considering their less successful works, where their techniques, quirks and habits are sometimes laid bare. 


Those who only read good work are leaving themselves vulnerable to charlatans, or to people who can imitate what sells. Bad work gives you a better appreciation of what is easy and hard to do, and helps you to calibrate your appreciation of supposedly better works. Bad work might be excellent in some respects - plot for example - but fatally flawed in another. It may be patchy - should a work be judged by its worst passages or its best? In "Reading like a writer" Francine Prose notes that "\textit{At lazy moments, F.Scott Fitzgerald could resort to strings of cliches}".



\subsubsection*{Pretending to like a text}

There are many reasons why people might say they like a poem, but if someone says they like a poem of yours, think twice before asking them why - it's likely to be embarrassing for both of you. The odds are that inter-personal expectations of behaviour affect what people say more than the desire for aesthetic authenticity. This isn't easy to prove, but if everyone who said they liked a poem read the book that the poem came from (or even bought it) the world would be a very different place.

How much does the public - or even poetry audiences - understand about poems? Maybe less than is generally assumed
\begin{itemize}
\item Jon Stone wrote on his blog "\textit{I'm still not sure, when I look around at poetry audiences, how many really notice or care about texture or music, and how many are jonesing for their next hit of clarity}"
\item Wayne Burrows in his Thumbscrew article suggests that "\textit{'Most people', quite simply, don’t know about poetry}".
\item Housman wrote "\textit{I am convinced that most readers, when they think they are admiring poetry, are deceived by inability to analyse their sensations, and that they are really admiring, not the poetry of the passage before them, but something else in it, which they like better than poetry}".
\item Harold Munro wrote "\textit{The public, as a whole, does not demand or appreciate the pure expression of beauty. Its cultured members expect to find in poetry, if anything, repose from material and nervous anxiety; an apt or chiselled phrase strokes the appetites and tickles the imagination. The more general public merely enjoys its platitudes and truisms jerked on to the understanding in line and rhyme; truth put into metre sounds overwhelmingly true}".
\item In the Rialto they said that "\textit{During a recent research project into reading habits conducted at the University of North Carolina, Greensboro, a cross-section of the public nominated poetry to be the most annoying category of book currently published .... after a sustained period of reading poems, thirty six complained of headaches or migraine, twenty-seven suffered indigestion, and two became argumentative resulting in violent exchange .... eighty-two of the hundred people tested did fall asleep for prolonged periods at some point during their reading of poetry}"

\end{itemize}


If anything, I think that experienced poetry-readers (even reviewers and judges) have more reason to dissemble. If they don't understand/like something that for career, personal or reputation reasons they feel they should praise (e.g. Rilke's poems), what else can they do?

 A combination of ambiguous statements and use of the Forer effect can effectively mask blind spots and inconvenient opinions (the Forer effect - used by fortune tellers - is when a person who's described in a phrase that could be applied to many people, think that it's especially applicable to themselves). How about "A sensitive, controlled writer"? Or a writer "with understated insight"? Suggesting that a work has "subtle irony" (or subtle anything, because "subtle" can mean "just a bit of") is safe, as is "deceptively deep" or "repays rereading". Then there are the unfalsifiable phrases that one might find in wine reviews - "muscular yet silky".



Can fakery be detected? It's not as simple as that. For a start, some people think that any use of the intellect rather than the heart is "faking it". Also one can begin by faking it then end up loving it. But if poets on R4's Saturday Review or BBC2's Review Night say that some Art Exhibition's "Extraordinary" (inarticulate gushing being a common enough strategy to cover ignorance) it would be interesting to see if they subsequently go to similar exhibitions. Maybe. Maybe eventually it's possible reach the stage when one can say (as amateurs also do) "I like all sorts of poetry as long as it's good" and be believed, but it's hard work.


\newpage
\section{Features}
The list of features dealt with here is far from comprehensive. Many text-books
will deal with rhyme, confessionalism, etc far better than I could do. The section about voice looks at how much a voice in
poetry should imply a character. If a voice jumps from topic to topic is the poet trying to suggest that the persona has ADS?


\subsection{Linebreaks}
\begin{table}[htbp]
\centering
\begin{tabular}{|l|l|}\hline
\textbf{Format} & \textbf{Frequency}\\\hline
2 line stanzas & 7\\
3 line stanzas & 14\\
4 line stanzas & 4\\
5 line stanzas & 5\\
6 line stanzas & 2\\
7 line stanzas & 2\\
8 line stanzas & 1\\
Misc stanzas & 8\\\hline
\end{tabular}
\caption{``The Brief History of a Disreputable Woman''}
\end{table}
Deviations from norms will be noticed. In most prose, line-breaks are deviations. The norms for poetry seem to be changing.   Here are the statistics of Jane Holland's ``The Brief History of a Disreputable Woman'' (Bloodaxe 1997). Note the high percentage of poems with regular 2 or 3 lined stanzas. For regularity, the ``Brighton Pilgrimage'' poem takes the prize - 18 7-lined stanzas
where the longest line is about 1cm longer than the shortest.

In the last decade or so, line-breaks seem often used to produce equally shaped stanzas in
this way. Like any pattern it offers the writer chances to thwart expectation - units can be end-stopped or enjambed, for example. The requirements of form also give the writer an answer to people asking why the poet broke a line. Stanza lengths can (indeed, should) be varied from poem to poem. The important thing is not to let any line stick out more than 2cm from a neighbour. Once the poem's been shaped, minor tweaks can be made to exploit a line- or stanza-break, but these shouldn't be too obvious - such effects are often pretty cheap, and they might draw attention to the other, form-driven line-breaks. For added variety regular indenting can be used too. The final stanza is allowed to be a line shorter or longer than all the rest. It can even be a  single-line.


Why do poets use the form? After all, the line-break's potential in this context is limited. I guess the form's purpose is partly to please the eye and partly to get people in the poetry mood, to get them to ``read into'' the work. As Culler wrote in ``Structuralist Poetics'', this will make readers see extra meanings (the word ``red'' will burst with connotations), and affect their interpretation of style (reportage will become ``restrained writing''). There'll also be a tendency to read a fragment as the tip-of-an-iceberg.

Note the key-role played by line-breaks. Not only do they indicate that the text is a poem (giving it a charge, an aura), but by encouraging minor closures they help readers to focus on (and magnify) particular phrases 
as well as generating extra interpretations - e.g. ``I am good/for nothing''.

Given the charitable status granted to poetry by readers, any text is likely to seem more significant when read as a poem, so I think that it's only fair to raise the bar for text with poetic pretensions. In ``A Lope of Time'' Ruth O'Callaghan writes ``smoking at an open window, the man notes the abandoned boat. Come spring he will replace it''. I misquote; actually she wrote
\begin{verbatim}
smoking at an open window
                      the man
        notes
                the abandoned boat

come spring 
        he will replace it
\end{verbatim}

I don't think this earns the right to be read too generously.

Rather than use the shape of the text to indicate that the work should be read poetically, writers can use the context. It's common nowadays for poetry books to include at least one poem that has no line-breaks.  Lachlan Mackinnon's ``Small Hours'' takes this approach further. On the flap it says  that the book ends with ``a long poem ...  written mostly in prose''. The piece in question (``The Book of Emma'') takes up 63 pages. Here are some extracts
\begin{itemize}
\item ''The only television I watched as an undergraduate was the separate inaugural speech President Carter had recorded for Europe on the subject of nuclear weapons. We just didn't. Nowadays people have sets in their rooms. And mobiles. They stay in touch with home friends in a way impossible and unimaginable for us. They text and email. This may be an epistemic shift but they feel terror loneliness and grief no less than we did'' (from section XL).
\item ''Of course in making this thing about you or around you I am talking about my youth and homesick for it. But that is not the point. The point is that at one time in one place I met someone who became to me a living conscience'' (from section XLVIII) 

\end{itemize}

It's interesting to note the reception to this piece
\begin{itemize}
\item \url{http://www.independent.co.uk/arts-entertainment/books/reviews/small-hours-by-lachlan-mackinnon-1900481.html} Boyd Tonkin (The Independent) - \textit{It is a poet's prose: thrifty, rhythmic, specific, given to darting shifts in pace and focus.}
\item  \url{http://www.guardian.co.uk/books/2010/aug/14/small-hours-lachlan-mackinnon-poetry} Carrie Etter (The Guardian) - \textit{''The Book of Emma'' creates much of its poetry through command of sentence rhythms, repetitions of sound, and epic movement between individual experience and historical perspective.}
\item  \url{http://blogs.warwick.ac.uk/morleyd/tag/lachlan_mackinnon/}
David Morley (blog) - \textit{''The Book of Emma'', which is neither prose poetry nor poetic prose but a vivid series of elliptical, connected flash-backs that have the quality of flash fiction - except we are clearly hearing a poem... - it is a highly successful experiment in form.}
\end{itemize}


I'd call it prose written in prose. Yes, it has shifts of time and subject, but thankfully so does prose. It has a consistent voice. Its imagery and analogies are developed at a leisurely pace. There are leit-motifs and unspoken interconnections. It doesn't exploit sound effects. But if it doesn't need line-breaks why does an earlier poem, ``Midlands'', need so many? It has these passages: 

\begin{itemize}
\item  ``TB and rickets/ are back in cities, but these towns/ are too small to support/ such destitution''
\item ''Canals hidden/ like avenues by trees// until the bank-holiday/holiday-makers come/ in narrow-boats dolled up/ like gypsy caravans/ with new gloss/ blue, orange, red''
\end{itemize}


It's commonly said that some poems are ``just prose chopped up'', but even if a text is ``poetry chopped up'' it's faulty. In Ruth O'Callaghan's piece what are line-breaks for? I'm not the only person puzzled by latterday line-breaks
\begin{itemize}
\item ''the free verse, now dominant not only in the US but around the world, has become, with notable exceptions, little more than linear prose, arbitrarily divided into line-lengths'', Marjorie Perloff, ``The Oulipo Factor'', Jacket 23
\item  ``The poetic line seems highly problematic nowadays and it sometimes seems better to avoid it altogether'', Frances Presley, ``Poetry Review'', V98.4, 2008
\item  ``Not only hapless adolescents, but many gifted and justly esteemed poets writing in contemporary nonmetrical forms, have only the vaguest concept, and the most haphazard use, of the line'', Denise Levertov'', ``On the Function of the Line'', 1979
\end{itemize}

\begin{table}[htbp]
\centering
\begin{tabular}{|l|l|}\hline
\textbf{Format} & \textbf{Frequency}\\\hline
2 line stanzas & 4\\
3 line stanzas & 3\\
14 line stanzas & 1\\
Prose & 3\\
Misc & 13\\
Sonnet & 1\\
Triangular & 1\\\hline
\end{tabular}
\caption{Nathan Hamilton's selection of recent poetry}
\end{table}

Asking the poet doesn't always help. In ``Acumen 85'' (May 2016) William Oxley asked Harry Guest ``\textit{how do you determine where to break the line?}'' He replied (to my mind unhelpfully) with ``\textit{... the poem finds itself ... The poem tells me when to go on to the next line and, if I find the advice reasonable, I obey}''. Hugo Williams wrote ```\textit{if I don't want the line to be broken, I break it in an inappropriate place, forcing the reader to go on, as if there were no break}'' (The Poetry Review,  V.104:1, Spring 2014) but why break the line at all? Of course, line-breaks do affect processing - see for example \url{http://www.let.rug.nl/hendriks/papers/koopsvantjagt2013.pdf} Look Before You Leap: How Enjambment Affects the Reading of Poetry by Ruth Koops van‘t Jag et al (Center for Language and Cognition, University of Groningen) - ``\textit{The two experiments reported here show that poetry is indeed processed differently from prose, and that different types of enjambment are associated with different modes of processing}''. However, I doubt that their full potential is exploited by regular, rectangular stanzas. Nathan Hamilton's selection of recent poetry in Rialto 70 (2010) has these statistics. It's unfair to compare this multi-author sample with single-author books, but maybe it's a sign that line-breaks are  regaining their power. In Mackinnon's ``Midlands'' the line-breaks are for making each stanza 9 lines long, which is currently considered a worthwhile aim, but perhaps ``The Book of Emma'' signals a further drift of norms. There's no need to add line-breaks to a text if Faber label it as poetry. If Faber accept this piece as poetry, what prose would they turn down? Anything with sections longer than 2 pages?  


\newpage\subsection{Punctuation}
In Feb 2010's "The HappenStance Story Ch 4" Helena Nelson wrote "Anne Stevenson
  ... told me that semi-colons and colons were prose punctuation, not poetry". She then asked for consistency, and asked "when can a line or stanza break serve instead of a comma?" Here I attempt some answers.

Punctuation has several uses  - grammatical-logical, rhythmical-oratorical, to report speech, or for visual effect. According to Franke Verlag "there was a movement away from rhythmical-oratorical punctuation to grammatical-logical usage between about 1580 and 1680 ... It was only in the decade of the 1840's that the
grammatical-logical theories finally triumphed."  There was also around that time (and not by coincidence) a shift from oral to written poetry, which accommodated more complex sentence structures.

In poetry new ideas don't wipe out the old, they co-exist. Nowadays Punctuation can be oratorical or
grammatical, sentences can be short or long. The amount of punctuation in the text can reveal "how writers view the balance between spoken and written language". In addition, page-based poetry compensated for the lack of a performer by exploiting layouts, so the appearance of the poem can be significant or irrelevant. Multiple styles of usage in a single poem can look
like inconsistency.

Let's look at a few examples. Here's  Stanza 2 of "anyone lived in a pretty  how town" by ee cummings.
\begin{verbatim}
  Women and men(both little and small)
  cared for anyone not at all
  they sowed their isn't they reaped their same
  sun moon stars rain
\end{verbatim}

Lack of punctuation may create ambiguities and encourage the passive reader to become active, but this piece creates few problems - the structures are simple and most of the missing punctuation is at line-endings. Short punchy paratactic phrases can survive a lack of punctuation as can a list of single words. Here, in contrast, is part of "Requiem for the Plantagenet Kings" by Geoffrey Hill
\begin{verbatim}
  Relieved of soul, the dropping-back of dust,
  Their usage, pride, admitted within doors;
  At home, under caved chantries, set in trust,
  With well-dressed alabaster and proved spurs
\end{verbatim}

The grammatical-logical punctuation here is needed to reduce the readers' confusion when parsing the long constructions.


It's a shame in a way that we have so few punctuation characters. Back in 1941 Frost wrote that "Poets have lamented the lack in poetry of any such notation as music has for suggesting sound." Some attempts have been made
\begin{itemize}
\item Hopkins used stress-accents
\item Dickinson's dashes may have lengths that correspond to the intended length of pauses

\item Olson used fixed-width fonts - "It is the advantage of the typewriter that, due to its rigidity and its space precisions, it can, for a poet, indicate exactly the breath, the pauses, the suspensions even of syllables, the juxtapositions even of parts of phrases, which he intends. For the first time the poet has the stave and the bar a musician has had."

\item Cummings - In ``The Deviated Use of Punctuation
Marks in Poetry: A Stylistic Study'' Saleema Abdul-Zahra points out that ee cummings exploited parentheses in particular. Following Tratakovsby, Abdul-Zahra classifies the usages, amongst them \textit{Iconicity} (their shape), \textit{Protection and intimacy}, \textit{Plural layers and Framed poem}, \textit{Hetero-glossia and Interpolation}, \textit{Subverting Formal Expectations}, \textit{Temporality, simultaneity, Tmesis} (suspending narrative even in the middle of a word) and exposing a word within a word.

\end{itemize}

More advanced notations exist, used by linguists transcribing soundtracks, and by poetry theorists.
Apart from these, line-breaks are the only non-prose way to indicate stress and pauses (which in the past was indicated with the aid of meter and missing beats). Colors, font-changes and animation could be used, but experimental poets often shy away from such methods. Consequently line-breaks have become rather over-burdened with significance.



If poets want their printed poems to be like a musical score rather than a visual feast they might sprinkle punctuation, italics and bold face  around and use line-breaks like an auto-cue script does. At the other extreme, there are poets for whom the page is the master version of the poem and for whom grammatical correctness is paramount. Then there are those - perhaps the majority - who are in between, who sometimes use a line-break to denote a pause, and sometimes use a comma
(depending perhaps on the visual effect). As usual, the issue ends up being a flexible negotiation between reader and writer, between expectations and intentions.


I think punctuation will tend to be used by those who
\begin{itemize}
\item optimize the use of paper (on expensive parchment line-breaks were sometimes
  replaced by a mark)
\item wish to minimize reader processing time and misreadings
\item follow the principle that deviations from the norm are conspicuous, adopting prose punctuation on the  assumption
that this will be unobtrusive whereas missed punctuation would be distracting.

\end{itemize}

Punctuation-free poetry isn't uncommon. "Physical" by Andrew McMillan (Cape Poetry, 2015) has no punctuation. Tom Paulin's later work uses no punctuation except dashes. I think line-breaks will be preferred by those who
\begin{itemize}
\item follow the aesthetics of Puritan economy of ink
\item don't want to use a fussy range of punctuation symbols when line-breaks can replace them all, albeit at the possible cost of readability
\item use  line-breaks because they think that the more poetic their work looks, the more poetic it is
\item want to destablize language
\item write shape poems or other poems where the position of words on the page matters significantly
\end{itemize}


 The reader response will depend on
\begin{itemize}
\item what type of punctuation  the  reader will assume is "normal" in the context (why shouldn't poetry have its own norms?)
\item the complexity of the sentence structure
\item the assistance that the line-breaks offer
\item the willingness to accept ambiguities and back-track after mis-parsings
\end{itemize}

I prefer using grammatical-logical punctuation on the grounds that it's conventional hence invisible. It gives me the freedom to use sophisticated sentence structures and I don't want to make the reader's life any harder than necessary. I give the poem's visual appearance a low priority. But in some short-lined, end-stopped poetry I ditch
  punctuation entirely.

\subsubsection*{See Also}
\begin{itemize}\item \url{http://litrefsarticles.blogspot.co.uk/2011/01/notation-in-poetry-and-music.html} Notation in Poetry and Music.
\item Debra San's "Literary Punctuation: A Test" in Literary Imagination 8.2 (2006) spends a page on the difference between
\begin{verbatim}
 I formed them free, and free they must remain
Till they enthrall themselves
\end{verbatim}

and
\begin{verbatim}
 I formed them free, and free they must remain,
Till they enthrall themselves
\end{verbatim}

from \textit{Paradise Lost}, then tackles Pound, Shelley, Whitman and Dickinson with similar thoroughness.
\item \url{https://www.youtube.com/watch?v=KBb161CXBMw} "Poets' Punctuation": a talk by Charles Lock | Boston | 2015


\end{itemize}


\newpage\subsection{Notation in Poetry and Music}

Poets rarely use more notation than prose provides. Hopkins used stress-accents and Dickinson's dashes may have lengths that correspond to the intended length of pauses, but apart from these, line-breaks are the only non-prose way to indicate stress and pauses.


More advanced notations exist, used by linguists  transcribing soundtracks, and by poetry theorists. Poetry theorists less often work from performance, marking up stress using purely the text, and breaking lines into "feet" - all of which makes
their work rather subjective, and easy to manipulate to support their theoretical viewpoint.



Of diacritical marks the most commonly used have been 
\begin{itemize}
\item the acute accent  for primary stress, stress in general, or "ictus"
\item the grave accent for secondary stress
\item the macron  to indicate a "long" syllable 
\item the breve (u-shaped marking) for a "short" syllable
\item dots above or after the symbol to indicate duration
\item the caret for pauses or omissions. Relative duration is indicated by addition of macron or breve or dot(s)
\item  | to separate feet 
\end{itemize}


In rhythmic analysis, diacritical marking of normal text is less satisfactory for most purposes than some kind of graphic transcription, extracting essential features
\begin{itemize}
\item Letters of an alphabet were used to represent prosodic values in ancient Greek prosodic (and musical) notation and in the ancient Sanskrit Chandahsutra of Pingala (where \textbf{G} = long or heavy, guru, \textbf{L}= short or light, laghu. Pingala used single letters also to represent systematic combinations of these values, or "feet": \textbf{M} =GGG, \textbf{N} = LLL, \textbf{R} = GLG, etc.); 
\item More recently in English, 
\begin{itemize}
\item \textbf{x} (or \textbf{.}) = unstressed, \textbf{a} (or \textbf{+} or \textbf{/}) =stressed. Iambic pentameter may be represented as

x / \textbf{|} x / \textbf{|} x / \textbf{|} x / \textbf{|} x /
\item \textbf{S} = stressed, \textbf{O} = unstressed, \textbf{L} = light stress, \textbf{p} = short  pause or replacing a light syllable, \textbf{P}= long pause, or replacing a stressed (following G.R. Stewart, Technique of English Verse, 1930) 

\end{itemize}
\end{itemize}

Less often, numbers have been used to indicate
stress or pitch. Notations to support other features exist - e.g where 
long/short syllables (classical Greek and Latin poetry) or pitch (Chinese
poetry) matter.

\subsubsection*{Music}

Serious music tends to have been written before played, written in measures even if one can't hear them. "traditional" music often gets written down after it's been played. The choice of notation matters to the composer and the archivist of "traditional" music. It also matters to experimentalists and theorists. Quoting from \url{http://encyclopedia.laborlawtalk.com/Musical_notation} ``Musical notation'' - according to Philip Tagg and Richard Middleton
\begin{itemize}
\item "musicology and to a degree European-influenced musical practice suffer from a 'notational centricity', ... a methodology slanted by the characteristics of notation
\item Notation-centric training induces particular forms of listening, and these then tend to be applied to all sorts of music, appropriately or not
\item Notational centricity also encourages reification: the score comes to be seen as 'the music', or perhaps the music in an ideal form."
\end{itemize}

Before music notation became standardised, several methods were tried. After standardisation the number of performance cues increased, but the notation remains restrictive. Alternatives have been developed. Graphic notation refers to the contemporary use of non-traditional symbols and text to convey information about the performance of a piece of music. It is used for experimental music, which in many cases is difficult to transcribe in standard notation.

\subsubsection*{Fusion}


Conventional musical notation has often been adapted for rhythmic analysis of verse. For English, occasional and partial use of musical notation begins with C. Gildon (Complete Art of Poetry, 1718), and recurs frequently in the later 18th and 19th century; since S. Lanier (Science of English Verse, 1880) full musical notation has often been used by writers whose analysis of verse is musical or exclusively temporal. Poe, for example, used it. At the very least, duration of pause may be indicated by conventional musical notation for rests.


Music scores are for performers rather than receivers. In poetry, receivers "perform" the poetry when they read it.
Music notation tends to restrict composers. Extra poetry notation would in
contrast restrict readers/theorists. It's unclear to me why so many poets
with widely varying views on poetry allow such liberal interpretations of
their work.
Given the increasing specialisation of of the poetry audience, the paucity of auxiliary notation is surprising.
\begin{itemize}
\item  \textit{Phonetics} - The text doesn't always unambiguously represent the sounds. The accent of the author may well matter. Even between standard English
and standard American there are many pronunciation differences. If sonic
effects mattered to the poet, one might expect some indication of how
the words should be said. Composers usually indicate the instrument
they're writing for; poets rarely suggest that (for example) a piece should
be rendered with a Geordie accent.
\item  \textit{Pauses} - Music notation has clearly defined pause lengths. One can
sometimes use metre to calculate the length of pauses.  In the hands of William Carlos Williams the typewriter offered a way to produce accurate spaces, though
there's little to suggest that the relative length of these spaces mean much. 
Line-breaks are coarse indicators
\item  \textit{Speed} - Should a poem be read slowly or quickly? Does relative speed
matter? Again, line-breaks can be used - short lines are supposed to slow readers down.
\item  \textit{Loudness} - crescendos could easily be rendered.
\item  \textit{Emphasis}  - bold/italic fonts along with upper-case are used, though
newlines are more common.
\end{itemize}

Given the resistance to other notations, it's not so surprising that 
the line-break is so popular - a single tool with many uses. However, it 
performs none of them very precisely.

\subsubsection*{Interpretive Freedom}

An interpretion needn't follow the author/composer's intentions. Sometimes
(e.g. when actors in Macbeth dress as Nazis) it's done deliberately. 
Elsewhen (e.g when old music isn't played on period instruments) the
decision may not have been consciously made. Some of these choices are
controversial. Poetry readers have at least as much freedom as performers
in other Arts. At poetry recitals audiences don't seem to mind an
English male performing an American woman's poetry as long as the subject
matter isn't too incongruous - sounds can be changed ("tomato" and "Z"
pronounced differently) as long as words aren't. 

Private readers may well make 
many unconscious (and uninformed) decisions. Perhaps it's assumed that
such misreadings are of minor consequence compared to conceptual
misunderstandings, and that it's not worth cluttering the visual 
impact of the text to reduce the errors. The "look" of the text matters,
especially when expressive line-breaks are used.


\subsubsection*{WWW}

Nowadays it's nearly as easy to have sound online as text. Even movies 
aren't too hard, with optional subtitles, like a DVD. Perhaps the age of
"The Master Text" is coming to a close. 


Even restricting oneself to text, I don't think one need tie oneself down to 
a minimalist notation, especially now that color and graphics are cheaply 
reproduced. 

\newpage\subsection{Gappy poetry}
In ``Tears in the fence'' No.59 Spring 2014, Mark Goodwin's ``Mind Will'' begins with
\begin{verbatim}
wind th    rives in sky's grasp the   wind
ing of cloth pulls   the sky's hear   t open

and takes the p ush of clouds \& distant
land into the text   ure of corn's matt talk
\end{verbatim}

Some of these gaps are between words, denoting a pause or break that's rather less than a line-break (they sometimes replace commas, or they're like the gaps in anglo-saxon verse). Poets not usually considered avant-garde sometimes use such gaps too - for example, Kim Moore's ``Some People'' (in ``The Art of Falling'') and  Liz Berry's ``Bird'' (in ``Black Country'') use them.  The gaps within words are more challenging. ``In Tears in the Fence'' (No. 65) Mark Goodwin writes about what he describes as his gappy poetry - ``with the development of the gappy poetry there was never really an aim, not an an intention, certainly not to begin with. It was all about play''. I can see that the gaps allow a little Joycean wordplay, bringing out new meanings. There's disruption too, stopping the reader using a standard novel-reading method of processing - letters rather than words need to be processed, and the 2nd line's ``ing'' will cause most readers to backtrack. In the 4th line, readers are likely to sense ``text of [the] talk'' and ``corn stalk''.

Having written the poetry, he later thought about the style -

\begin{enumerate}
\item
as I thought about it, and tried to analyse what I was doing, and no doubt also constructed reasons for what I was doing, I began to see, especially as most of my gappy poetry is concerned with landscape and place, that this gappy form has much to do with the way we continually attempt to read and reinterpret the layers of our worlds … and how we get from one layer of landscape to another, how we go over horizons, how we get from one valley to another. We go via gaps, gaps in hedges, or via a col, or pass, or gap between mountains. What is a path but a form of gap - a strip of place where material has been worn away

\item words on the page are governed by the gaps … By re-arranging the gaps, and adding gaps, you are still left with the base-layer interpretation/score/material of the original poem (or landscape) before the procedure was imposed on it, but you also have a new surface that reveals, from a new angle, or point of view, or position of hearing, some of the pure sound or music of language detached from what we usually experience as familiar speech, and also you get sudden shifts in meaning, all generated by moving or adding space(s)

\item Perhaps I can say that my gappy poems happen to be a particular pattern at a particular time, but that the gaps can lead on or rather invite another reader-maker to break the poem (or even world) down again, and re-map as they will
\end{enumerate}

So,

\begin{enumerate}

\item gaps aren't necessarily a lack of something, they can be a means of going from one place to another (a corridor, an airlock) or a space waiting to be filled - he writes ``I often approach writing poetry … much as a painter might approach a canvas, and also in some ways how a dancer might approach and proceed across a floor'')
\item gaps can destablize the readers' attempt to organise their sensory input into layers. It does so without making the original text inaccessible
\item gaps offer the reader more interactive play between the parts
\end{enumerate}

These aims aren't new of course.


\subsubsection*{Influences/precedents}
\begin{itemize}
\item \textit{Line-breaks} - Much of what applies to using line-breaks also applies to gaps - if poetry is cut-up prose, then gappy poetry is cut-up poetry.  Linebreaks have lost their power to disrupt, but gaps haven't. Rosmarie Waldrop wrote that ``Perhaps the greatest challenge of the prose poem (as opposed to ``flash fiction'') is to compensate for the absence of the margin. I try to place the margin, the emptiness inside the text. I cultivate cuts, discontinuity, leaps, shifts of reference, etc. 'Gap gardening,' I have called it''

\item \textit{Mimetic expression} -
Olson proposed that nuances of breath and motion to be conveyed to the reader through typographical means. Gaps are an obvious means.

\item \textit{Negative space} - rather than looking at the gaps, one can look at the clusters thus formed, and consider how the gaps create a framing effects -  ``Typographic isolation does not ``emphasize''; it frames, rendering a familiar word or phrase momentarily unfamiliar ... Emphasis limits the range of possible meanings'', Stephen Cushman, ``William Carlos Williams and the Meanings of Measure'', Yale, p.60

\item \textit{Clustering} - words are in sentences, grouped into clauses determined by syntax. The freedom to move words around on the page provides a different way to create word clusters. The idea is similar to the use of phonemes - matching phonemes can be spread amongst words not connected by syntax - or how a realist painting can sometimes be viewed as an abstract - a ``study in blue'' where the blueness by chance belongs to various objects.

\item \textit{Disruption} - 
In ``Broken English'' (Wesleyan Univ Press, 1993), Heather McHugh points out that  ``[Poetry] is a broken language from the beginning, brimming with non-words: all that white ... the making of lines is the breaking of lines ... All poetry is fragment: it is shaped by its breakages, at every turn ... The poem is not only a piece, like other pieces of art; it is a piece full of pieces''.


In informational prose, letters or sounds create words which in turn have meanings that a sentence organises into higher meanings. Disruption of this mode of comprehension can remind people of the arbitrariness of spellings, classifications, etc. of course, it's not a new idea - e.g. ``The radical indentations [in ``Tintern Abbey''] let space into the verse column at irregular interval, signaling the abrupt discontinuities and shifts associated with the Romantic ode'', Stephen Cushman, (``William Carlos Williams and the Meanings of Measure'', Yale, p.57). But such disruption seems to me the moon/June, love/dove of modern poetry - easy to do, with a deadened effect through over-use.

That said, Ralf Webb writes of Emily Berry's poems that ``Several of these use “tabulation”; large blank spaces appear mid-line, as if the poems’ sutures had been ripped out, creating irregular, staccato, breathless rhythms, so that to read them is to enact and experience the urgency of the speaker’s appeals'' showing how gaps can produce a different effect to that of line-breaks.

\end{itemize}


\subsubsection*{Readings}
''Spacing in poetry is nothing to do with space and everything to do with time'' wrote John Fuller in ``Who Is Ozymandias? and other puzzles in poetry'' (Chatto and Windus, 2011, p.22). When poetry is read out, it can be hard to hear where the line-breaks are. Nevertheless, poetry readings often successful, bringing into question the importance of the line-breaks. What about gaps?

Mark Goodwin wrote ``Encountering my gappy poetry as a reader is of course completely different to encountering it as a listener, and not least because when performing the gappy poems to a live audience I tend to read the poems twice, in different ways: once honouring the gaps, reading in a very clipped style; and once reading the poem \textit{without} honouring the gaps, and generally by reading in a more natural way. I've found that an audience that might be driven away by the appearance of the gappy poems on the page, once they've heard the two differing but connected musical versions of a gappy poem, well, they 'get' it. That's not a 'get it' so much to do with 'understanding', but rather they 'receive' the music''

I'd like more evidence, but it's an interesting observation.

\subsubsection*{Mind the gap?}
End-rhymes add an effect, but usually at a cost. If gaps offer extra effects without consequences we'd all be using them (in the way that poets use line-breaks nowadays, there being nothing to lose).  While we're at it we might as well use coloured text and multiple fonts - they too add meaning-laden features without destroying the original. But we don't.

When New Formalists write about the positive effects of rhyme they usually don't mention disadvantages. Similarly it's not surprising that those who promote gaps don't list their disadvantages - 
\begin{itemize}

\item they distract the reader from considering ``meaning'' (visual effects are often considered more superficial than ``meaning'')
\item the use of gimmicks (uncommon notation) can put readers off (gaps aren't common - they still seem rather quirky) 
\item the use of apparently random devices can put readers off
\end{itemize}

Even some practitioners have doubts. In \url{http://thequietus.com/articles/12182-emily-berry-dear-boy-interview-sam-riviere} an interview Emily Berry says that ``These spaces appeared initially as a way of indicating a kind of stutter or inability to speak/write except in a fragmented way (which is probably a textual representation of how I feel when trying to talk about emotions!). They started appearing in other poems I wrote, mainly the ones about dealing with absence and I guess you could also see them as symbolising the way in which someone’s absence can seem so physical. I don’t seem to be using them so much any more. I like the way they give you a bit more freedom in terms of line breaks, but they’re also quite annoying to work with, you end up spending ages deciding how many spaces a particular gap should be, which is not the greatest use of one’s time''. 

In ``Tears in the fence'' No.65 Mark Goodwin's ``Mind Will'' is quoted from with different gaps. Perhaps I've miss-typed.


\newpage\subsection{Juxtaposing}
Juxtaposition happens in all texts, but sometimes the lack of continuity is more extreme than usual. At its purest (in a translated oriental poem, perhaps) we might be offered 2 one-word sentences. Assumptions readers might make to connect the words include
\begin{itemize}
\item Equality (or analogy)
\item Opposition (disjoint options)
\item Sequence (temporal - perhaps one turns into the other)
\end{itemize}
When we come to a fracture in a longer text (between paragraphs, chapters, etc) we still try to make a connection between the parts. The way we do this will vary according to the type of text we're reading, but typically I suspect we first assume that the text is jumping ahead in time, leaving a gap that will be filled in later. Then perhaps we might think it's a flashback, or a parallel storyline that will be revisited. Only as a last resort do we concede that there may be no causal connection or character continuity.

One of the features of conventional text is that writers have to present things sequentially even if they happen simultaneously or independently. In such situations where narrative breaks down, the terms Montage and Collage become useful. Both describe a non-hierarchical way of incorporating diverse fragments producing a multicentred work. Montage more often concerns assembly using things that have already been ``created''. Collage is more often used when material of different types is used together and when there's no particular common theme. Gregory Ulmer described collage as ``the single most revolutionary formal innovation in artistic representation to occur in our century''. This may be because it cuts across the long-cherished Aristotelian notion of organic unity, where each component of a work is a necessary part of a whole. Max Ernst claimed that ``Collage is a hypersensitive and rigorously exact instrument, a seismograph capable of registering the exact potentialities of human welfare in every epoch''.

In relation to poetry, David Antin remarked ``for better or worse, 'modern' poetry in English has been committed to a principle of collage from the outset''. With collage in particular, use is made of the difference between the source/material of the fragment and the meaning in the context of the whole - the observer is expected to bob up and down between surface and depth. Flexibility along this dimension is more characteristic of poetry reading than prose reading.

There are likely to be connections between parts, but they may be more to do with surface than meaning - leitmotifs without a plot. In ``Thirteen Ways of Looking at a Blackbird'' there's a common theme. In ``The Waste Land'' the links are more tenuous. In other works fragments are only related in that they each mention a red dress, or an accordian, or have someone shouting ``Damn''. These latter relationships can seem gratuitious, leading to ``washing line'' pieces (where the only point of the connection is to have somewhere to hang the pieces from) but this is to devalue the surface, which in collage is more relevant than usual. Also such connections make the lack of causal connection more palatable. Sometimes there are traces of a narrative thread in a few of the fragments. There's no narrative impetus or suspense though. Some consider ``thematic interplay'' the poor man's ``conflict and dynamism'', a ``compare and contrast'' task that requires too much from the reader and masks the authorial persona.

The idea of a decentralised network of ideas has been described by Deleuze and Guattari ('rhizomes') but of course goes back much further than that, beyond Wittgenstein's ``family resemblances'' - ``The governing principle of much Persian poetry is circular rather than linear; rather than a logically sequential progression, a poem is seen as a collection of stanzas interlinked by symbol and image - the links being patterns of likeness and unlikeness, of repetition and variation - which 'hover', as it were, around an unspoken centre'' (Glyn Pursglove, Acumen 25, p.9). Parataxis replaces syntaxis - ``the dethronement of language and logic forms part of an essentially mystical attitude towards the basis of reality as being too complex and at the same time too unified, too much of one piece, to be validly expressed by the analytical means of orderly syntax and conceptual thought'' (Martin Esslin, ``The Theatre of the Absurd'', 1962.)

On a small scale, juxtaposing can happen on a line and can be read as an implicit (though perhaps surreal) simile. ``In Surrealist metaphor, two terms are juxtaposed so as to create a third which is more strangely potent than the sum of the parts ... The third term forces an equality of attention onto the originating terms'', ``Statutes of Liberty'' (Geoff Ward, Macmillan, 1993, p. 73-74). More generally, ``The meaning-relationship is completed only by the simultaneous perception in space of word-groups that have no comprehensible relation to each other when read consecutively in time ... modern poetry asks its readers to suspend the process of individual reference temporarily until the entire pattern of internal references can be apprehended as a unity'' (David Lodge, ``The Art of Fiction'', p.82)

On a larger scale, sentences can come alternately from 2 fields - in 'The naming of parts' for example, the reported speech and internal thought alternate. 'Moby Dick' and 'USA' (Dos Passos) inserted non-fictional fragments. Found text can be inserted randomly into a poem, or fragments of different kinds of poems (rhymed and free-form) can be interspliced using a variation of Burroughs' cut-ups technique. Bakhtin's carnival and polymorphism can come into play too.

\subsubsection*{See Also}

\begin{itemize}
\item Collage and post-collage (Pierre Joris)
\item The Collage Poetry Mail Group - from the International Museum of Collage, Assemblage and Construction (IMCAC)
\end{itemize}


\newpage\subsection{Poetic leaps}

Matthew Stewart pointed out on his blog that "the success of a poem often hinges on whether its pivotal syntactic leap makes to the other side of a semantic abyss". He went on to write that the connections "must be unexpected, revelatory and inevitable once made".

I've been interested for a while in such leaps. Here I'll mention a few of their uses and dangers.

\subsubsection*{Size of leap}

When the reader makes a leap there's often satisfaction - like solving a riddle. Each simile or metaphor is a leap, which can vary in degree of difficulty.
\begin{itemize}
\item    When Amy Clampitt writes that a cheetah's lope "whips the petaled garden/ of her hide into a sandstorm", the reader needs to imagine what an accelerating cheetah might look like. All the clues are provided
\item    When Joni Mitchell begins "Blue" with "Songs are like tattoos" the reader needs to work harder. The follow-up phrase "You know I've been to sea before" only partly helps.
\end{itemize}

Bronowski in "Science and Human Values" suggested that ``every act of imagination is the discovery of likenesses between two things which were thought unlike". To some extent the bigger the leap, the more the eventual pleasure. The risk of a big leap is that some readers won't be able to manage it. The leap may require readers to be more imaginative than when they merely identify likenesses, they may need to invent something new - "In Surrealist metaphor, two terms are juxtaposed so as to create a third which is more strangely potent than the sum of the parts ... The third term forces an equality of attention onto the originating terms", (Geoff Ward, in "Statutes of Liberty").

The size of the leap also affects the reader's sense of speed. Rapidly completing the gaps can be exhilarating, even if the leaps aren't unexpected.

\subsubsection*{Types of leap}

Here are just a few -
\begin{itemize}
\item      \textit{The end-of-poem lift} - A sudden leap out of the poem to the world is common as a conclusion. Also common at the end is a bigger than usual leap between 2 images (Larkin's "somewhere becoming rain"). Haiku typically end with a leap.
\item      \textit{The turn/volta} - Common in sonnets
\item      \textit{Zoom-out} - jumping outside the frame or context.
\item      \textit{Inferences} - Gaps may need to be leapt by making an inference. If someone replies to the query "Does Jack like porridge" by answering "All Scots like porridge!" it's reasonable (but not strictly logical) to assume that Jack is a Scot who likes porridge.
 \item     \textit{The stage dive} - The poem hopes that the reader will offer support, otherwise the poet will fall flat on her/his face.
\item      \textit{Description} - When the argument or narrative stops and a scene or painting is being described, there may not be much significance to the order of the statements. Gaps may form between the statements. At the end of the description the reader will assemble the scene, having tolerated a localized lack of continuity on the understanding that it won't last long.
\item      \textit{Conventional leaps} - Readers will accept a phrase like "Years later" without feeling the need to plug the gap.
 \item     \textit{Conversational leaps} - If you listen to someone having a conversation on a phone, you might be tempted to fill in the gaps. Some poems are similarly one-sided.
 \item     \textit{Unfillable leaps} - It's natural for readers to seek a connection between successive images, but sometimes 2 images are just 2 images.
\end{itemize}

A few of these are situations where there's a shared understanding between 2 parties that makes their communication between each other difficult to understand for third parties. A context shared by the poem and the reader assists the readers efforts to leap more confidently.

\subsubsection*{Failed leaps}

Any leap is likely to involve work. If the reward is nearby and guaranteed, readers are likely to make an investment. If sometimes they fail, little is lost. More problematic are the situations where gratification is delayed - readers may have to keep many loose ends in mind awaiting a final resolution. This might not happen until the end in poems that are more spatial than linear - see Linear/Spatial Form ("modern poetry asks its readers to suspend the process of individual reference temporarily until the entire pattern of internal references can be apprehended as a unity" - J. Frank).

But what about passages like the following? - "down the chain/ let play with money/ ahead in/ undeclared war/ full employment/ keep the silver clean/ or die/ intrude into metaphor". It's the beginning of "allowed to complain" by Tom Raworth. Maybe given enough effort it could be conventionally resolved, but it could equally be a random cut-up. Looking upon the poem as a sequence of failed leaps is probably inappropriate.

\subsubsection*{Types of reader}

It's impossible to make the leaps appropriately "unexpected, revelatory and inevitable" for all readers. Poets tackle that issue in various ways - by not caring about reader variation; by offering notes; by offering alternative ways of reading, etc.

When Disney animations were hand-made, the master artists drew the key frames (the "keys"), leaving assistants to complete the frames in between (a job they called "tweening"). If apprentice artists could tween, why not knowledgeable audiences? Poetry has such an audience. But there are consequences to sacking the tweeners -
\begin{itemize}
\item    Suppose people tween differently? As long as the distance between keys isn't large, there shouldn't be problems. The keys act as checkpoints so that people can resynchronise if they feel they need to
\item    If the distances become too large, some readers might lose the narrative thread. Consequently there's a tendency for each key scene to become more self-contained, the keys becoming a series of disconnected tableaux - a triptych, a gallery.
\end{itemize}
A common way of tweening in literature is to supply a supportive context of backstory, motivations, or justifications - in short, telling rather than showing. The amount of this varies according to the style. In the TV series "The Wire" there's little "telling"; the writers decided that all sound had to be sourced - no voice-overs and no background mood music. All music had to come from a car radio, an open tenement window, etc. Some poetry has a similarly purist approach, using juxtaposed images to keep "telling" to a minimum. The risk is that such poetry becomes a game of charades, a dumbed-down mime-show. Complex arguments are difficult to show, concepts like fate harder still.

Sometimes the "telling" (the interpretation, the moral) is only at the end, though this is rather unfashionable nowadays. One way to convey the information without despoiling artistic purity is to employ metalepsis, making it hard to distinguish between the "show" and "tell" elements. A cinematic example would be for there to be a voice-over scene during which a character walks into the frame speaking the voice-over.

%\begin{figure}[htbp]
\begin{wrapfigure}{r}{0.3\textwidth}
\centering
\includegraphics{rupert.jpg}
\end{wrapfigure}
%\end{figure}

Another, more reader-friendly approach is that adopted by the Rupert annuals. The Rupert Bear stories began as a newspaper cartoon strip, but soon became better known for the annuals. The page layout supports several reading modes. Each page has the story title at the top. Beneath that there's a page subtitle. Young children can follow the pictures and get help filling the gaps. Each picture has a rhyming couplet beneath it - e.g. He meets Pauline, and straight away/ He tells her all he has to say. At the foot of the page is prose which fills in less obvious gaps - Rupert and Snuffy run towards the tent. Pauline is the first Guide he meets and he pours out his story. People can read the verse, the prose or both.

An entertaining exercise is to take a poem (by Larkin, say, The Whitsun Weddings) and give it the Rupert treatment, pictorialising the imagery (at 1.20pm on a sunny day, a quarter-full train with all its windows open leaves a city station), adding sub-titles to describe how none thought of "how their lives would all contain this hour". Trying the same exercise with Larkin's "Toads" would yield a very differently proportioned layout. I suspect that with some poets their poems would all have the same proportion of text to pictures.



\newpage\subsection{The Sound of Poetry and the Poetry of Sound}

 With the rise of isms (deconstructionism, eco-feminism, post-colonialism) in recent years, literary theorists have rather neglected sound effects, often quoting Saussure's view that the sounds of words are arbitrary.

But they're not. Of course we've always known that chickens cluck and cows moo, but the influence of sound goes wider and deeper than that. We 'chip little bits' but 'chop logs'. Twigs are small; trunks are big. There are exceptions (big should refer to little things, and bugs should be big) but words derive from many sources and we should expect some exceptions.

The more that these trends are studied the more universal they seem - \textit{petit}, \textit{piccolo} and \textit{klein} contrast with \textit{grand}, \textit{grande} and \textit{gross}. And the trends go beyond simple concepts like size. With some poets it's possible to guess the theme of the work without understanding a word of it by calculating the relative proportion of sounds - the guess isn't always correct but I'm amazed that it's possible at all. People have tried to create dictionaries of sound meanings. Here's an extract about the L sound from Galt's book - Positive skews in love poems and narratives: strong positive skews in "tender" and "musical" poems. Negative skews in poems of family and home, nostalgia, and humor, with a negative skew for "non-musical" poems which is just below the level of significance. This phoneme certainly distinguishes, in Storm's verse, between "musicality" and its opposite, and its presence can evidently also contribute to a feeling of "tenderness"

If isolated sounds aren't arbitrary, still less are the sounds of sentences and poetry whose patterns produce effects that isolated words can't. Derek Attridge in "Peculiar Language" calls them nonce-constellations, writing that "The operation of nonce-constellations is probably more significant than genuine phonesthemes in onomatopoeic effects", citing John Hollander. In Violi's book Haj Ross spends 40 pages on the sounds in "The Tyger" pointing out dozens of features such as
\begin{itemize}
\item    The sound F only occurs on even-numbered lines, and gangs with R.
\item    while all the words in the Tyger line except one are bisyllables, this line being the most polysyllabic of the whole poem, all of the words in the Lamb line are monosyllabic
\end{itemize}

These effects are in addition to the regular patterns of stress, rhyme, etc. However, with free verse these dispersive patterns are beginning to dominate. We lack the vocabulary to describe them well, and I suspect they often go unnoticed (at least consciously) by readers. Here's an extract by Ruth Padel where she describes an easily missed pattern in Michael Longley's \textit{Ceasefire} -
\textit{Achilles, the key name, appears in every stanza. Its central syllable is repeated in the first stanza ("until", "filled", "building", with a sideways echo in "curled" ...), reappears in the second, resonates in the third with "built" and "still" (plus an echo in "full"). and reaches a climax in "killer": bringing out the fact that "Achilles" has the sound of that word "kill" in his name}

Interest in sound effects has revived because 1) computers can now analyse a lifetime's work in minutes; 2) brain-scanning has enhanced our understanding of music's effects; 3) the study of pragmatics has attracted attention to the non-semantic effects of words. If music can be profound, why not the sound of words? It too uses repetition combined with variation. It too has incantatory power. The semantics can modulate sound's meaning much as the choice of instrument can affect music's meaning. A violin's C isn't the same as a trumpet's, just as oo is recognisable but differently received in moon and spoon.

Though we may never return to the clogged tongue-twisting of William Barnes'
\begin{verbatim}
       With fruit for me
       The apple tree
       Do lean down low in Linden lea.
\end{verbatim}
we might hope for more tolerance of poets like Dylan Thomas and Wallace Stevens. When they stop making sense perhaps they're not lapsing into non-sense but instead bringing out the tonality of words, an alternative mode of meaning making sense an echo to the sound. 

The significance of these patterns is unclear. In "Choosing between sound and sense" I quote from people like Bunting for whom sound was a generator of meaning, and from people like Valéry for whom sound was important but independent of conventional meaning.

 Here's part of a review by Forrest Gander of Jorie Graham's "The Scanning" (Boston Book Review, Summer 1997) \textit{We hear first the echo of "kiss" in "its" and "mathematics". But even before those three notes are reinforced by "hiss", "missed," "distance," and "pianissimo," Graham introduces a counterpoint, the growling consonance of "glint," "gripped" and "glides" and the long o's of "show" and "over". Look how the word "show" recollects the second syllable of "harrowing" from the second line, and prepares our ears for the deep vowels in "pianissimo," "telephone,"}

Eliot's
\begin{verbatim}
Arms that are braceleted and white and bare
(But in the lamplight, downed with light brown hair!)
\end{verbatim}
is described thus by David Trotter ("T.S. Eliot and Cinema", Modernism/Modernity 13.2 (2006))
\textit{The intensity of Prufrock's arousal produces or is produced by an intensification in the verse. By comparison with its sparse and evenly paced predecessor ("and white and bare"), the line describing the hair on the women's arms seems positively swollen: the echo of "lamplight" in "light brown hair" and the internal rhyme on "downed" and "brown" fill it from within with the sameness of sound, with emphasis (p.243)}

This kind of criticism raises various issues
\begin{itemize}
\item    whether all the perceived patterns exist - they may be the result of selective highlighting in the text. "Lit. crit. has a very bad record for selective quotation and selectively quoting supporting evidence while excluding all contrary data points." [J.C.]
\item    if these patterns exist, are they accidental (i.e. are they as likely to occur in non-literary language)? Texts (particularly literary ones) will have bunched patterns of sounds. For example, while writing, one's short-term memory will contain recent sounds which may encourage the further use of those sounds, thus leading to clumping (echolalia).
\end{itemize}
Computer programs might be used to help resolve these issues, though quite what output they should produce is unclear. A few years ago I wrote a program that counted fricatives, plosives, end-rhymes, etc. It did quite well at identifying sonnets but it couldn't report on sound clusters. It could begin to convert these sound patterns into graphics. The graph below was an early attempt, showing the concentration of I (the bottom surface), W and L (liquid) sounds in the 1st stanza of Gray's "Elegy Written in a Country Churchyard".
\begin{verbatim}
The curfew tolls the knell of parting day
   The lowing herd wine slowly o'er the lea,
The plowman homeward plods his weary way,
   And leaves the world to darkness and to me.
\end{verbatim}
\begin{figure}[htbp]
\centering
\includegraphics[width=10cm]{asp3.jpg}
\end{figure}

The furthest edge represents line 1, the nearest edge (along the axis that runs from 0 to 10) represents line 4. Note the humps on the top surface at line 4, syllables 2 and 4 corresponding to 'leaves' and 'world' in the text. Note also the long ridges on the top surface along the 2nd and 6th syllable marks - indeed, many of the 'L' sounds fall on stressed syllables. But such pictures don't show a landscape which corresponds to how the sounds affect me. Perhaps the graphs should emphasise stress and end-rhyme more than they do. 



\newpage\subsection{Repetition in Jon Stone's 'School of Forgery'}
Jon Stone frequently uses repetition - within and between his poems. His poems also repeat phrases from other sources. Repetition's a technique I rarely use, so I thought I'd use his "School of Forgery book to examine the effect.

\subsubsection*{Repetition}
According to Tannen, ``Repetition ... is the central linguistic meaning-making strategy, a limitless resource for individual creativity and interpersonal involvement'' (``Talking Voices: Repetition, Dialogue and Imagery in Conversational Discourse'', CUP, 1989).
Various technical terms are described on the Wikipedia page on rhetorical repetition (Anaphora - repetition at the start of lines, Epistrophe - repetition at the end of each clause, etc). The list of effects below comes from Al Filreis' Repetition page and elsewhere.
\begin{itemize}
\item \textit{Sound/ritual} - ``Primitive religious chants from all cultures show repetition developing into cadence and song'' (Filreis)
\item \textit{Providing structure} -  ``a refrain, which serves to set off or divide narrative into segments, as in ballads, or, in lyric poetry, to indicate shifts or developments of emotion. Such repetitions may serve as commentary, a static point against which the rest of the poem develops, or it may be simply a pleasing sound pattern to fill out a form.''  (Filreis)
\item \textit{Unifying} - ``As a unifying device, independent of conventional metrics, repetition is found extensively in free verse, where parallelism (repetition of a grammar pattern) reinforced by the recurrence of actual words and phrases governs the rhythm which helps to distinguish free verse from prose''  (Filreis)
\item \textit{Emphasis of the succeeding phrase} - ``Sometimes the effect of a repeated phrase in a poem will be to emphasize a development or change by means of the contrast in the words following the identical phrases''  (Filreis)
\item \textit{Indicating closure} - the final line being a repetition of the first or penultimate line
\item \textit{Generating expectation} - which can lead to surprize
\item \textit{Backtracking} - an indication that a path of enquiry has ended (failed), that one has to go back and try again
\item \textit{Habitualisation} - In ``Flesh and Blood Repetition and Obscurity in Gothic Poetry'' (Sara Deniz Akant,
Wesleyan University) it's suggested that a way of making the strange familiar is to repeat it - ``poetic repetition does not aim to provide the reader with a resolving grasp on something that is obscure, but rather to make its inherent obscurity a continual source of his pleasure.''

\item \textit{Sheer pleasure} - In ``Beyond the Pleasure Principle'' Freud wrote that ``repetition, the re-experiencing of something identical, is clearly in itself a source of pleasure.''

\item \textit{Emphasising the significance of context over content} - Because of context, the 2nd occurrence of a phrase won't have the same meaning as the 1st (in villanelles but also with ``and miles to go before I sleep'')
\item \textit{Contrasting with change} - In ``Nietzsche and Philosophy'', Gilles Deleuze distinguishes between Platonic repetition (which effects semblance and strives towards unity) and Nietzschean repetition (emphasizing divergence and difference). Repetition is part of the ``same/different'' binary that drives narrative. Narrative stands between repetition (where the text is the same) and random juxtaposition (where there's no repetition). Narrative keeps some things the same (the context, the characters, etc) while changing something else. The longer the sequence where the division between mutable and non-mutable remains stable, the more likely the sequence will be considered as narrative - foreground against background.
\end{itemize}

Monet and Warhol are amongst the artists who have produced series of similar works. Monet's paintings of haystacks and Rouen cathedral emphasise the differences. Warhol's repetitions sometimes dilute the image's meaning.

There are attendant risks. As with many rhetorical devices (but especially those used by preachers and politicians) repetition can evoke distrust in readers. Beginners use much repetition - once end-rhyme is rejected it's one of the easier ways to sound poetic, to carry on when you've run out of things to say. It might be merely verbose, unnecessary - first-draft scaffolding. It's used by people who used to write short poems but now want to write longer ones - each repetition is like a new start. It can give fragments spurious unity - the repeated pegs on a clothes line of imagery. It's a way to induce trance. It can degenerate into sing-song echolalia. Or it can be plain boring. 

\subsubsection*{Users}
In "Lexical Repetition in American Poetry" Alan H Pope points out that repetition is commonly used. In ``Ariel'', Plath uses reiteration in 23 of the 40 poems. In Stevens' ``Harmonium'' at least 26 poems use repetition (6 begin and end with the same line/stanza). Stevens' longer poems (and some of Eliot's) repeat a central argument or statement (each time with perhaps a different, more complete understanding).  Poets with oratory styles - Ginsburg in ``Howl'' for example - exploit repetition.

Helen Dunmore's "Glad of these times" uses it. ``The Art of Falling'' by Kim Moore (Seren, 2015) has much too. For example, the first 7 couplets ``In That Year''  begin with ``And in that year''. The 8th and final couplet is ``And then that year lay down like a path/ and I walked it, I walked it, I walk it''. It's also used by less mainstream poets - see for example ``We needed coffee but ...'' by Matthew Welton (Carcanet, 2009)

\subsubsection*{School of Forgery}
Jon Stone's book doesn't include standard forms where there's word repetition (sestina, villanelle), so the expectation/surprise aspect is missing. He doesn't seem to be a theory-purist - word-repetition is used in many different situations.
\begin{itemize}
\item ''Dojinshiworld'' is 7 \texttt{xaxa} stanzas, 4 of them beginning with ``We came to''.
\item ``The Mark'' is in couplets, the line-endings being ``emotion's/emotion'', ``absence/absence)'', ``hoodwink/something'', ``hiding/emotion'', ``something/emotion''.
\item Each line of ``Mustard'' ends with an anagram of the title.
\item Successive stanzas of ``Send in the Mink'' begin ``Send in the mink'', ``Send in the savage mink'', ``Send in the unsuitable mink'' 
\item The ``Near Extremes'' poems on p.3, 13, 20, and 29 are related by repetition - the first line of each is ``Where I came from it's the other way round''. They all have 2 5-lined stanzas
\item The ``Swallow'' poems on p.12, 19, 28, and 35 all have on their first lines ``know of nothing beyond the''. They all have 5 3-lined stanzas.
\item ''The Year Long Dress Rehearsal'' and ``All Year Dress Rehearsal'' have similar first lines - ``I'm going to be mad - my first major role'' and ``I'm going to be sorrowful - my first big part''
\item 4 of the 5 pairs of lines in ``III. Hurricane Polymar'' are of the form ``With a *** ***, Detective Takeshi,/ you become a futuristic ***."
\item 4 of the 7 lines in ``Far Dancing and ...'' (part II) begin ``I cure her,''
\item 4 consecutive lines of ``The Laughing Body'' begin ``They want to drink breast''
\item In ``Second-Hand Kite Feathers'' (part II) the 1st line of all 10 \texttt{aabxb} stanzas is ``Very highly recommended''
\item In ``Adcock Modulations'' each part has 2 stanzas of 4 lines. The 1st stanza begins with ``They'' and the 2nd with ``But s/he''
\item All 7 shaped poems begin with ``in which''
\item On p.72 is a shape poem where John Steed's umbrella is created from 11 repetitions of the word ``to''.
\item ''The Year Long Dress Rehearsal'' has 4 consecutive lines of the form ``Your ... kill(s) me'' and 3 consecutive lines that begin with ``That ticking which is my''. There's more (incidental?) patterned repetition - line 1 from the top and bottom both end with the word ``role''. Line 3 from the top and bottom both have the word ``lines''.
\end{itemize}

The layouts are standard. They use black text, one font-face, little italics, no bulletmarks, only 1 gap in a line (p.46) and no indentation (except for p.44-45 which has continuation lines, and the shape poems), so the words have to do extra work.

It's presumably not coincidence that 3 of the ``Swallow'' poems immediately precede ``Near Extremes'' poems, and that one of each sequence is unattached, but I can't see the purpose. Anyway, why weren't the ``Swallow'' poems put together, making them stanzas of a single poem, or a series of variations? It's done elsewhere in the book.

At times repetition helps hold these poems together (when there's little else that does). The repetition is never percussive, never has a [mock-] Churchillian charge. It helps provide structure, and may perform habitualisation. In pieces like ``Second-Hand Kite Feathers'' there may be a Warholian effect. Sometimes it helps emphasise the succeeding phrases, but in poems like ``III. Hurricane Polymar'' I don't get it at all.

As an experiment one could try removing repetition from some of these poems and adding repetition to some poems that currently lack it. These, and the originals, could then be tried out on new readers.

Al Filreis points out that ``Allusion or quoting is a special case of repetition''. Jon Stone uses that case too.


\newpage\subsection{The Poetic Voice}
%\begin{figure}[htbp]
\begin{wrapfigure}{r}{0.3\textwidth}
\centering
\includegraphics{emptystage.jpg}
\end{wrapfigure}
%\end{figure}
In \url{http://magmapoetry.com/archive/magma-52/articles/speaking-the-poems-voice/} (Magma 52)
Polly Clark wrote ``\textit{Anyone who has taken a creative writing course knows about ‘voice’. This is the elusive, essential, treasured characteristic of a poem on the page, the one thing we must have, as a poet above all else. We talk about ‘finding our voice’ and we know when someone has found it and when they haven’t.}''. And in ``The Writer's Voice'' Al Alvarez wrote that ``\textit{a writer doesn't properly begin until he has a voice of his own}''.


In pre-literate days, nobody experienced a poem without also hearing a voice and being in the presence of a speaker. However, by the time the New Critics had arrived, the inevitability of voice had gone. Paul de Man thought the tendency to seek a voice in lyric poetry ``delusional''. When people nowadays talk about poetic voice, I suspect several issues may become conflated - 
\begin{itemize}
\item \textit{Style} - Poets used to have styles, but now that we're the offspring of Romantics and Confessionalists, modern readers seem to want to construct the person behind the words. Alvarez thought of style and voice as two very different things - he wrote that ``\textit{in order to find his voice [a poet] must first have mastered style}'', but putative authenticity and distinctiveness seem to be what makes a style into a voice for him. For me, some styles associate easily with personality types and are more likely to be described as voices.
\item \textit{Therapy} - Polly Clark points out that ``\textit{Many of us became writers because we were silenced in some way, and the written self on the page speaks more authentically than we do as individuals}''. Once poets use poetry for self-exploration, their style will become more of a voice.
\item \textit{Authenticity} - Eavan Boland considers it more essential than ever that poets should discover ``a real voice, a true voice''. Clarke also writes ``\textit{A poet writing in their true voice can persuade you of anything, so authority is also an element}''. Believing that the poet's using their own voice removes some of the obstacles and distrust that hinder communication, encouraging the notion that the poet's speaking ``from the heart''. Of course, if a poet has more than one voice, authenticity is compromised.
\item \textit{Distinctiveness} - It helps with marketing for the resulting voice to be easily identifiable, but that doesn't ensure quality.
\end{itemize}


Somewhere between the uniqueness of the person and the common currency of language, there's a negotiated voice. Some poets are like character actors, happy to explore different styles. They might do this via biographical poems, adopting historical personae. Other poets stick with what gave them their breakthrough role, in which case the person and the role are more easily confused with each other; as the person changes, the poetry-voice must change too, because a new voice can't be created. 

How should that voice sound? I.e. how should a poet sound? The preferred type of persona is partly a matter of fashion. Over the years the prevailing voice has changed, some examples being - 
\begin{itemize}
\item Wise and all-knowing, controlled and thoughtful - Keats.
\item Colloquial - ``Language and Creativity: the art of common talk'' by Ronald Carter (Routledge, 2004) looks at the creativity of everyday speech.  Examples include early Armitage.
\item Psychotic - one needn't be rational or stick to a single voice in a poem

\end{itemize}


I don't think that these types of characters were suddenly thought to be poetic in themselves, but they provided a platform for what was currently considered poetic - a voice was found that's ``in character'' for the desired style; a mouth to put the words in. The idea's not new - 
\begin{itemize} 
\item ''\textit{in order to write poetry, you must first invent a poet who will write it}'', Machado
\item ''\textit{En somme, le Langage issu de la Voix, plutot que la Voix du Langage}'', Valery
\end{itemize}


The narrator's stance relative to the story is important too. The narrator can be -
\begin{itemize}
\item a participant in the story
\item present, but only as an observer
\item completely outside the story
\end{itemize}

All of these can be used with a range of voices, though the first option is most common. Omniscience is an option too, though limited 1st-person PoV is most common when voice is an important feature.

\subsubsection*{The psychotic voice}


When tangental, disjoint progress doesn't suit the aloof monologue of a voice like Wilbur's, another voice needs to be sought. Currently, with closure having low priority, surrealism always an option,  and juxtaposition dominating over narrative, the psychotic character's useful.

I've had friends with severe mental problems. Their monologues had the strange connections, discontinuity and novelty that poetry sometimes has. I was impressed. Some of them wrote poetry. Lots of it. I sometimes helped to make it (in my opinion) publishable, trying not to edit out too much of the bi-polar quirks and affectations. In some ways their condition is only a more extreme version of the moods and bursts that many poets have. It's a matter of  trying to balance surprise with control, individuality with communication.


Readers attracted to confessional poetry, unconventionality, or to reading about lifestyles they're unfamiliar with, can also be drawn to such texts, especially if they don't have schitzophrenic friends (or, as common nowadays, these friends are drugged). The novelty can soon wear off. One person's honesty is another's melodrama. Repetition can become unrestrained, and often there are simply too many words, too little control. There are complications regarding reviewing too - if one knows that the poet is (or even was) in therapy, it's difficult not to use kid gloves when commenting on their work. Also content can become too dominant - it's tempting to analyse the illness rather than the poem.


In 1911, Bleuler (who coined the term schitzophrenia) quoted this passage from a medical report -
\textit{I always liked geography. My last teacher in that subject was Professor August A. He was a man with black eyes. I also like black eyes. There are also blue eyes and grey eyes and other sorts, too. I have heard it said that snakes have green eyes. All people have eyes}

Compare that with some extracts from Emily Berry's ``Picnic'' that involve eyes, rain and the sea

\begin{verbatim}
If you are not happy, the sea is not happy 
...
Watching the sea is like watching something in pieces continually striving to be whole
...
The mood of the sea is catching
...
Its colour became the colour of my eyes and the salt made me cry oceans
...

I started to be able to see in the dark
It hurt my eyes
                   My, yes, salty, wet, ocean-coloured eyes
...
When the rain came after the drought they said it was not good enough
It would not change things
It was the wrong rain
The rain came out of my eyes
\end{verbatim}

The first line in the ``Picnic'' extract associates ``sea'' and the self, the self affecting the sea, preparing for the 2nd line. The 3rd and 4th lines suggest that the sea affects the self. Rain and tears are conflated. Towards the end there's a suggestion that some cathartic release was merely physical - ``the wrong rain'' - but who are ``They''?

In the extract below, towards the end of the poem, self and sea,  tears, rain, language and other people come together. Language is a mirror aiding self-reflection, but can a moving self ever be captured in words? 
\begin{verbatim}
Who are you. Who are you. Who are you

Stop, language is crawling all over me
Sometimes if you stay still long enough you can make it go.
...
If a person standing still watched another person minutely moving
                              ;would it seem after a while as if they were watching the sea?
I remember just one thing my mother said to me:
Never look at yourself in the mirror when you're crying
\end{verbatim}


By embedding the language in the voice of a slightly confused individual, the poet has managed to use many fancy/cliched similes without coming over as contrived.

\subsubsection*{Voice-centred poetry}


Here is another passage from Emily Berry's ``Picnic'', where switches come thick and fast.
\begin{verbatim}
I like curved things
     Apples, peaches, the crest of a wave
We once agreed the apple was the only iconic fruit

I like it when I am writing a poem and I know that I am feeling something
To be poised and to invite contact
Or to appear to invite contact
\end{verbatim}


Once the ``voice'' is presumed to require a persona to produce it, the reader might go a step further, reacting as if in the presence of the person in a social situation (on a bus maybe)

\begin{itemize}
\item Line 1: The speaker is telling us about their likes, communicating well, though it's a rather odd predilection
\item Line 2: Perhaps realising that the first line might not be helpful, more details are provided; again, a good sign. However, the list of 2 similar objects then a very different one is rather odd
\item Line 3: Using the apple as a link, another person is introduced. After having previously drawn us in, an intellectual albeit interesting conclusion is reported. The speaker's straying off the topic
\item Line 4: The speaker's telling us about another of their likes, in another line that ends in ``thing[s]'', (as if a self-revelation needs to be balanced by an abstract concept). How does this interest relate to the previous one, which it's connected to by anaphora? Should we take the second phrase of this line to mean that the persona needs to write poetry so that they know that they're feeling something?
\item Line 5: ``Poised'' = balanced. ``invite contact'' = ready to engage with others. These are socially desirable goals.
\item Line 6: The difference between appearance and reality is again emphasized - others don't know the real feelings of the persona, who may only be pretending to be sociable. Again, having approached the reader, the persona withdraws, without asking for comments.
\end{itemize}

Poems like these exploit readers' conversational skills, using their reactions to the persona as the pivots that articulate the movement within a poem in preference to using their ear for music. Because discourse-based poems emulate speech, they tend not to use sound effects (regular ones, at least), using register changes instead. In ``Poetry, voice, and discourse analysis'' I look in more detail at how changes of intimacy, intensity and evaluative approach can be used to add dynamism to a poem.



\newpage
\section{Integration}
Features are often designed to work together - rhyming words are often followed by a line-break, for example. Identifying and prioritizing features is a complex task. Familiarity with the poet's other work may help.

Sometimes the poet deliberately withholds information so that integration can only happen at the end. Prior to this the reader
may be led to draw another conclusion. ``Attention, Agility and Poetic Effects'' considers how the reader's focus of attention might flit forwards and backwards though the text (each end-rhyme leading attention back briefly to the matching rhyming word), and up and down layers (the reader's following of the poem's plot might be interrupted by checking the acrostic).  

\subsection{Literary Unity}
In ``Problems and poetics of the nonaristotelian novel'', Leonard Orr writes "\textit{unity in literary works is an essential part of the concept of form, and it must be achieved both through the author's imagination and through the audience, assimilating all of the information, seeing the structure (the beginning, middle, and end), and in seeing that all of the parts serve to inform the whole}".
Aristotle is behind many of these ideas, and also perhaps behind the idea that the quality of a text can be judged by assessing its unity - the notion that in perfect texts everything is essential. But there have always been "baggy" texts (even Shakespeare's plays are often abridged). Novels in particular have asides, detours, and peripheral atmosphere-building material. More recently, methods have appeared that challenge more directly the classical notions of unity, (organic or otherwise) - "\textit{Collage, the art of reassembling fragments of pre-existing images in such a way as to form a new image, was the most important innovation in the art of the twentieth century}" (Charles Simic).

So how accommodating should readers be nowadays about apparently superfluous material? If a stanza of a poem doesn't do anything, do you just skip over it or do you penalize it (or, analogously, when you mark a multiple choice quiz, should you penalise wrong answers)? Perhaps the material's there for other readers, not you. But suppose it affects your enjoyment of the rest of the poem? Maybe it's crass, sexist, derivative, etc. Would you ignore it then? Suppose instead of a bad or pointless stanza it's a questionable line, or word, or line-break?

Obviously, it depends. But on what? The proportion of the whole that's affected presumably, and the nature of the flaw. But also it depends on the type of work being read, the reader, and why they're reading. Some poets (Selima Hill?) tend to write uneven pieces. Others (Heaney?) don't. The perceived unevenness may be because the work is multi-style or polyphonic, the reader not equally at home with the various styles/voices. Some people (especially if they're judging) will judge a poem by its worst line. Others won't mind panning for gold, seeking rare and beautiful wonders. My impression is that
\begin{itemize}

\item Non-narrative, discontinuous poems are more likely to be read with a pick'n'mix approach
\item If one line of a poem is much better than the rest, it's sometimes said to "be worth the admission fee alone", the rest of the poem excused
\end{itemize}

I don't see why readers should be more lenient with non-narrative pieces, and though I can understand why a good goal in a soccer match might justify going to an otherwise ordinary game, a poem's not a live event - it can be edited. Of course, you lose that process-over-product feel, but process, like product,  can be synthesised.

Some discontinuous poems contain a 2-D constellation of quotable fragments. What should fill the gaps between them? Dead wood or padding? Or could some continuity been provided? Options include
\begin{itemize}
\item Nothing - compacted fragments and nothing else
\item White space (to "let the images breath"; to provide space for fields to be generated between the poles)
\item Absorbent text that attempts to magnify/refract the effect of the powerful fragments
\item Text that attempts to provide continuity between the fragments (which may either amplify the fragments or mask their effect). For example, the fragments could be put into the mouth of a mad person, or be notebook extracts.
\item Text that's part of a different, continuous thread.
\end{itemize}
The reader's strategy can in time affect the poet's writing. If readers are going to ignore (rather than penalize) the bits they don't get, the poet's more likely to add more stanzas or more obscurity - after all, there's nothing to lose. I think the current fashion amongst frequent poetry readers is more indulgent than it was a decade or so ago. Perhaps the increasingly competent and voluminous output of creative writing students makes readers crave for something "a bit different".

Sometimes when another reader and I disagree over a poem, I've asked them about certain phrases only to find that the reader's ignored them rather than try (and fail) to incorporate them into their interpretation. I don't see any problem with a poem containing self-contradictions, but if the reader edits those contradictions away, something's surely wrong somewhere.

Sometimes a reader (for personal/professional reasons) \textit{wants} to like a particular text. Fragmentary poetry gives such readers more scope to do this - nothing needs to be explained and anything in it might be vital to the poem as a whole, so nothing can safely be deleted. Parts might be there to provide a change of texture, or to make other parts seem, in comparison, lyrical. 

Readers may search for a unity (a gestalt, a resolving interpretation) that isn't in the text. They may give up, using variations of the "there are only so many hours in the day" argument

\begin{itemize}
\item "If the poet's not going to bother editing their work, I don't have time to do the editing for them"
\item "I don't trust the poet. If there are so many phrases that make little sense to me, maybe the effect of the other words is delusionary, luck, a mirage. Maybe the poet's trying to bluff me"
\end{itemize}

I don't mind unresolvable paradoxes. Nor am I against "hopeful monsters" as such though I sometime wonder whether they need to be published - why not learn from them and re-write? I'm less tolerant of padding (in the form of words or space) than many other readers are. Surplus words and line-breaks can make me wary of the other words and line-breaks that at first sight seemed effective. I begin to lose trust in the poet. Before long, the implicit reader-poet contract isn't worth the paper it's written on, so I start reading another book instead.

\newpage\subsection{Attention, Agility and Poetic Effects}


The idea of cognition being a process where an "executive function" determines how
attention will be deployed isn't new. Here are two descriptions
\begin{itemize}
\item 
"Bialystok (1990, 1994), in her classic discussion of
cognitive development describes learning in terms of two cognitive
processing components - analysis and control. Analysis changes the way knowledge
is represented in the mind of the learner. Through the process of analysis, language knowledge changes from implicit
knowledge organised at the level of meanings, to explicit knowledge
organised at the level of formal or symbolic knowledge. 
Control involves a development in the learner's ability to selectively
focus on relevant and appropriate information. Control, in this sense,
means the process of allocating attention to specific representations of
knowledge and the ability to move between representations (or particular
aspects of these representations) in a manner which allows the fluent
completion of the task" (condensed from Hanauer). 


\item 
"Miller and Cohen draw explicitly upon an earlier theory of visual attention
which conceptualises perception of visual scenes in terms of competition among
multiple representations - such as colors, individuals, or objects.
Selective visual attention acts to 'bias' this competition in favour of certain
selected features or representations. ... According to
Miller and Cohen, this selective attention mechanism is in fact just a special
case of cognitive control - one in which the biasing occurs in the sensory
domain. ... Within their approach, thus, the term 'cognitive control' is applied to
any situation where a biasing signal is used to promote task-appropriate
responding, and control thus becomes a crucial component of a wide range of
psychological constructs such as selective attention, error monitoring,
decision-making, memory inhibition and response inhibition" (from Wikipedia)

\end{itemize}

Hanauer has applied this theory to literature.  As proposed by Graves (1996), the study of expert
and novice readers of literature is a useful methodology for investigating
the workings of the literary system. Hanauer summarise the empirical
studies as follows
\begin{itemize}
\item 
Experts analyse the literary text on multiple levels and integrate this
information into their interpretations; novices relate to the local level
of the text.
\item  Experts analyse the communicative context of the literary
text and the function of various literary patterns within this context;
novices follow the narrative and dialogue structure of the literary text.

\item Experts manipulate and focus on specific information in the text in
order to produce literary interpretations; novices were very influenced by
the local level of the text.

\item Experts can explicitly discuss the role offormal schematic and textual features in the construction of an
interpretation; novices paraphrase the meaning of the text.

\end{itemize}


 I'm going to look more
closely at the "levels" aspect of these conclusions, incorporating  the factor of speed.


\subsubsection*{Stratified Literary Features}

 A literary work has many features, some of which might be
considered as "layers". For example, Roman Ingarden developed
Aristotle's concept that a literary work of art has at least 4 layers,
starting with sound, then sense. Perceived properties (like beauty,
difficulty, etc) can slide from one level to another. In an experiment by Song
  and Schwarz where people were shown
recipes in different fonts, a recipe in a font that's hard to read was
thought to be harder to execute than the same recipe in an easy font.
Similarly, speed of reading (controlled by layout) can affect the perceived 
speed of the narrated events, and a surprising layout can be synchronised 
with a narrative surprise. I suspect that experts are more aware of the transference of
such  characteristics. In the "recipe" experiment  the effect 
disappeared if the experimenter apologised to the subject for the readability
  of the font. An expert reader might not have needed such an apology in order
  to compensate.


Upper layers can have emergent features absent from lower ones (emotions
  only exist at higher levels) or can have effects that contradict those of
  lower layers (a story full of jokes can be sad).

Interpretation is not a straightforward progression from lowest to highest level. A higher level interpretation can provoke a re-interpretation
of a lower level. Years ago there was a "Truth to Materials" credo in
sculpture - a belief that wooden sculpture should exploit knots and the
grain rather than try to gloss them over. Many poems seem not to
acknowledge the "graininess" of language - the lower levels (the choice of font, color, etc) are soon ignored. Low-level features are more likely to "show through"
palimpsestically in poetry than in prose, though many prose examples exists. Some examples:

\begin{enumerate}
\item If in a hand-written letter one reads
"this is written with my blood", the physical medium becomes significant again.

\item When reading handwriting one might reinterpret letters after having failed
to deduce a satisfactory meaning.
\item 
  "Janet likes John" is a simple enough sentence. The interpretation of  "Janet likes \textit{John}" is slightly different.
\item "love's sore return" - The letters painlessly form into words that in turn combine to form a
sentence. When appended by "(4)" and seen as a cryptic crossword clue
however, words are broken down into letters, and the meaning of the apostrophe changes.
\item 
  "Flower: exploding star in retreat" - An imagist poem? No, another clue.
An exploding star is a
nova, which reversed spells the solution, which is "Avon" - something that flows and 
hence is a 'flower'. In this case we need to regress back to the 
"letters" level and re-create an alternative sound and meaning for 'flower'
\end{enumerate}

 In the table
below I attempt to describe the effects of some layered poetic features on
attention.
The up/down directions mentioned below pertain to the hierarchy formed by
letters, words/sound, localized meaning, and general meaning. The in/out
direction is relative to the text. I also consider the narrowness of the attention.

\begin{table}[htbp]
\centering
\begin{tabular}{|l|l|l|l|}\hline
\textbf{Effect} & \textbf{In/Out} & \textbf{Up/Down} & \textbf{Intensity/Focus}\\\hline
Meter & In & Down to sound & Wide\\
Rhyme & In & Down to sound & Medium\\
Acrostic & In & Down to letters & Medium\\
Internal reference & In & Same level & Narrow\\
Intertextual reference & Out (text) & Same level & Narrow\\
Nouns & Out (World) & Up out of language & Medium\\
Proper nouns & Out (World) & Up out of language & Narrow\\\hline
\end{tabular}
\end{table}

I hope  these entries are not far from
your subjective impressions. Meter, for example, is a field effect, a 
wide dissipated awareness of sound emerging from below. End-rhyme has a more
  local affect. On the other hand if someone
called James Lawson was reading a poem that said "James Lawson is a
prat", attention would be narrowly focussed away from the poem and away from
  language. Wherever the attention is dragged, sensations like "difficulty"
might be carried along too.
Some points to note


\begin{itemize}
\item The fewer "in" effects, the more transparent the text appears to be (the
  fewer layers it seems to have)
\item Something like a proper noun that can distract attention from the text
can be anchored to the text by making it rhyme, etc.
\end{itemize} 


Rather than use the analogy of layers, this situation might be described in terms of processes with feedback. 




In this document I'll stick with the "layer" metaphor for the sake of argument
(and because it helps with the "palimpsest" metaphor).


\subsubsection*{Layer confusion and compression}

The layers are not always clearly distinguished. For example,
\begin{itemize}
\item Cryptic crossword clues conflate the layers
\item Phrases like "love is a four letter world" make readers change depth
\item Finnegan's Wake exhibits layer instability, as do poems like 'What's in a Homophone' (Josephine Abbott, Staple 25) which begins
\begin{verbatim}
How can I bare it?
My idle,
My bridle partner
Left me last weak
For a made.

What a waist
Of ours
And ours.
\end{verbatim}

\end{itemize}


Gerard Genette used the term  'metalepsis' for when boundaries between layers are crossed by characters or other textual
elements. For example, in Coleman Dowell's novel "Island People" a low level framed story becomes the top level, taking over the narrative, creating a kind of Mobius band. 




Sometimes, especially during reading, the concept of layers is ignored; 
layers are merged together. This is also done in Cartography, in painting 
(perspective), and in Photoshop (to save storage space). It's often done to
"fix" the interpretation from a particular viewpoint. The disadvantage
is that the process is irreversable and makes alternative interpretations
difficult. Making the reader do this might be the intent of the author, lulling the reader into a false
  sense of security.


\subsubsection*{Time}

Within a layer each new element
can suggest meanings or eliminate possibilities (for example, consider the
effect of each new fact in a whodunit). There can be an association with other 
phrases in the same layer or with elements in other layers. E.g.

\begin{itemize}
\item 2 ambiguous phrases taken together can produces an unambiguous result. 
\item A line-break (a lower layer feature) can change the interpretation of a phrase. 
\end{itemize}


Reading often doesn't proceed linearly through the text.  Even in prose there's
typically 10\% of saccades (eye-movements) backwards. In poetry this percentage
is likely to be higher. In particular ambiguity and confusion will
provoke backtracking or regression to lower layers. The amount of backtracking needed will influence the
reading strategy. Options include

\begin{itemize}
\item \textit{serial processing} (if readers pick the wrong alternative they
backtrack to the last fork - usually on the same layer)
\item \textit{parallel processing} (keeping all options open - multi-tasking) 
\item \textit{minimal commitment} (choosing an option but accepting that the chosen
  option is provisional - using peripheral vision) 
\end{itemize}

Chapter-ends and poetic line-breaks tend to force a decision/resolution. The
sentence above in the 'layer confusion' section works better with line-breaks

\begin{verbatim}
   love is
   a four letter
   world
\end{verbatim}

Rather than just using the emergent meaning one can have a richer reading 
experience by retaining the lower-level meaning too, leaving that up/down dimension open for traversal.
End-rhymes force an awareness of the lower level of sound, though it's only
one of many poetic effects that do this.
In general, the path through the text is less linear for the poetry reader
than for the prose reader. Poetry readers are more likely to bob between layers
and move backwards and forwards in the text - they're more agile, and they
might read prose in a similarly agile way.


\subsubsection*{Mental Agility}


When it's said that someone lacks mental agility, what is meant? Clearly
it's something to do with speed (which hasn't yet been mentioned) as well as
movement. I think the movements alluded to fall into 3 main types

\begin{itemize}
\item Moving forwards and backwards in the text
\item Changing between 2 modes of attention (for example, from visual effects
to
plot-following, or seeing an issue from others'
viewpoints)
\item Zooming in/out while staying in the same mode.

\end{itemize}


Control of this movement is the role of what's often called the "central
executive function". A related notion to this movement is multi-tasking (or dual tasking). There's quite a lot of
research about this so although it's not strictly relevant, it's worth
considering the material because it too involves task-switching. There are 
situations where we perform multi-tasking - when a task is automated
(e.g. driving), or when the tasks
can be clumped into a fewer
activities (a pianist doesn't have to worry about each finger or hand in
isolation) but as Earl Miller has shown,
we really only focus on one or two items at a time. Multitasking is essentially
delegation or  fast task-switching. Autistic people tend to lack this ability. 
It's harder when the tasks are similar, or use the same part of the brain;
 easier when the tasks are easily interruptable.



Expert readers have fewer difficulties
\begin{itemize}
\item Expert readers are likely to use more parts of their brain because
they look for more features
\item Expert readers have more opportunies to clump, putting less load on
  working memory.
\end{itemize}

\subsubsection*{Plus points}


This mixture of tightening
and loosening the reader's attention, of turning their attention inward, upward,
then outward, of rocking them backwards and forwards through rhyme, can be
choreographed by the poet, and presents a challenge to readers with inflexible
attention strategies, resulting in simplifications and tunnel vision.
In some poems inflexibility isn't important because there's a uniformity of effect. For example
\begin{itemize}
\item With Dada "sound poetry" attention is always "down to sound"
\item A list poem (of regions used in BBC weather forecasts, for example) predominantly
uses one effect
\item Confessional poetry has a narrow focus
\end{itemize}

During reading, some people resist the movement between layers. 
They become stuck in a layer, trying to read Finnegan's Wake as if it were
a poorly spelt straight story. They lose the possibility of seeing how the 2 movements 
(forwards and backwards; up and down) interact. It's as if an archaeologist 
ignored the depth at which artifacts were found. For this reason (and because they
  notice peripheral clues) agile readers might be less prone to whodunnit
  punchlines.

At other times (e.g when reading an ironic text),
  not only is flexibility required 
but speed is too. Using a cinema analogy,
what's required is not only a good editor to combine the viewpoints afterwards, but
a good director too, otherwise important data may not be
collected in the first place. An effective director might perform an
initial scan through the modes to see if anything stands out (much as one
might scan the horizon or flick through TV channels, or size up a poem) - a
  shallow, wide search. In a live TV show a director needs to be an editor
  too. They may have a bank of
  screens showing the output from various cameras, and like a reader will
choose which to prioritise whilst retaining a periperal awareness of
happenings on other screens. This is analogous to the situation a poetry
listener is in - speed of control and task-switching become important. 
Once interesting
features are identified, attention might be centred on those modes that
recognise the features. To be effective, people require
\begin{itemize}
\item quick processing (so
the director can decide quickly where to focus attention next)
\item sufficient working memory storage so that a task can be continued from where it
was left off when task-switching.
\end{itemize}


With a good director, the person can still emulate someone without agility
(they can narrowly focus on a single task), but they can also rapidly
juxtapose 2 viewpoints (hence produce comedy, insight, dramatic irony, etc)
producing responses that 
are hard for non-agile readers to appreciate (like not getting a joke)


\subsubsection*{Minus Points}


Uncontrolled flitting can be a problem. It can
look like
\begin{itemize}
\item  inappropriate detachment - withdrawal; lack of response
\item  multiple personality
\item  confusion
\item  attention deficiency
\item  shallowness
\item  dilution of the poem's impact by considering too many types
of features.
\end{itemize}

 By breaking "agility" down, problems and diagnoses can be more
  accurately diagnosed. A fast director (an imprecise one especially) can increase the workload for
  the "editor" stage. Continuing the film analogy
\begin{itemize}
\item If a good director has few cameras they'll be good at a few tasks,
but they'll have blindspots
\item If a good director has bad cameras they'll not miss the obvious (common sense
  rather than insight)
\item If a bad
director has good cameras someone else can use their work. it might be saved
by a good editor
\end{itemize}

\subsubsection*{What can be taught?}

Both speed of task-switching and range of tasks can be increased by
  training.  Readers can practise switching, superimposition, etc against the
  clock, and they can become more sensitive to issues like Form, Implied
  Addressee, etc.


According to Hanauer both implicit (theory) and explicit (examples) teaching methods
should be used. The former allows the development of individual literary
patterns and the latter widens the options of types of literary pattern
that can be considered.

\subsubsection*{An Example}
\begin{verbatim}
               Going Down
   Cycles cluck past, two boys walk with each girl,
   And upstairs James, not Jim, strums his sitar,
   Making words tadpole in your desklamp's pool,
   Breaking the concentration of the hour.
   ...
\end{verbatim}

Rather than analysing a single factor in any depth, an expert reader will
  briefly
consider layout, sound, the title and the first few lines of a poem to see
  which
tools might be most appropriate for the task and genre. This pieces look like a sonnet,
  so the reader's attention might be attracted towards sound - particularly  line-endings.
Later there
  are enough references for the reader to realise the poems about Oxbridge
  life. Once they've finished a first reading, the title might be revisited ("Going
Down" means leaving University). But the title's also a clue that the
poem's an acrostic, a realisation that leads the reader to re-visit the lower
  levels. 
Readers need to go backwards and forwards as well as up and down. They can't
afford to forget about a layer once it's been interpreted.

\begin{figure}[htbp]
\centering
\includegraphics[width=10cm]{depth.png}
\end{figure}
 Schematically (and somewhat
  fancifully) here's a trajectory through the poem. The reader begins at
the title (1) and perhaps guesses a theme, then does a low-level scan of 
the whole piece (2 to 3) then returns to a fairly high level (4) to
read the piece, their attention draw back and down to the sounds of 
previous end-rhymes. At the end (5) there's another dip into lower levels (6)
to read the initial letters of the lines, then finally (7) a high-level
conclusion. A similar graph of a prose reader's journey would be a
  near-horizontal line.
\subsubsection*{References}
\begin{itemize}
\item Bialystok, E. (1990), "Communication Strategies: A Psychological Analysis of
  Second-Language Use". Oxford: Basil Blackwell Inc.
\item Bialystok, E. (1994), "Analysis and control in the development of second language
proficiency". \textit{Studies in Second Language Acquisition} 23, 157-68.
\item Graves, B. (1996). "The study of literary expertise as a research strategy". \textit{Poetics} 23, 385-403.
\item  Hanauer, D. (1999), "Attention and Literary Education: A Model
of Literary Knowledge Development". \textit{Language Awareness}, 8 (1), 15-29.

\item Miller, E.K. and Cohen, J.D. (2001), "An
  integrative theory of prefrontal cortex function". \textit{Annual Review of Neuroscience} 24,
  167-202.
\item Song, H. and Schwarz, N. (2010), "If it's easy to read, it's easy to
  do, pretty, good, and true", \textit{The Psychologist}, V23, February 2010
\end{itemize}

\newpage\subsection{Reading strategies: when top-down meets bottom-up}

Reading took evolution by surprize. Various regions of the brain had to be co-opted to provide the necessary skills. The way we use and coordinate these functions varies according to the language and circumstances. Experienced readers of novels will read a chunk of words at a time, recognising words by their shape, but infants learning English build words up letter by letter, sound by sound. That childhood strategy isn't lost as we develop, it's brought into play when there are new words, misspellings, etc. Sometimes sounds matter too, activating other brain regions.


With some texts, eye-movement won't be regular - they'll be some forward and backward jumping. Physically it's not just our eyes that are involved in reading. People who sub-vocalise will struggle with tongue-twisters, and the body sometimes mirrors activities that are read about - if a character wriggles their toes, readers are likely to.


Poetry can exploit these low-level, often dormant possibilities. It can also stretch experienced readers in the top-down direction too. Seeing a text for the first time, they might assess quickly whether it's a love letter, a maths proof or a phone directory, and begin reading accordingly. Within broad genres there are sub-genres - knowing that a novel is literary SF might lead them to read in a different way to when reading a ``Mills and Boon'' or Harlequin novel. That initial assumption may prove misguided or unhelpful (indeed, the author may deliberately subvert the genre) but readers have to start somewhere. The assumption helps readers decide how fast to read, whether to read linearly, whether to look for irony, and whether to laugh or cry. As well as being aware of genres and subgenres, experienced poetry readers are likely to have a collection of templates in mind - ``the list poem'', ``the Naming of Parts poem'', etc.


Poetry and prose are often thought to encourage different reading modes. Reading ``poetry'', people tend not to expect plot, and the persona's more likely to be conflated with the author. Bottom-up processing is likely to matter more - low-level features like sound may convey meaning. But the prose and poetry genres, like many others, overlap. Readers needs to remain flexible. 


There can be clashes. For example
\begin{itemize}
\item Sometimes the assumed genre (or template) has such a hold on the reader that subsequent counter-indications have no effect. The shopping list or note on the fridge never becomes a love letter, or a  text read in a poetry magazine may continue be read as a poem despite its content. 


\item Sometimes people take in the music of a poem without reading the words one at a time. Bottom-up interpretation clashes with the top-down, impressionistic feel. For example,  reading a poem called ``Mirror'' the close-reader may see that the poet is contrasting themselves with a passive mirror, whereas the top-downer might assume that the poet's identifying with the mirror (perhaps they misread ``I am looking at a mirror'' as ``I am a mirror'' to make it fit their assumptions, or they missed a significant ``not''). Is the top-downer ``wrong''? Maybe not; despite the words, the top-downer's interpretation may be the more valid one, the persona in denial perhaps. 
\end{itemize}


I think an experienced reader is likely to negotiate between top-down and bottom-up strategies, especially when reading poetry. The experience of reading one way informs the other. Poets subconsciously or otherwise can exploit this. In ``Tears in the fence'' No.59 Spring 2014, Mark Goodwin has some poems. Here's the start of ``Mind Will''
\begin{verbatim}
wind th   rives in sky's grasp the   wind
ing of cloth pulls   the sky's hear   t open

and takes the p   ush of clouds &amp; distant
land into the text   ure of corn's matt talk
\end{verbatim}

The gaps allow a little Joycean wordplay, bringing out new meanings, though the effect's rather muted. It's more like disruption, stopping the reader using a standard novel-reading method of processing - letters rather than words need to be processed, and the 2nd line's ``ing'' will cause most readers to backtrack. In the 4th line, readers are likely to sense ``\textit{text of [the] talk}'' and ``\textit{corn stalk}''. The next poems in the magazine are by Chris Hall. Here's the end of ``Five Surrealist Paintings''
\begin{verbatim}
th rose in th orangery
purpl turtl
th writing on the carapace
th blood on the flagstone

no fish
\end{verbatim}

Dialect? As with Hall's piece, prose processing is impeded, but this time some lower level aural as well as visual processing needs to be adopted. If nothing else, reading will be slower. Hall ends his selection of poems with an author's note
\textit{My poetry is expressed on the page in an unusual verbal form. This is note because of any particular lexical experiment or linguistic virtuosity on my part. Rather, it is an attempt to force readers into 'voicing' the content in their own head, as the poems are intended to be experienced aurally as much as visually, and usually emerge for the first time in public readings rather than in print.}

Poets don't often leave such notes for the reader. The intent's laudable, though maybe phonetic spelling could have been used throughout?


For some types of texts (maths, but also some poetry) each symbol matters, and readers may benefit from being made to read in a non-prose mode. In \url{http://web.princeton.edu/sites/opplab/papers/TICS686.pdf} ``The secret life of fluency" Daniel Oppenheimer wrote that for some exercises, ``\textit{participants were significantly more likely to detect the error when the question was written in a difficult-to-read font. This suggests that they were adopting a more systematic processing method and attending more carefully to the details of the question}''. It's possible that the painstaking reading strategy that dyslexics are forced into may even be of benefit to them in some subjects. Perhaps poetry is one of them.



\newpage\subsection{Strange Forms}
Ancient Hebrew cultures valued poetic word-play, as did Greek, Roman and early Arab writers. In medieval China (400–600 AD) the “New Songs from a Jade Terrace" collection was playful, and the 7th century poet Magha wrote elaborate Sanskrit. Historians (Barbara Tuchman, etc) have suggested that in medieval European households word-games provided an important source of entertainment somewhere between the poems or stories of troubadours and games such as cards or chess. Dante, Petrarch, and Boccaccio used numerology as well as various word-games. Late 15th century France had its “Grands Rhétoriqueurs". Such writing was still quite common in the 16th and early 17th centuries even amongst the greatest writers. “Love's Labor's Lost” pushed word-play so far that G.B. Harrison (editor of “Shakespeare: The Complete Works") thinks that critics have tended to “leave the play to those who are more interested in literary puzzles than in poetry". Ben Jonson's “The Alchemist” (1610) begins with a 12-line “Argument” whose initial letters spell out the title, though elsewhere he also made fun of verbal tricksters – a portent, because by the 18th century the reaction against “false wit” was well established. Pope's “Essay on Criticism" and Joseph Addison (especially in the Spectator Nos. 58–61 of 1711) were influential voices condemning the prevalent “trick writing” which included poems that intentionally banished a given letter, rebuses, echo poems, limited-word exercises, acrostics, anagrams, and chronograms. Addison blamed English monks with too little talent and too much spare time for reviving these Latin and Greek tricks. Further criticism of “false wit” came from Dr. Samuel Johnson. In his wake, the steamroller of 18th century neoclassicism and rationalism followed by the Romantic revolution broke a tradition which has never fully recovered.

Word-play survived in the UK amongst the masses as parlour games, in advertising, and in the popular “Miscellania” publications of Victorian times. Freed of the lyrical imperative, post-modernism revived ludic interest, with poets like Heather McHugh and Paul Muldoon exploiting various devices. Word-play and constraints suited the ego-suppressing aims of some language poets too – Jackson Mac Low used procedural constraints – but such work remains steadfastly marginal. The most sustained attempt at developing forms rather than dabbling in sporadic word-play has come from the Oulipo (Ourvoir de Littérature Potentialle) movement. Whereas readers seem prepared to tolerate a little word-play (finding it acceptably liberating, fissuring down into the core of language) rigorous Oulipian forms are all too often granted no value in themselves, being considered self-imposed handicaps, impediments to truth. Indeed, they often have a negative value that content can at best only compensate for. Oulipo writers claim that free verse is never free – if authors don't define a constraint, constraints will in turn define their work for them. Even sonnets are too restrictive for many poets today, but there are many more challenging forms. Poetic forms commonly use patterns based on sound (metre, rhyme) or number (syllabics) or both (iambic pentameter). Graphological forms (based on spelling) are less popular. Perhaps this is because
\begin{itemize}  
\item      like syllabics, they work on the page rather than orally, and the oral tradition still dominates.
 \item     being uncommon they're too easily misunderstood or dismissed as merely playing with words, an opinion strengthened by their overuse as lesson exercises or as cures for writers' block.
 \item     forms often work by using repetition to establish an expectation which is then only partially satisfied – e.g. variation on the meter; strong and weak rhyme, etc. Forms based on spelling are less able to exploit such effects – if a word’s misspelt, it’s “wrong", not “interesting".
  \item    suspicions remain that although the challenges of strict forms may give rise to fracture and compression, the results appear in poetry magazines under false pretences. The reason that (e.g.) Abecederians are in poetry magazines may simply be that venues for short prose have disappeared.
 \item     the technical difficulty of the forms render practitioners vulnerable to accusations of intellectual or social elitism despite linguists' claims that word-play is natural and universal, as common in the Australian outback and central Africa as in Sorbonne common rooms.
\item      historically, such forms have encouraged a currently unpopular competitive element, the “degree of difficulty” of the form mattering as much as the execution.
\end{itemize}
As exhibit 1, let's take this extract from Two ways to make it
\begin{verbatim}
         Oh Eros, the hot heroes
    like you once
          rose sore
    from bed,
          each ache
    a proof of love. Now
          actors co-star
    in divorce
          suits – it's us
    they envy.
\end{verbatim}
Anagrams have a long history of use in poetry, and can augment meaning just as other low-level effects (rhyme, alliteration, etc) can. As a carver might relish the feel and grain of wood, so poets can exploit (rather than gloss over) the raw material of their craft. In a poem like U.A. Fanthorpe's Word Games crossword clues are used haphazardly. Here the form is regular – alternate lines contain anagrams. The form relates somewhat to the theme, and (except for the line-breaks) doesn't disrupt the poem. Indeed there's a case for saying that the form doesn't disrupt the poem enough – readers might not notice it. An ostentatious form needn't be advertised, but readers nowadays might need a footnote if the pattern isn't obvious. More rigorous still is Bill Turner’s Anagram Homage (published in Iota), every line of which is an anagram of the same UK poet. Here’s one stanza
\begin{verbatim}
    Is a pen neutral? I
    peer (Italian sun!)
    at plain ruin, see
    in alien pasture
    a supernal tie-in.
\end{verbatim}
See also "Uncouplings" (Craig Arnold), ``On Anagrams'' (Luke Kennard) \url{http://www.pennedinthemargins.co.uk/index.php/2016/05/on-anagrams-luke-kennard-exposes-the-building-blocks-behind-31-biblical-poems/}, ``Posterity, Look out'' (an anagram poem by Kevin McFadden, published in Qualm), and \url{http://www.anagrammy.com/forum/message/146289} ``About Paris'', a shaped anagram poem.


    Another carefully titled poem is ``Lost Letters'', which begins
\begin{verbatim}
    "Too staid", critics said, “too sad. Poems shouldn't
    mean but be". So must the work of men like me who
    chose Jarrell's hose or Heaney's hoe become
    sparse, hard to parse as they disappear up our collective arse?
    Can't they swing and sing as if prose were a sin?
\end{verbatim}
People will notice the internal rhymes but there's a more regular pattern – each line has a triplet of words (“staid/said/sad”, for example) where letters are 'lost'. George Herbert's Paradise uses similar word-play
\begin{verbatim}
    What open force, or hidden        CHARM
    Can blast my fruit, or bring me  HARM
    While the enclosure is thine       ARM
\end{verbatim}
The following stanza starts a poem that uses a more radical technique known as “slenderizing”
\begin{verbatim}
    A poet's double life (draft)

    He went gray, too
    guilty to stray,
    wanting to graze
    on beauty without
    needing to pray;
\end{verbatim}
If you remove the Rs you get another poem. Here’s another 2-for-the-price-of-1 form – a multi-word pun
\begin{verbatim}
    Doubled up in pain

    He'd long desired her. Twilight restored,
    he wondered on the way, doubtful of fate,
    still only a boy, far from sure. No ring –
    it was finished, a lover gone. No mistake.

    He'd longed. He sighed, hurt, while high trees stalled,
    he wandered on the wade out, full of hate,
    still lonely, a boy far from shore, knowing
    it was finished, all over, gone. No missed ache.
\end{verbatim}
Forms can be borrowed other cultures or eras: multi-word puns are more popular (and easier) in French; Anglo-Saxons used regular alliteration; the Chinese had 'magic square' poems which could be read in various directions. An early English example of this is A square in verse of a hundred monasillbles only: Describing the sense of England's happiness, written in honor of Elizabeth I by Henry Lok. By tracing the patterns of sub-squares or crosses, several other poems appear in it. Lewis Carroll relished such challenges, writing the following
\begin{verbatim}
    Square poem

    I often wondered when I cursed,
    often feared where I would be –
    wondered where she'd yield her love,
    when I yield, so will she.
    I would her will be pitied!
    Cursed be love! She pitied me …
\end{verbatim}
which is the same whether it's read the usual way or column by column. Note that the poem rhymes. It’s not uncommon for these forms to be combined with more conventional effects. Acrostics, for example, are frequently sonnets or blank verse. I think it was only 20 years ago that a near-acrostic was discovered in Act III, Scene I of “A Midsummer-Night's Dream", spoken by Titania
\begin{verbatim}
    Thou shalt remain here, whether thou wilt or no.
    I am a spirit of no common rate,
    The summer still doth tend upon my state;
    ANd I do love thee. Therefore go with me.
    I'll give thee fairies to attend on thee;
    And they shall fetch thee jewels from the deep,
\end{verbatim}
Accidental? Maybe those million monkeys finally got lucky, but in any case acrostics have a respectable pedigree – even Dante used them. Here (and in Muldoon's “Capercaillies") the unit is the line, but word and stanza units are possible. Several other features can operate on more than one level. Carroll's square poem worked at the word level, though in modern periodicals one sometimes chances upon poems made of a grid of phrases that can be read row-wise or column-wise to make different poems. Palindromes at the letter-level struggle to be poetic, but mainstream poets have used the device at the line level – "The Back Seat of my Mother's Car" (Julia Copus) is a line-palindrome, and there are two in July 2008's issue of Weyfarers. ``The abecedary form - Carolyn Forche'' (Martyn Crucefix) \url{https://martyncrucefix.com/2015/05/28/the-abecedary-form-carolyn-forche/} covers another form in detail.

All these forms have wide application. As yet, some have only reached the stage of feasibility studies – potential literature. In particular there’s scope for exploration of bespoke, specialist forms. Here are two examples.
\begin{verbatim}
    8 by 8

    "Supercomputer Hydra slays U.K.'s top chess player” (June 2005).
    Making war, young Moguls mated,
    men stylized by courtly Moors.
    Europe's chequered board helped Bishops,
    until the noble game was hacked;
    castles fell to hypermoderns –
    Marcel Duchampion duchess
    played John Cage and mocked past masters:
    the king is dead – the Hydra heads.
\end{verbatim}
The lines are alternately iambic and trochaic – the stressed syllables represent the board’s white squares. By placing the poem's pieces (the kings, bishops, and castles) on the squares corresponding to their syllables (and guessing their colours correctly) a chess puzzle is created where you can work out what the previous move must have been, as illustrated on

Here’s a section of Harry Goode’s Against the Jostle and the Thrust
\begin{verbatim}
                        Soil makers turn and sift     as by shuffle
                  and whirr and veined wing       caught in amber bright
                           or coal dark shaft      and the settle and fold.
                                   Tegmina,      testa, bark, bristle and bone,
                                upright       against pull, support for the push.
                                Arc      guarding eye, skull, beak, talon and claw.
                  Skin soft      Africa ape, with knowing thumb.
                  Enjambment      across mountains, plains and seas,
               stride far reaching        covers dreams, covers worlds
\end{verbatim}
A white zigzag cuts through these decametre lines. On each side of the divide are either T and A (e,g. “sift     as"), or C and G (as in “wing       caught"). Those letters are used by biochemists to symbolise the 4 bases of DNA, which only combine in the pairs the poem uses. The zigzag denotes the double-helix.

Most of the forms illustrated here (especially the specialist ones) support the content and might even be described as “organic", but not in a way that will satisfy everyone, which is a shame because organic form is when the form and the words evolved together - when form's not an afterthought. The later the constraint's applied, the more likelihood there is of strain - like the budding poet who only worries about the rhyme when they're on the 2nd line of a couplet, as if form were taking belated priority over content. These forms establish an unbreakable link between poetry to words. To some readers sound has an intrinsic, even visceral, effect that letter-based patterns can never replace. But some types of Art deliberately and uncompromisingly assert form or procedure over content. It's a different game, but one with potential. Escher wouldn't compromise, nor in many cases would Cage. It's like the game mathematicians play. It's interesting to see what "beauty" survives or emerges if you keep to the rules, it's almost as if it were a discovered (rather than invented) "truth", something buried or inherent in the form/constraint that has to be "brought out". If you don't share Johan Huizinga's views in the value of play, or Margaret Boden's on the creative value of constraints, or if you consider form a lifeless container, then these poems may be arduous to read let alone write, but I think there's room in the mainstream of poetry for many more of these transcultural, hybrid forms than we're currently using. See "Adventures in Form" by Tom Chivers (ed) (Penned in the Margins, 2012) for futher examples.

\subsubsection*{References}
\begin{itemize}
\item    “Shakespeare’s Lost Sonnets: A Restoration of the Runes", Professor Roy Neil Graves (\url{http://www.utm.edu/staff/ngraves/shakespeare/} – where much of the historical information in this article came from).
\item      “Oxford Guide to Word Games", Tony Augarde, OUP, 1984.
\item      “Palindromes and Anagrams", Howard W. Bergerson, Dover Publications, 1973.
\item      “Silent Poetry: Essays in Numerological Analysis", ed. Alistair Fowler. Routledge and Kegan Paul, 1970.
\item      “Pattern Poetry", Dick Higgins, SUNY Press, 1987.
\item      “Oulipo Compendium", edited by Harry Matthews and Alastair Brotchie, Atlas Press, London, 1998.
\item      “The New Princeton Encyclopedia of Poetry and Poetics", Preminger and Brogan, Princeton University Press, 1993 (see the Anagrams entry).
\item      “Word Ways: The Journal of Recreational Linguistics” (1968- ).
\item      \url{http://www.joma.org/images/upload_library/4/vol6/Growney/MathPoetry.html#Counts} shows Henry Lok's poem.
\item      ``Without a Net: Ernest Hilbert on Optic, Graphic, Acoustic, and Other Formations in Free Verse'' \url{http://www.cprw.com/without-a-net-optic-graphic-and-acoustic-formations-in-free-verse-by-ernest-hilbert}
\item      ``Poesia per gioco'', Giovanni Pozzi, il Mulino, 1984

\end{itemize}
Unattributed poems are by the author: the first and third from Poetry Nottingham, the others unpublished.





\newpage\subsection{Truth to Materials and Heather McHugh}
Woodcarvers use a natural material that has grain and knots. They could paint over their finished work, masking the irregularities. Alternatively, they could exploit them. A knot could become an eye. More often, the irregularities are used to create an independent source of interest. Artists might even choose a piece of wood with these effects in mind. Similarly, some painters don't mind their brushstrokes being obvious. Soutine for example didn't disguise the fact that his paintings were made of paint.

In the 1900s several architects and sculptors felt that the nature of the materials they worked with shouldn't be hidden. Henry Moore and others went further, claiming that certain materials suited certain purposes, that an art work should exhibit "Truth to Materials". The sculptor, Brancusi, believed that his art might "coax an image from within the material rather than forcing an image onto the materials".

In poetry the material is words. They have visual and sonic roots - letters and phonemes - so poets have two ways to demonstrate their truth to materials. Those two ways are related but not equivalent. Words with nearly the same letters often have nearly the same phonemes - "rough" and "tough" for example - though sometimes they don't - e.g. "rough" and "bough". Unless you're a crossword addict, words comprising the same letters aren't as strongly associated with each other as words that rhyme, but the option exists. An example is Jon Stone's "Mustard" where instead of each line ending in a rhyme they end with an anagram of "mustard".

Exploitation of these effects draws attention to the media. Just as varnish can accentuate the wood grain, so line-breaks can accentuate sounds. And as with wood, the effects can synchronize with the meaning or be largely independent of it ("The remarkable result of Valéry's treatment of sound and sense as consciously separated variables is that it allows the semantic components of the poem to take on structural value and the structural values of the poem to take part in a semantic or signifying action in turn" - "Paul Valéry and the Poetry of Voice", C. Crow, CUP, 1982, p.55).

On woodworkinghistory.com it points out that "The truth-to-materials doctrine appears as a consequence of technological development", and that there are connections to the Arts and Crafts movement - a reaction against mass (non-individualized) production. Devoting attention to the material at the expense of the content would tend towards the appearance of craft rather than art, which when backed up by doctrine might lead to unsuccessful works, especially if the audience is unfamiliar with that type of art. Just as people without an ear (or the ability to integrate sound and meaning) might think sound effects obtrusive, so people who struggle with wordplay might over-emphasise its relevance, hinting at methodological similarities to Jewish mysticism (the Kaballah), or the poet's apparent psychological obsession with form over substance.

Perhaps Concrete poetry exhibits Truth to Materials. I prefer the example of Heather McHugh (Paul Muldoon uses wordplay too, but I find McHugh's work more approachable, her aims more conventional). According to her poetry foundation bio her "work is noted for its rhetorical gestures, sharp puns and interest in the materials of language itself". In her work the words often retain a trace of their origins, pun and wordplay used to advance the poem. Here's part of her "Language Lesson 1976"
\begin{verbatim}
On the courts of Philadelphia
the rich prepare

to serve, to fault. The language is a game as well,
in which love can mean nothing,

doubletalk mean lie. I’m saying
doubletalk with me.
\end{verbatim}
and here's the start of "Ghoti" (a word GB Shaw invented)
\begin{verbatim}
The gh comes from rough, the o from women's,
and the ti from unmentionables--presto:
there's the perfect English instance of
unlovablility--complete

with fish. Our wish was for a better
revelation: for a correspondence
\end{verbatim}
and yes, she's into anagrams - here's the start of her transliteration of Sonnet 23 (“As an unperfect actor on the stage”), where each line's an anagram of the original:
\begin{verbatim}
AS AUTHORS CAN’T PERFECT ONE AGENT

so e-agents can’t perfect an author.
His art (howbeit swapped shut) is his fire—
\end{verbatim}
There are risks associated with this style. Once wordplay becomes a factor, readers may well think there's too much or too little of it. They might think it displaces (rather than augments) the content. Quite possibly they'll be distracted by the wordplay even if it augments. Flippancy is a common criticism; is it right to play with words when the poem's about a parent's death? One answer to that may be that the gulf between words and the world is so wide that any attempt to capture the notion of death in words is flippant, and exploiting the instability of language is mimetic.

McHugh (who had a poem in the New Yorker while a student) is not without her critics.
\begin{itemize}
\item    Hugh Seidman thought she sometimes "manipulates language to produce resonances of meaning without necessarily creating a psychological depth that might justify her insights and conclusions.".
\item      Joshua Weiner (in The Boston Review) wrote that her "'will to be peculiar' (her own phrase for Dickinson) encourages a syntactic and semantic contraction into enigma; sometimes her jokes overkill. Such faults have developed among persistent strengths: in these formally distinctive, deeply felt, and intellectually challenging poems, McHugh has invented a style for herself that acknowledges the materials and contingencies of language without sacrificing poetry's primal resource in song."
\end{itemize}
She accepts that she is more sensitive to words than others are, possessing almost a type of synaesthesia. In an interview she said
\begin{itemize}
\item      You know, I never could tell things apart the way healthy people do. Meaning and means. Form and substance.
\item      I was never very good at settling for any one sense of sense. So semantics became largely a matter of syntactics for me. Poems don’t make sense; they make senses.
\end{itemize}
My suspicion is that hers is the kind of cleverness that's currently unfashionable in the UK, where the voice is more important than the word. The academic voice isn't considered as revealing as the slightly deranged one.

\newpage\subsection{Choosing between sound and sense}
When readers engage with a poem, they pump in effort and attention.
Where does that energy go? Usually it seeps out as paraphrasable meaning and
emotion, having soaked up through the words. Sometimes the words are less
permeable, so the reader might need to do some fracking -  initiating small,
controlled explosions under the surface to release the content.

But
suppose that route to the surface is blocked - where does the thwarted energy go? Readers might simply give up, cutting their losses. Alternatively,
attention might spread sideways, focusing on the language, so that sound is given the role of generating meaning - not the same type of meaning obtainable through paraphrase,
more the effect of a musical phrase (from now on I'll use ``meaning'' and ``music'' to distinguish the 2 effects). This ``music'' is often present as a secondary
effect in paraphrasable poetry. Take for example the following from Eliot's ``Prufrock''

\begin{verbatim}
Of restless nights in one-night cheap hotels 
And sawdust restaurants with oyster-shells 
\end{verbatim}

This has a clear enough prosaic meaning. Sonically
it's dense too. As well as being a rhyming couplet in iambic pentameter, the interplay of the
S sounds against the T sounds creates a rich texture. This isn't mimetic -
the S sound isn't the wind, or sadness. As in music, the rhythmic variation of 2
contrasting tones generates an effect which is abstract, ready in this case to be infused by what's nearby.

Suppose meaning were partly suppressed - trivialised perhaps? Suppose Eliot had written

\begin{verbatim}
The restless knight's old mum might ask the belles
``will sawdust restore ants and destroy smells?'' 
\end{verbatim}

Where would the reader look for meaning then? I suspect the music might encourage them work harder to find an obscure meaning, though I don't expect them to feel compensated. The sonic texture adds intensity and memorability, but if the meaning doesn't live up to its billing, the result can be comic.

Poets can disrupt the reader's search for meaning in other ways. Brian Reed (2007) introduced the term ``attenuated hypotaxis'' to describe a sequence of ``tenuously interconnected'' clauses and phrases ``possessing some relation of subordination to another element'', but with the connections blurred, ``inhibit[ing] the formation of clear, neat, larger units''. In this type of poem readers are more likely to feel the music  - maybe something like Bernstein's
\begin{verbatim}        
Casts across otherwise unavailable fields.
Makes plain. Ruffled. Is trying to
alleviate his false: invalidate. Yet all is
``to live out'' by shut belief, the
various, simply succeeds which.
\end{verbatim}


The suppression might be more radical than this. Here's part of a poem by Susan Howe
\begin{verbatim}
  amulet    instruction       tribulation 

  winged     joy       parent   sackcloth   ash 

  den   sealed    ascent     flee 
\end{verbatim}

The next step might be Sound poetry. Severed from representation, the poem has to become more self-sufficient. But what incentive will lead the reader away from meaning, especially since, as the Eliot example shows, words can have both meaning and music. Purist concentration on one aspect has artistic worth as an experiment, but does it work as literature? Perhaps representational vs abstract art's an analogy for this. Perhaps a song's lyrics and music provide another analogy - bland lyrics can become intense given a good melody. Put intense words and music together and you often get less than the sum of its parts.



Some poets assert the primacy of music.
\begin{itemize}
\item  ``To [Elizabeth Bishop], the images and the music of the lines were primary. If we comprehended the sound, eventually we would understand the sense'' (Dana Gioia) 
\item ''Bunting would say that you should hear the 'meaning' of the poetry purely in the sound ... Word patterns which may at first appear dense and complicated on the page become articulated and clarified, resonating across the poems' structure. The subtleties and echoes of language which hold a poem together are revealed by the process of sounding it'' (Richard Caddel)
\item  Eliot wrote that ``The chief use of the 'meaning' of a poem, in the ordinary sense, may be ... to satisfy one habit of the reader, to keep his mind diverted and quiet, while the poem does its work upon him''

\end{itemize}

Though music and meaning can be closely coupled, there are benefits in assuming independence - 
\begin{itemize}
\item 
``A statistical analysis which shows that sound effects in Pope are likely to coincide with lexical meanings whereas in Donne there is a discordance, probably intentional, between phonetic effects and semantic units"  (George Steiner)
\item ''The remarkable result of Valéry's treatment of sound and sense as consciously separated variables is that it allows the semantic components of the poem to take on structural value and the structural values of the poem to take part in a semantic or signifying action in turn'' (C. Crow)
\end{itemize}


Should the poet draw attention to the sounds? If they don't, will the reader (even subconsciously) notice them? If the meaning is strong, they might not. Received forms at least alert the reader to attend to sounds. Some forms impinge on meaning, emphasising certain words; others offer opportunities for surprise. In ``Reader's Strategies in Comprehending Poetic Discourse'', P.Begemann suggests that ``Patterns immediately recognised could possibly influence meaning construction from the very start, thus gaining an 'autonomous' semantic function, whereas others may be chronologically and semantically subordinate to lexical meanings''. Factors affecting obviousness of a pattern include ``distance between equivalent sounds; frequency; degree of similarity; size of repeated segment; stress/unstress; statistical frequency of repeated sound; lexical category (function words vs content words); position (on line); stylistic convergence (parallel patterns on other textual levels''

It's not so much that sound and sense compete for the readers' attention, more that one aspect might (perhaps because it's dominant at the start) eclipse the other. Sometimes sound or sense has to be compromised for the sake of the other, though if one is completely absent, readers might not consider the result a poem. If the reader only notices the flaws without appreciating what has benefitted as a consequence, they're not getting the maximum from the poem.


\newpage\subsection{Misdirection}

Discussion of misdirection and distraction has turned up in several contexts during my recent reading, in several contexts
\begin{itemize}
\item \textit{Magic} - "One of the most important things to remember when thinking about misdirection and magic is this: a larger movement conceals a smaller movement" (from Wikipedia)
    \item \textit{Literature} - "Misdirection is also a literary device most commonly employed in detective fiction, where the attention of the reader is deliberately focused on a red herring in order to conceal the identity of the murderer" (from Wikipedia). Of course, many other genres use the device too - e.g. a title has an extra meaning that's only revealed at the end.
    \item \textit{Writing Poetry} - "It is difficult to select just a couple moments in Larkin's work where he employs this strategy of misdirection because it is so often the case. Nevertheless, I will begin with a poem from The Whitsun Weddings called "Water" where the strategy is hard to miss. In it the speaker creates a hypothetical scenario that is initially presented in kind of a jocular vein, but by the end we sense how fully invested, emotionally and intellectually, the speaker is in this thought experiment and it transforms into something other than humorous" (from ``Misdirection and excess in Ginsberg and Larkin'')
    \item \textit{Reading Poetry} - "The chief use of the 'meaning' of a poem, in the ordinary sense, may be ... to satisfy one habit of the reader, to keep his mind diverted and quiet, while the poem does its work upon him." T.S. Eliot, "The Use of Poetry", 1933
\end{itemize}
In these situations the true purpose of the action is being disguised in order to surprise the audience later - to set a time-bomb under-cover. In the following example (and perhaps the Eliot example too) the enemy is logic, reason, narrative, or the self-critic, and the distraction lets other (perhaps more delicate) faculties have a chance.
\begin{itemize}
    \item \textit{Poetry Workshop Exercises} - "This exercise is the old distraction gambit of the card sharp or shell-game artist. Worry about one hand while the other pulls off the trick" (Thomas Rabbitt, from "The Practise of Poetry" by Robin Behn and Chase Twichell (eds), CollinsReference, 1992)
\end{itemize}
These enemy faculties need to be sated otherwise their needs will dominate. If these more mundane needs can be satisfied while performing other tasks, so much the better. The trick is not to let the fulfillment of these needs interfere with the other more delicate faculties. In poetry for example, the reader's desire for tidiness and order might be satisfied visually, freeing grammar from its obligations.

Distraction depends on there being more than one feature or viewpoint. Poetry's good at offering features for diversion - sound versus sense, line versus sentence, form vs content, etc. Distraction's useful because it makes surprise more effective - suddenness matters if twists, punchlines and juxtaposition are going to work. People don't want to see a scene constructed piece by piece before their eyes. Better that the lights go out between acts so that the new scene can suddenly appear. The distraction's like an egg-shell, protecting as well as hiding the growing entity within.

The risks are that distraction can look like showmanship or mannerism, or it can seem to be unintentional flitting, a lack of focus. Also its effects can be hard for the writer to predict: they'll depend on the experience of the reader - a writer's supposedly subtle clue may look clumsily obvious to some readers. The reader might be supposed to see the distraction for what it is - part of a double bluff perhaps, or maybe the writer's going to focus the source of interest in something other than tension.

In all these cases, the reader's ultimate enjoyment is what matters - all is revealed at the end. Contrast that with the psychology tests where the subjects often aren't told the true purpose of the experiment beforehand because self-awareness will affect the outcomes - the joke's on them. Some literature might be like that too - the author or critics having the last laugh: "The Name of the Rose" perhaps - a medieval whodunnit or wicked, pretentious satire? "Finnegan's Wake" - comic masterpiece or flop? 




\newpage\section{The Real World}

How might be poem fare when released into a hostile world? Do everyday terms like truth and precision have a place in poetry? Is science a friend or foe? What affect on writing has the emergence of Flash. Do poets care if their work doesn't impress intelligent latpeople? How can poets establish their aesthetic authenticity?

\subsection{Allusions}

"The test for allusion is that it is a phenomenum that some reader or readers may fail to observe" [12, p.39]

Allusions are far from being the sole preserve of literature. Cinema, painting and music frequently contain quotations from other works. Early still-lifes in particular depended on a rich vocabulary of symbols which many admirers are unaware of today. But how important for the contemporary reader is the awareness of a poem's allusions? And how has this importance changed?

\subsubsection*{The Mechanics of Allusion}

According to Ben-Porot [5, p.109] the process of a reader's actualisation of an allusion involves
\begin{itemize}
\item    recognition of marker
\item     identification of evoked text
 \item    modification of the initial local interpretation of passage
 \item    activation of evoked text
\end{itemize}
Full actualisation may be frustrated at each stage -

\subsubsection*{recognition of marker}

If an allusion is disguised or unobtrusive (it doesn't appear in quotes, it has a tempting non-allusive interpretation, etc) the reader may not realise that it exists. Some poets may use this ploy to satisfy certain readers for whom "the pleasure of recognition [is] proportional ... to the difficulty or unobtrusiveness of the allusion". I.A. Richards said that these ploys are "not to be confused with literary or poetic values" [13, p. 170] but it's at least "tactful" (as Empson called it [7, p.167]) of the poet to give the words that form the allusion a meaning in their own right. This, however, increases the risk of the allusion being missed and if the intrinsic meaning is plausible but weak, the reader may miss much. The poet may intend the reader only to recognise the allusion later, or for only a part of the readership to pick up the allusion - examples are dramatic irony, in-jokes, pantomime asides and innuendo. Plagiarists hope that the marker won't be recognised at all.

Some poems can survive the loss of this allusive power - indeed the reader's attention may be profitably focussed back into the text (for instance, some parodies work even if readers are unaware of the original. Yet there's a certain effect produced by references out of the text, whether they're spoofs or not; they show ways out of the poem, and often ways back in.

\subsubsection*{identification of evoked text}

There is no longer a canon of work that the reader can be expected to know - readership is wider, the Bible is less popular, and there are more books. Modernist authors are more likely to allude to obscure, private, ephemeral or even non-existent texts. When a text refers to many, widely ranging texts, noting one allusion is less likely to prime the reader for the next. To circumvent this, some poems explain even well-known allusions within their text or by footnotes.

\subsubsection*{modification of the initial local interpretation of the passage}

Pre-modernist poems more often than not had a primary meaning, perhaps based on initial observation. Once established, this meaning could pull in an allusion without being overbalanced; the poem's centre of gravity remained within the text. Nowadays, poems need no longer establish a solid melody before improvising. Some set a foundation by alluding to the canon or genre, others don't even try. There has been a re-ordering of linguistic priorities: common, denotative meanings becoming secondary. Where there is no primary meaning, significance is distributed. The poem loses its surface, its graduations of depth. The allusions more prop up than dangle from the poem. Attention is diffused.

\subsubsection*{activation of evoked text}

"While reading text, readers establish local coherence in short-term memory - small scale inferences from few small units of information... These hypotheses are refined as the reading of the text proceeds ... In semantic memory, each concept is connected to a number of other concepts. Activating one concept activates its adjacent concepts which in turn activate their adjacent concepts. Thus, activation spreads through the memory structure, determining what is to be added and what is to be removed from the interpretation of text. This process continues until further activation of adjacent propositions does not change the propositions used to interpret the text." (quoted from [3] which in turn acknowledges [15]). Activitation spreads more easily through and beyond the remote text if there are repeated references to the same text (parody) or if the remote text is far more interesting than the base text, especially if the base text lacks coherence.

Wallace Stevens, in the context of metaphors, said "The proliferation of resemblances extends an object. The point at which this process begins, or rather at which this growth begins, is the point at which ambiguity has been reached" [14]. If what's evoked merges into the presented text, the text becomes "an entrance into a network with a thousand entrances" [4, p.12], and we are as likely to take a further leap away than return to our point of departure. Alternatively, the text alluded to may not so much extend the original text as help create a new third entity. As in "Surrealist metaphor, two terms are juxtaposed so as to create a third which is more strangely potent than the sum of the parts...The third term forces an equality of attention onto the originating terms" [16, p.73-74]. When ordinarily unassociated elements are juxtaposed, the reader is called upon to determine. But if this determination is not logically possible, if the relation between the two is undecidable, something else appears in this gap. Eliot and Pound even spoke of "emotion" in this context.

\subsubsection*{The Allusion Field}

Collage and fragmentation have opened cracks into the text, dissolving the text's boundaries. External references are given equal weighting to internal ones, thus destroying any chance of "organic form" (with its internality, assimilation and wholeness). Inter-textuality's one of the most celebrated concepts of post-structuralism. Barthes considered it a "prerequisite for any text", even for Language Poetry, and thought that it "cannot be reduced to a problem of sources and influences; it is a general field of anonymous formulas whose origin is seldom identifiable, of unconcious or automatic quotations, given without quotation marks" [2]. It's more a "semantic atmosphere, or milieu, rather than the possessive individualism of reference" [1, p.36], the particles of reference becoming a field of allusion. The kind of texts that best display these traits are only now coming online. Landow's noted that literary critics and hypertext theoreticians both "argue that we must abandon conceptual systems founded upon ideas of center, margin, hierarchy, and linearity and replace them with ones of multilinearity, nodes, links, and networks." [10] He goes on to say that "Both were looking for solutions to the limitations of the linear, static, discrete texts of the print tradition. They wanted to liberate text from given context. ... both schools advocate 1) treatment of text as small units or lexia, a term used by Barthes, 2) networking and linking of these units, 3) de-centering and equalizing, 4) non-linearity, and 5) interactivity and blurring the line between reader and author." The shared terminology can be taken further

\subsubsection*{Cohesion and Coupling}

In the communication age when texts old and new are easily drawn into the allusion field, closure has become less certain. The effects of this on textual cohesion have been studied. Childs mentions 4 types of textual cohesion: phonic, grammatical, rhetorical and semantic [6, p.98] and points out that modernist texts tend to use different types of cohesion to earlier texts. Analysis of inter/intra-textuality has been matched by work in computer science. In his analysis of the structure of computer programs, Yourdan [17] lists various levels of cohesion that the lines of a module can have; from the weakest to the strongest they are
\begin{itemize}
 \item   Coincidental
 \item      Logical (they all do the same type of thing)
  \item     Temporal (they all need to be done at the same time)
  \item     Procedural (they form a single task - like parts of a prose description)
  \item     Communicational (operate on same data)
  \item     Sequential (the result of one line feeds into the next - narrative)
  \item     Functional (all must be done for something to work)
\end{itemize}
If one ignores phonic cohesion then there's a fair but not strong match between the 2 taxonomies.

Computer modules also differ in the extent that they interact with each other. Programmers aim to write modules that have strong cohesion and weak inter-module coupling. Extending these concepts leads to a way of classifying poems. The traditional sonnet tends have strong cohesion (emphasised by the form) and is weakly coupled to other poems, whereas a typical modernist piece (The Waste Land) has weak cohesion and strong coupling/inter-textuality. The cohesion of computer modules isn't easily measured or characterised, but one can make a rough assessment of a poem's cohesion and coupling by comparing the number of internal and external references. If a poem's cohesion is strong then it can survive allusions being missed (and indeed, they're more likely to be missed).

\subsubsection*{Examples}

A few illustrations will show how the reader's need to trace allusions varies.
\begin{enumerate}
\item 
\begin{verbatim}
A CARAFE, THAT IS A BLIND GLASS - Gertrude Stein

A kind in glass and a cousin, a spectacle and nothing strange a single hurt 
color and an arrangement in a system to pointing. All this and not ordinary, 
not unordered in not resembling. The difference is spreading
\end{verbatim}
The weak semantic and formal cohesion are hallmarks of a modernist text. The weak coupling in this example puts a further strain on the reader.

\item 
\begin{verbatim}
Her smile - Tim Love

Her smile as she falls asleep -
a bird always
landing on its shadow.
\end{verbatim}
Weak coupling (drawing on the genre but not on any particular text) and strong semantic cohesion typifies the lyric poem.
\item 
\begin{verbatim}
Animal Lover - Tim Love

Dolphins always smile my way on salmon-
Chanted evenings and hawkmoths wink all night.
O stuffed dodo do what you done done done
Before, you toucan with your songs delight.
Yes, I confess my sole intent; to wit
To woo beasts two by two, pander and bare,
Cheer the worm's turn, watch horses get a bit
On the side, then, with swallowed pride, home where
My faithful quick brown fox jumps just for me.
How long can this go on? Is my fate sealed?
Oh deer, paw me. But if I must I'll flea -
I'll go to the dogs or pick up booze-swilled
Slugs then flock to packed terraces and crow
"O earwig, O earwig, O earwig O!"
\end{verbatim}
The form and thematic continuity do little to disguise the lack of higher level cohesion. The extensive use of puns to convey references means that coupling isn't wholly at the expense of the primary text. The references lead nowhere and don't interlink; if a few are missed it's not the end of the world. They're one-way - the poem doesn't suck significance in from distant texts. The combination of weak cohesion and numerically strong coupling is common in ludic and post-modernist work.
\item 
\begin{verbatim}
Mummy's Boy

Each Sunday, visiting the home,
you take a pocketful of dust,
praying she won't see it slip away.
Back home you dig with teaspoons
a secret tunnel that only you can ever use.
\end{verbatim}
This is almost a riddle poem. There are 3 allusions to a single key external image - that of a prisoner of war preparing for escape - making for focussed coupling and, in Yourdan’s terms, strong communicational cohesion. Those who know about ‘The Wooden Horse’ have a distinct advantage when reading this piece because attention isn’t drawn to the allusions. The semantic cohesion and a measure of pattern give others some chance of satisfaction, enough to make them think that they haven’t ‘missed something’, although they have. Were it called ‘The Great Escape’ the allusions are less likely to be missed, but the relationship between the two people would be less certain.
\end{enumerate}

\subsubsection*{Textual metrics}

In order to compare poems it’s useful to count the number of internal and external references, as well as the number of texts alluded to. The method needs to be simple to be useful. However, there are some difficulties involved.
\begin{itemize}
\item    \textit{Number of external references} - should an obscure allusion count as much as an explicit allusion? Here, a simple, unweighted count will be used.
  \item   \textit{Number of external sources} - should an allusion to a haiku count as much as one to a novel? Here they will both be worth the same.
 \item    \textit{Number of internal references} - This is the hardest metric to evaluate. As we have seen, there are many forms of cohesion, some much stronger than others. If a rhyme counts as 1 point, how much should a 10 line description count for? Moreover, different people will have their own idea about the relative importance of links, rhyme, lists, description and narrative as cohesive forces. Subjective impressions are hard to avoid (even software engineers use them). The scheme used here gives one point for: each pair of semantically linked words; each well-formed sentence; each sentence that forms part of a narrative. There are half-points for: each sentence that forms part of a list; each list. The sentence is used as the unit of syntactic binding. Perhaps the clause would be better. Indeed the measurement of coherence could be made far more elaborate, but a relatively easily derived count is sufficient for our purposes.
\end{itemize}
To see whether these metrics confirm our intuitive responses, let's evaluate the above sample of poems.
\begin{table}[htbp]
\centering
\begin{tabular}{|l|l|l|l|l|l|}\hline
         & Int\% & Ext\% & Sources & Ext-Int	& Ext/Sources\\\hline
Carafe	& 	7	& 	0	& 	0	& 	-7 &	-\\\hline
Her Smile	& 	27	& 	0	& 	0	& -27 & -\\\hline
Animal Lover	& 	12	& 	6	& 	6	& 	-6 & 1\\\hline
Mummy’s Boy	& 	24	& 	9	& 	3	& 	-15 & 3\\\hline
\end{tabular}
\end{table}
Those who evaluate bridge hands will find the procedure familiar. "Mummy’s Boy" for example has 35 words (including the title). There are 3 references to 1 text (‘The Wooden Horse’). 3/35 is 9\%, 1/35 is 3\%. There are 4 linked pairs (sunday-praying, praying-dust, home-home, dig-dust) each scoring a point. Each of the two sentences is well constructed, the syntactic binding meriting a point, and the sentences connect, forming a narrative which according to my points scheme adds another point for each sentence. That totals 8 points. 8/35 is about 24\%.

The Ext-Int column gives an indication of the allusiveness of the text - a difference seems more appropriate than a ratio. The more negative the number, the more that internal references dominate. If both the Int\% and Ext\% values are high, allusions are likely to be missed. If the values in the Int\% and final columns are high, then there may be an easily missed key text. "Mummy’s Boy" is such a poem. "Animal Lover"’s allusions are less likely to be missed because there are so many sources.

Modernism and post-modernism have increased the variation of these individual parameters and have brought into being texts with new combinations of these parameters, increasing the range of reader responses. Nevertheless a simple metric seems sufficient to predict reader response to the allusions within a text.

\subsubsection*{Bibliography}
\begin{enumerate}
\item "The L=A=N=G=U=A=G=E Book", eds B Andrews and C. Bernstein, Southern Illinois University Press, 1984.
\item "Encyclopaedia Universalis", vol XV, 1973. Barthes
\item \url{http://www.isg.sfu.ca/~duchier/misc/hypertext_review/chapter1.html}, V. Balasubramanian
\item "S/Z", Barthes, New York: Hill and Wang, 1974.
\item "The Poetics of Literary Allusion", PTL: A Journal for descriptive poetics and theory of literature 1, Ben-Porot, 1976.
\item "Modernist Form", J. S. Childs, Associated University Presses, 1986.
\item "Seven Types of Ambiguity", W. Empson, Hogarth Press, 1984
\item "Intertextuality, allusion, and quotation: an international bibliography of criticial studies", compiled by Udo J. Hebel, Greenwood Press, 1989.
\item "Literary Quotation and Allusion", Kellet, 1933.
\item "Hypertext: The Convergence of Contemporary Critical Theory and Technology", George P. Landow
\item "The Poetics of Quotation in the European Novel", Meyer, Princeton
\item "The New Princeton Encyclopedia of Poetry and Poetics", Preminger and Brogan, Princeton University Press, 1993.
\item "Principles of Literary Criticism", I.A. Richards, Routlege and Kegan Paul, 1961
\item "The Necessary Angel", Wallace Stevens, 1942
\item "Hypertext '91 Proceedings", Thuring, Manfred, Haake, Jorg M., and Hannemann, 1991.
\item "Statutes of Liberty", Geoff Ward, Macmillan, 1993
\item "Managing the Structured Techniques", E. Yourdon, Prentice-Hall, 1979.
\end{enumerate}


\newpage\subsection{Charlatanism, Poetry and Art}

Hoaxes happen often in Poetry and in Art - lost works by masters 
such as Shakespeare and Vermeer are rediscovered only to be revealed
as frauds by stylistic or forensic analysis. For the hoax to work the copied
master needs to be unavailable for comment - either by being dead or
by never existing. Literary examples of the latter include Ossian (Scotland)
Ern Malley (Australia) and more recently Araki Yasusada (Japan) - all created
by skillful hoaxers who are frequently surprised that their work is taken seriously.



Writers too develop their characters and self-promote. They may invent 
personae and pastiche works. They may even
identify with the persona and take the persona seriously. They 
flourish in similar environments to those where hoaxes profilerate.
They induce criticism because to some their fame or wealth seems undeserved,
even fraudulent, distracting attention from the quality of the work itself.
Such criticism is enflamed by artistic flamboyance or media hype. The critics
and members of the public  
with few (typically one) criteria of judgement - verisimilitude, for instance -
are amongst the most vociferous. The supposed charlatans sometimes emerge as
important artists, or at least (as with Dali) the case remains open, so in
ideal conditions one doesn't want to be too quick in starving these artists of
funding. Nevertheless, with competition for public money increasing,
it's worth exploring the public reception of "charlatans", in particular 
in Art and Poetry.


 
The potential for charlatanism in Modern Art is greater than in Modern
Poetry, and public reaction is greater. The reasons for this fall into
3 main categories -

\begin{itemize}
\item \textit{Ease of production}
\begin{itemize}
\item  In Art, it's easier to get away with having little technical skill, thus
making entry-level charlatanism simpler. When a praised
  piece of abstract expressionism turned out to be done by an chimp, some
  critics still defended the quality of its Abstract Expressionism.
\item  Moreso than literature, Modern Art deals with the ephemeral (pop art), 
  the 
  secondhand (postmodernism), and the trashy (kitsch art). Also the mere act of presenting an object constitutes a "treatment",
an artwork.
\item  Because translation may not be necessary, artists can more easily and quickly borrow from styles distant in time and space. Being ahead of their audience they can more easily present the 
  merely exotic as art.

\end{itemize}

\item \textit{Definitions and Boundaries}
\begin{itemize}
\item  Art has changed more quickly than poetry, with more artists than 
poets challenging boundaries. History has more famous artists once
  dismissed as charlatans than it has resurrected writers, thus 
making critics more hesitant about criticising charlatans. But could
critics have made any difference anyway? Matthew Collings
writes (p.121) that "the 80s ... was the decade when critics were laughably weak and 
galleries and collectors were shockingly strong"


\item  Art still has a cult of Authenticity, a desire to create (and own) the 
  unreproducible - leading to  Installations, Happenings, etc that cross
media boundaries.
\item  Many things are described as "Poetic", but poetry books are still texts.
Use of the term "Art" has widened, and because the media
used in Art has widened, many of these products termed "artistic" can be 
accommodated in Galleries - or anywhere.

\item  Genre and Art/non-Art boundaries are unclear both in literature and Art,
  but Performance Art and Conceptual Art have made the boundaries fuzzier. If
  2 people carry a plank through a city centre and call it art, can we
disagree?
  Does a voyeur become an artist by making a project from it and keeping a
  diary? The Pornography/Erotica dimension in particular troubles people, with writers
  like Henry Miller teasing the boundary as much as artists do. In the last
decade several stories have hit the headlines both in the UK and
the USA, making grant-giving bodies cautious. Photographers and performance
artists predominate -
\begin{itemize}
\item Robert Mapplethorpe (photographer of "homoerotic images")
\item Jock Sturges (photographer of nude young children, 1990).  A federal
grand jury failed to indict Sturges, and his career was enhanced by
the notoriety.
\item Marilyn Zimmerman  (tenured professor at Wayne State University, 
photographer of her nude daughter, 1993). Charges dropped, but her ex-husband
used the photograph controversy to gain primary custody in court.
\item Natsuki Uruma (Performance Artist, pole-dances to London tube travellers, 2000)
\end{itemize}

That shocking the public (or "making them think") can be artistic
comes partly from Surrealism. Breton himself said that 
"Surrealism attempted to provoke, from an intellectual and moral point of 
view, an attack of conscience, of the most general and serious kind". 
The creator's genre classification of their work is a hint
  about the way the viewer might approach the work. It may be (deliberately) unhelpful or provocative. We needn't 
trust what creators say about their work. 
  We needn't believe their claim that their own work is
  "Wonderful", so why should we heed their classifications - and intentions? 
And if we don't, where does that leave the work's value?
\end{itemize}
\item \textit{Social Factors}
\begin{itemize}
\item  There's big money to be had in Art, especially in the USA. In addition
to the tax breaks for private collectors, several
states and some 40 cities have a "percent for
art law" that require a percent or more of public building
projects be set aside for the purchase of public art. The giving of
public money to artists has brought the issue of the nature of Art into
the open - when Art and Hospitals are competing for funds, public reaction
can be heated. 
Studying public art controversies has itself become a growing field,
represented by a flood of books and studies that aim to help agencies
head off complaints before they occur and lessen their intensity after
they arise. See \url{http://www.thenation.com/issue/991129/1129grant.shtml}
The Nation's article for details. Poetry developments have been
shielded from the glare of publicity.
\item  There's more Minimalism in Art, which is provocative - though people sympathetic 
to the Arts might 
  think it fair enough that one should take time with overtly obscure and
difficult works like the Wasteland or Ulysses, 
  many would not be prepared to stare at a pile of bricks too long.
  And it's not unusual for gallery furniture to be mistaken for pieces of
  art. Matthew Collings writes (p.225) "Relatively recently the assumption was that there was no point  in
thinking about contemporary art because it didn't mean anything to
anyone except artists... Now there is a growing anxiety that there might
not be any point to it because its meanings are too available and also
too available elsewhere".
\item Images have more of an impact than words.
\item Mainstream poetry is closer to public sensibility than mainstream Art;
literary avante-gardism is peripheral even to the literary world.
Whereas winners of major art awards (the Turner prize in the UK, for example)
initially provoked anger, poetry winners are welcomed by indifference.
Now, as Matthew Collings points out (p.226) " the Turner Prize .. is an 
amusing talking-point, a laugh on the cultural calendar, but not an outrage".
It's something that the media enjoy and thus sustain. 
\end{itemize}

\end{itemize} 


Where Modern Art is, will Poetry follow?
\begin{itemize}
\item \textit{Theory}

 Theoretical approaches are 
converging.
 When evaluation of a work emphasises issues like Ethnicity, Gender, Power
and Politics, other more aesthetic issues can be neglected. Novelty is
prized not only in the works but in theories about the works, and the more
theories there are, the easier it is to justify a piece.
\item \textit{Contextual devices}

Poets
are making more use of contextual devices.
 Art (Duchamp, etc) exploited the "Gallery
  Effect" (the changed perception of an audience when in aesthetic mood)
  long ago with his readymades.
  Audience reaction has become part of an artist's work - the work is
  incomplete without it. Part of the "art" of being an artist is being
  able to judge the right time for such a piece of work. Picasso, for
  instance, had the idea of producing a blank canvas long before someone
  did it (he also had the idea of coating common objects with fur) but
  maybe he felt that part of the work (namely the audience) wasn't ready.
With "Found poetry", poets are catching up. 
\item \textit{Reaction}

Publicity stunts are becoming more
common (the Dada poets and Dali providing role models), and shock value 
sells books. Bluff and double bluff are on
the increase. Suspected pseudo-science and pseudo-intellectuals are attacked 
by their counterparts, but suspected pseudo-artists are only attacked by
non-artists. Sometimes there are theoretical backlashes (see for
instance the \url{http://www.aristos.org/editors/booksumm.htm} summary
of "What Art Is: The Esthetic Theory of Ayn Rand") but the art world is
self-sustaining, and it's hard to see how any re-evaluation is likely
to succeed. 
\end{itemize}

Perhaps charlatans are nothing to worry about. After all, they
are sometimes only sincere, pretentious creators who have gambled and lost
in the posterity stakes - we need artists who gamble. But I think it's 
worth comparing the ability of intellectual pursuits to deal 
with charlatans. The sciences in the main deal with them rapidly. Art
seems slower than Poetry at coming to a settled judgement. This is laudable
in that all such judgements are provisional, but if public funds are limited
and Art gets much more funding than Poetry, Art has at least some obligation
to show that in more ways than one it takes fraud seriously. 

\newpage\subsection{The Great Poetry Hoax}

Schools and the legacy of Leavis encourage the idea that poetry teaches us something about life. Poetry traditionally is treated with tolerance and reverance, and thought of as a vehicle for eternal truths and deep emotion, so some people cast their opinions into poetry to make their thoughts deep and eternal. Populists and propagandists, knowing that people are more gullible to poetry than to prose (and more gullible still if it's set to music), cast their message in an overtly traditional poetical form or use overly poetic phrases, hoping to cash in on poetry's heritage and bask in the aura of the greats.
 

But it's not only Lady Diana-lovers who do this. Good writers too know that calling one's short prose poetry is a useful (sometimes the only) way to get it read. As well as gaining respect by categorical association with past works, a poem offers the writer many ways to make the reader believe that the poem is significant - e.g.
\begin{itemize}
\item putting bold, unsupported declarations at key points (end of the poem or stanza) - see Keats' "Ode to a Grecian Urn"
\item adding some emotive words in key places - see Gray's "Elegy Written in a Country Churchyard"
\item name-dropping - an initial quotation or dedication; allusions to great works.
\item dealing with a serious subject (death, god, poetry, etc).
\item using extensive white space (a form of underlining) around key words.
\item adding gnomic semantic gaps, removing punctuation or fracturing the syntax so that the reader has to slow down - readers will tend to justify time
spent on a text. 
\item using obscurity or ambiguity to overload the processing of language -
if readers find a text hard to read, they may think it profound.
\item repetition and chanting - see Walter de la Mare's "The Traveller"
\item ostentatious display of credentials and awards, authenticating the text so
that readers question their own abilities rather than the poet's.
\end{itemize}

The above list of ploys includes features that some have described as defining characteristics of poetry  - 
\begin{itemize}
\item "the machinations of ambiguity are among the very roots of poetry", 
Empson, Seven Types of Ambiguity
\item "poetic effect ... is an acculumation of side-effects", Sperber and Wilson,
"Relevance" 
\item "The technique of art 
is to ... increase the difficulty and
length of perception, because the process of perception
is an aesthetic end in itself and must be prolonged. Art is a way of
experiencing the artfulness of an object; the object is not important", 
Shklovsky, "Art as Technique"
\end{itemize}

 Poetry seems rich in such facilities, which is perhaps why readers often think a poem deep but can't say why. As so often, evolution may provide an explanation.
 

The human brain is an amazing organ, one that's developed through evolution from being able to clobber mammoths to knocking off sonnets.  However, evolution never does any more than necessary. As sight evolved, there was no survival advantage in being able to deal with situations that didn't happen in real life - our eyes might be able to help us distinguish friend from foe at a hundred yards, but they're easily tricked by optical illusions. Thought and language processing is even more complex than visual processing, so perhaps it's not surprising that poetry can create an illusion of depth and meaning by short-circuiting the normal routes (much as stereograms give the effect of depth though they have none), exploiting a loop-hole that evolution has left open. As with stereograms, surface obscurity may be necessary to produce the effect, and there's a lot of skill involved in producing an effective illusion. Indeed, I'd say that not merely skill is involved; it's an art. 
 

The intelligent layman no longer looks to poetry for insights into our times. The only surprise is how poets got away with it for so long. Yeats' "Easter 1916" and Eliot's "The Wasteland" may have summed up a generation's hopes and fears, 
but that was long ago. The game's up now. Even poets admit that poetry's explicit statements aren't always to be taken at face value. TS Eliot said that "The chief use of the 'meaning' of a poem, in the ordinary sense, may be ... to satisfy one habit of the reader, to keep his mind diverted and quiet, while the poem does its work upon him."


There's still much of interest in poetry but we shouldn't kid ourselves that poetry will solve our problems or have anything to say about the human condition; it says more about human conditioning. Something that provokes feelings of profundity needn't be profound. The modern lyric simply doesn't have sufficient width to support such depth. Let's leave profundity to prose.


\newpage\subsection{The Poetry Circus}

In a secondhand bookshop I found "The Poetry Circus", by Stanton A. Coblentz, (Hawthorn: New York, 1967). It is "a frontal attack on the sloppiness, pretence,  and just plain sensationalism that prevails in much of contemporary poetry". In one section ("How to write a Modern Poem") he shows how an embarrassingly bland text (e.g. "\textit{Every nation, isolated in its own house, seeks to wall out all other nations}")
might be modernised by substitutions leading to something that "may be a little vague and somewhat hard to figure out, perhaps even contradictory, but no one will say it is trite" (e.g. "\textit{Every nation/in the isolation of its own libido/seeks to cro-Magnonize all others with the psychology of the alter ego}"). Of course no-one consciously proceeds through these stages, but poem explanations sometimes perform the reverse process. Are they attempts to normalize, to remove from the work all that's odd to us, all that's novel?  Do they dumb down? Whatever the explanations do, they don't always explain what's lost in this process. The paraphrase may even be an improvement on a confusing draft.

Coblenz's battle against the Emperors' New Clothes failed to change the course of US poetry, but I share some of his doubts about the purposes of difficulty. There are several reasons why a vague or difficult poem might be more effective than a direct one -
\begin{itemize}
\item  In "Nature", 17th Mar 2005 they reported that blurry images can have more emotional impact than clear ones - the emotion-modules in the brain don't need detail; the detail activates other activity that might distract.
\item  A "Rorschach" poem can give a reader more scope for imagination.
\item  Exploring a difficult poem can be a reward in itself.
\end{itemize}

- but vagueness and difficulty can of course be evasion, bluffing, or a sign of more general communication difficulties.

I read the "Tears in the Fence" magazine to encounter types of work I don't often read, work that challenges and stretches me. Both poets and critics are given space to make their case. Even so, I sometimes feel that I'm encountering crop circles rather than new life-forms. Here's the start of "Love Poem 2" by Lisa Mansell
\begin{verbatim}
slick in the lactic stale of sextet
                                        they crabform in their calculus
        and listen to the music that kilts and sucks their scarab-wracked skin
               tantric and crystal                a tryst
                                               rustic and cusp

oceans slip by denizens of noose
       and coil unctuous          vultured in love-letter-scrawl
                       as laval scions the deserting vulva
         aztec and volatile                        liquid
\end{verbatim}


It's rich in sound effects - "slick in the lactic stale of sextet" has many 'S' and 'K' sounds that are repeated through the piece. Later, 'L' and 'V' sounds begin to dominate. Sound has its own meaning-making mechanisms. The dadaists wrote "sound poetry". Less extremely, Mallarme and Basil Bunting foregrounded sound. I side with Eliot when he said "the music of poetry is not something which exists apart from its meaning". The balance between sound and "sense" can vary in poetry. In this piece several word-choices look strange if one merely considers their sense. Why "slick in the lactic stale of sextet" rather than "slick in the lactic staleness of sex"? What does line 2 mean? "calculus" might be something to do with bones or with calculation. What sort of "music" is being referred to? What is "kilts ... their skin?" (by the way, "unctuous" means oily or smarmy; "scions" means offspring). What does "deserting vulva" mean - that it's going dry? that it's making something else dry? that it's leaving? How do any of those interpretations connect with the rest of the line (presumably they do in some way, otherwise there'd be a line-break). The poem's grammatically parsable, but commas have been replaced by line-breaks. That  doesn't fully explain the splattered layout though - why the inline spaces?

I like the sound of it - I can imagine people being seduced by the sonic constellations alone - but it might as well be in a foreign language for all the "sense" I can make of it. When writing a Rorschach poem it helps to retain some referential clues - partly to tantalize. But readers aren't to know whether there's a riddle to be solved, or how much work is expected. Here for instance "kilts" could suggest the swaying of seaweed, or maybe it's something to do with "kilter" (as in "out of kilter"). The lovers could be whales, "slick" could allude to "oil-slick". Perhaps vultures and scarabs are Aztec symbols (there must be some reason why "aztec" is there). There are other symbolic links too - crab, ocean, liquid; sex(tet), tantric, vulva; desert and ocean. The rest of the poem doesn't help me, though there is "\textit{their unbelief in binary rubs at the solent-soft of her love}" which reminds me of the concocted examples of modernized poetry that  Coblentz developed from a simple statement. 

Formalist poems are sometimes accused of being rhyme-driven, with artificial inversions introduced merely to regularise the rhythm.  Mainstream poems often have mundane settings into which some mystery is embedded (a "lift" in the final line, for example) with sound having a minor role. There are sonic forces driving this poem like an ambulance siren pushing mainstream meaning aside. Or alternatively one could say that  the setting is sonic, generating effects (a field, if you wish) that isolated referential meanings expand into and modulate.


Rather than send in the clowns perhaps it's time to start dismantling the Big Top. Whose fun would it  spoil? Would it throw the baby out with the bathwater? Does it risk the accusation of being right-wing, reactionary, nostalgic for "The Movement"? Even suggesting that one create a table of pros and cons for the special effects displayed in such poems  risks being accused of over-rationalism, of workshoppery taken over by accountancy. The poet has a Ph.D and lectures in Creative Writing so I presume all my points have been taken into account. In the end all one can say is that the proportions of the ingredients don't suit me. Maybe there are also some ingredients that I'm oblivious to. It takes all sort to make a world and I'm sure this poem has its share of fans.


\newpage\subsection{Attention to Detail}
The mouth of Franz Hals' "Laughing Cavalier" is a smear of red highlighted with what looks like toothpaste. Close up you'd be hard pressed to say what the marks represented. By 1851 that style was out of date. Late one summer's evening, Millais rowed across a river to sketch the flowers on the far bank by lamplight. You'll see those impossibly detailed flowers in the Tate's "Ophelia". In contrast there's Picasso's drawings - bare lines on a white expanse - and Schiele's work where areas of untouched canvas compete with highly wrought flesh. Later still came hyper-realism, where painting aspires to photography.

When a poet writes "he rowed across the river to paint flowers", readers are trusted to fill in the details if they wish. It doesn't matter if they erroneously think that daffodils proliferate on the riverbanks. The poet's "flower" is that of a toddler's first doodle: a generic icon. If the writer says that the flowers are "yellow", we wonder why that feature is emphasised. Writers cannot mention without pointing, whether it's at objects or their secondary properties. Because of that, poets have to be as careful as pre-budget chancellors about what they say. Every word matters, but they don't tell the whole story. In contrast, painters cannot paint mere "flowers" - they have to be yellow, drooping, or windblown. Because there are always so many details, none can be singled out.

The world's becoming more graphic, less poetic - we are offered a choice of details to concentrate on rather than being trusted to fill them in. Any lack is a vacuum that must be filled: we have to know the history of each killer, interview those they knew at school, know in which beauty spot the body was buried. If we mistrust what we see our response is to zoom in. There are always more details to find, more trees to obscure our view of the woods. Yet we are scared to get away from it all. Walking in the countryside, we take along our binoculars, our mobile phones and Sunday papers. Even if we notice hosts of daffodils we would no longer describe them as golden - in truth they aren't. Yet facts are often no more than props for poets to feel their way into a poem, scaffolding. Like an actor's false nose, they are as much for the performer as the listener. Like experimental findings, they are useful only for what we can derive from them. We are becoming obsessed by the fine print and the appendices, forgetting the abstracts. Too many writers play safe by giving us everything on the principle that more is better, betraying a lack of self-confidence, an inability to select. The law of diminishing returns applies, further details becoming exhausting rather than exhaustive, obscuring rather than illuminating the original, squeezing the reader out of the text. Picasso was right when he said that it takes a master to know when to stop.

Art ranges from Pure Abstraction to hyper-realism. Poetry too has abstract forms (Schwitters' sound poetry for instance) but its range (excepting perhaps dialogue) doesn't extend to the reproduction of the real world. Poetry has to accept that it can present appearance little better than it can smells, that because the natural world has to be translated into words, all aspects of reality are equally available, equally distant. Poetry needs to combine the untouched abstraction of "flower" and "summer" with the selective power of adjectives, the pinpoint precision of quotation, and especially of proper nouns. The blanket-bombing of Millais is absent. Instead, details are pruned back to let the spaces speak, and lines define not just area but volume. Under magnification the phrases and words may look mundane or even careless but the pieces aren't meant to be observed in isolation. Each word is modified by its context. Meaning is distributed, oblivious to word boundaries and even the boundaries of the work. When a modern poet writes "golden daffodils" the extra meaning isn't discovered by zooming in on the individual words, but by panning out to take in Wordsworth's poem and our response to the Romantics. The measure of a poem's precision is not the amount of detail it contains, but how well it targets the factories of knowledge in the hinterlands of the reader's mind, where the details are best left.


\newpage\subsection{Factoids}

Well, that's what I'll call them - facts (sometimes contextless and isolated) that are put into poems. They can be interesting in their own right - strange but true. They can be a piece of information everyone's expected to know (e.g. "London's the biggest city in England"), the reader thus expected to ponder on the implications (easiest city to be lonely in?). They can be a minor piece of knowledge shared by reader and poet, perhaps a piece of public knowledge that had a particular significance to the poet. As an example of their usage, here's the first and the final stanza from "Then in the twentieth century" which won 2nd Prize in the 2002 National Poetry Competition. It's by David Hart.
\begin{verbatim}
Then in the twentieth century they invented transparent adhesive tape,
the first record played on Radio 1 was Flowers In The Rain by the Move,
and whereas ink had previously been in pots, now it was in cartridges.
...
Men quarrelled about scrolls found in pots near the Dead Sea, the library
at Norwich burned down, milk was pasteurised by law, I have four children,
all adult now, small islands became uninhabited, Harpo never spoke on film.
\end{verbatim}
Most of these factoids seem to lack any special significance to the persona, nor do they seem closely inter-related. There's not really any narrative either. There's some theoretical justification for this approach. Facts help to anchor the poem to the verifiable world, and are never really isolated.
\begin{itemize}
\item    "although it is possible to reach what I have stated to be the first end of art, the representation of facts, without reaching the second, the representation of thoughts, yet it is altogether impossible to reach the second with having previously reached the first", Ruskin, "Modern Painters"
\item    "Structuralism ... starts off from the observation that every concept in a given system is determined by all other concepts of the system and has no significance by itself alone ... there is an interrelation between the data (facts) and the philosophical assumptions, not a unilateral dependence", Garvin, "a Prague School Reader on Esthetics, Literary Structure and Style"
\end{itemize}
Supposed facts may be loaded with implicit assumptions. There's psychological justification too - after all, what we remember isn't just the personal, or the personal responses to public events, we also remember public events much as many others might.

Some poets never use this approach - it's non-lyrical; just dead facts; a collage that depends on juxtaposition; an essay. I like Hart's poem and the style. I use factoids - I like finding out that Defoe, when he was pilloried for criticising the authorities in 1703, was pelted by the public with flowers, or that Hitler and Wittgenstein went to the same school. In the poetry game, facts play a role they don't play in prose. They're perhaps further from poetic truth than beautiful imagery is, but they're useful all the same.
\begin{itemize}
\item    "beauty is truth, truth beauty, - that is all ye know on earth, and all ye need to know", Keats, "Ode On A Grecian Urn"
\item    "Art arises out of our desire for both beauty and truth and our knowledge that they are not identical", Auden, "The Dyer's Hand and Other Essays",
\item    "Every poem starts out as either true or beautiful. Then you try to make the true ones seem beautiful and the beautiful ones true", Larkin
\end{itemize}

\newpage\subsection{Reality and Symbols}
Symbolism is seen as a reaction to Realism, attempting to capture more absolute truths which can only be accessed by indirect methods. Here I'd like to look at how readers and poet position themselves along this Realism-Symbolism dimension.

\subsubsection*{How do Symbols acquire meaning?}

Symbolic thought is of course not restricted to poetry. When someone says "Let x=2", x is a symbol. When someone says "Let x=Pain and y=Tears. Then y/x is a measure of cry-babyism" they are beginning to think symbolically. Here the symbols are introduced explicitly and what they "stand for" is clearly presented. Language uses words as symbols ("candle" to represent a candle for example), but in poetry we're used to reading more into words. In the charged atmosphere of reading a poem, a lit candle represents the fragility of life. These conventional meanings are common in non-poetry, and occasional readers of poetry are good at detecting them (indeed, they might expect a poem to exploit them) but they have trouble with other symbols. Some poems develop their own definition of a symbol - Coleridge's Albatross, for example, or Larkin's Toad - and other poet can allude to these symbols.

\subsubsection*{When do Readers read symbolically?}

Readers are more likely to read symbolically if Realism cues are reduced. A Forest is a place of testing, of challenge, but if the forest is named and described, the urge to symbolise is attenuated. A poem that's bland or obscure or that repeats a word may tempt readers to look beyond literal meanings. It's usually fairly clear to the reader when a poem's trying to develop its own symbol definition, but the meaning of the symbol may not be clear. Blake's mass of symbols may be bewildering, but they're not mistaken for Realism.

\subsubsection*{Substitution or Augmentation?}

Once readers have seen a symbolic interpretation they might need to decide how ornamental the symbolism is. When a symbol is introduced by a simile (e.g. "her eyes were like stars") the first item is real; the rest is "just words". In some more extreme types of symbolism, the real part is suppressed entirely. Elsewhen it's mentioned then discarded, the symbolic meaning taking over.

In the charged atmosphere of a poem, words like "star" and "ladder" are going to be given symbolic meanings even if the poet meant them literally. This readerly urge towards symbolification is exploitable by poets. One of the more common ploys when writing poetry (I use it myself) is to use a title with more than one meaning ("Going Down", say, or "Still Life"), then start the poem with an anecdote where a literal interpretation of the title is followed (going down in a lift, for example, or looking round an art gallery). Then in a punchline the more symbolic interpretations are invoked (depression, etc). The punchline doesn't deny the reality of the anecdote, it adds depth. Similarly in Pound's
\begin{verbatim}
   The apparition of these faces in the crowd;
   Petals on a wet, black bough
\end{verbatim}
both images are real enough; the implied symbolism not destroying them. Pound wrote "Go in fear of Abstractions", and "I believe that the proper and perfect symbol is the natural object, that if a man use 'symbols' he must so use them that their symbolic function does not obtrude; so that a sense, and the poetic quality of the passage, is not lost to those who do not understand the symbol as such, to whom, for instance, a hawk is a hawk". Following this principle, one's allowed to use an hour-glass instead of "Time". One can smash the hour-glass instead of "annihilating Time", or even become "Time's assassin", but I suspect one can't end a poem (like Yves Bonnefoy does) with "The trampled snow is the only rose".

Gertrude Stein wrote that "A rose is a rose is a rose". Freud said that sometimes a cigar is just a cigar. William Carlos-William's red wheelbarrow doesn't dissolve into symbol. Neither is it eclipsed. But the reality of the image can be compromised, the poet using reality like Wittgenstein's ladder - to be thrown away after use. A Raven who symbolises Death might start talking. When Wallace Stevens begins "Anecdote of the Jar" with "I placed a jar in Tennessee/ And round it was, upon a hill" we soon realise it's no ordinary jar, it's a "Jar". Another poem of his begins "The lion roars at the enraging desert,/ Reddens the sand with his red-colored noise." where Realism doesn't last long at all.

Once symbolic interpretation isn't ballasted by reality, anything - even even vowels and perfumes - can become pregnant with potential symbolic value, everyday objects becoming invested with the power of Symbol. At that stage, I suspect the common reader gives up, or at least, treats the text as a Rorschach blotch.

\subsubsection*{Decisions, decisions}

Where does all this leave the reader? Inexperienced poetry readers might have trouble with poems that develop their own symbolism. Some readers want to know what's Real and what's Imaginary and don't like boundaries being blurred. Some readers are unable to leave Reality for long, especially if the poem begins as a Realist piece - when the symbolism takes over, these readers don't release their grip on Reality and are likely to describe the text as confused. Coping with something being simultaneously Real and Symbolic shouldn't be difficult (after all, millions of Christians cope with the Trinity) but in practise it requires more flexibility and re-contextualising than many readers can afford. Pound might have a point - poets who follow his dictum might risk readers only seeing the literal meaning, but at least they'll see something.

The final stage in applied maths calculations is to de-symbolize. Symbols may be compact and easy to manipulate. In the midst of calculation it may be impossible to identify what aspect of the Real World is represented by a particular expression, but the conclusion should feed back into reality, even if one needs to extend one's reality to do so. Anti-matter exists, as you can see if you hold your breath long enough under the surface of reality. So do Emperors of Ice-cream.

\newpage\subsection{Adapting short texts for the market}
When texts are short, the poetry and prose categories can merge. So why bother trying to categorize? Once people other than the author are involved, there are several reasons - e.g. 

\begin{itemize}
\item Editors of anthologies and specialist magazines need to draw a line somewhere
\item Writers submitting to certain magazines need to specify whether their submission is poetry or prose. What type of work does the author want the submission to be printed amongst? Magazines often demarcate, though Arts Council "New Writing" anthologies sometimes had a "Texts" section for unclassified works, and "The New Yorker" used to have "Casuals" and "Shouts and Murmers".
\item Readers may benefit from knowing which reading strategies to initially adopt, and which expectations to develop.
\end{itemize}

\subsubsection*{Categories}
The classifications are more fine-grained than just "Prose" and "Poetry". There are Haibun (combining prose and haiku), anecdotes, vignettes, contes, short-shorts, microfiction, Flash, ketai fiction, Twitter lit, prose poems, found poems, etc. The existence of line-breaks usually suffices as a marker of poetry, though some free verse, shorn of its line-breaks, might easily fit into these prose categories. Some forms are defined by word-count. For some others, definitions abound. Here are a few -
\begin{itemize}
\item \textit{Flash} - The Bridport Prize's web site suggests that Flash "\textit{contains the classic story elements: protagonist, conflict, obstacles or complications and resolution. However unlike the case with a traditional short story, the word length often forces some of these elements to remain unwritten: hinted at or implied in the written storyline}". They impose a 250-word limit.
\item \textit{Prose poetry} - In "This Line Is Not For Turning", a prose poetry anthology she edited, Jane Monson provides a description of the prose-poem "at its most disciplined" - "\textit{no more than a page, preferably half of one, focussed, dense, justified, with an intuitive grasp of a good story and narrative, a keen eye for unusual and surprising detail  and images relative to that story, and a sharp ear for delivering elegant, witty, clear and subtly surreal pieces of conversation and brief occurrences, incidents or happenings}". She goes on to write that her anthology "\textit{focuses and captures a particular style; which shares in tone, pithiness and brevity the best traits of 'flash fiction', 'micro fiction', 'sudden prose', and the 'short short story' rather than the strengths and weaknesses of 'free verse', 'blank verse' and 'poetic prose'}" 
\item \textit{Prose poetry} - In Summer 2012's "Poetry Review", Carrie Etter wrote "\textit{While some poets and critics insist that we must resist defining prose poetry for it to retain its subversive genre-blurring character, I find some basic distinctions crucial for its appreciation ... a prose poem develops without 'going' anywhere}"
\end{itemize}

Magazines like ``Sudden Prose'' ("Prose Poetry and Short-Short Stories) and ``Double Room'' (seeks to "explore the intersection of poetry and fiction") seem to accept the overlap. The 2 denominations occupy similar terrains though they have different histories and look in different directions. Calling a piece a prose poem is still making a statement - 
\begin{itemize}
\item "\textit{Prose Poetry is some of the funniest—and strangest—writing you’ll find anywhere. It lends itself to the comic, and the absurd. Maybe humor is easier to convey in a sentence than in a line break. … Flash Fiction is something else, as it’s about character (and change), and it’s therefore more difficult to pull off in such a short space}" -  \url{http://thebarking.com/2010/11/how-to-tell-prose-poetry-apart-from-flash-fiction/} Brett at Bark.
\item "\textit{Flash fiction focuses on story (whether that be character or plot or place or time). Prose poetry focuses on image and/or emotion}" - \url{http://bhamreview.blogspot.co.uk/2010/11/prose-poetry-or-flash-fiction.html} Chris (Bellingham Review)
\item "\textit{In spite of the prose poem’s history of breaking rules and redefining itself ...}" - \url{http://www.flashfictiononline.com/c20100402-small-rebellions-prose-poems-bruce-holland-rogers.html} Bruce Holland Rogers (flash fiction online). 
\end{itemize}

\subsubsection*{Market trends}
I think the current short-text literary arena is currently dominated by 3 overlapping terms
\begin{itemize}
\item \textit{Poetry} - the dominant term; so much so that in the late 1900s a short text had to be made into a poem to have much of a chance of publication. In the age of relaxed free verse, inserting regular line-breaks often sufficed to create a "poem". A text labelled as such carries some of the weight traditionally associated with the term, and is most likely to be read beside other poetry. It needn't have plot or character, nor need it be written in sentences, though it frequently has all of these properties.
\item \textit{Flash Fiction} - a fairly recent and popular term for a cluster of genres that have been around for a long time. Derived from the short story, it's expected to have plot and character (though the proportions may vary), and is likely to be read beside other (perhaps much longer) fiction. Venues now exist for such work - in dedicated magazines, but also magazines in general are more likely to accept short texts nowadays. Specialist outlets impose word-count limits (from 250 to 1000 words).
\item \textit{Prose poetry} - Initially a rebellion against the rhyme/meter of Formalism, then later a challenge to the one remaining obvious feature of poetry - the line-break. Shorn of its rebel image, it retains its feel of being different - though examples appear in many collections and magazines, there are rarely more than 2 or 3 examples per publication. It's likely to appear beside other poetry.  Sometimes the only prose-like feature it possesses is the layout, but more often it's in sentences, and can (or even should, according to some practitioners) have narrative impetus.
\end{itemize}
The popularity and increasing acceptance of Flash (and to a lesser extent of prose poetry) should mean that fewer texts have gratuitous line-breaks nowadays, but understandably, progress is slow. Re-classification of texts previously published as poetry would help change the climate. When creating his poetry anthology, Yeats used a fragment of Pater's prose. Even pieces as long as Carolyn Forche's "The Colonel" have appeared in both poetry and Flash anthologies. In Monson's prose poetry anthology, someone contributed part of their novel. I think the distinctions between micro-fiction and prose-poetry are rather in the eye of the beholder, and the prose-poetry/free-verse distinction can be merely the result of typing habits or previous adaptions for markets.


\subsubsection*{Adaptions}
Given these fuzzy theoretic definitions and the fluidity of the market, it's tempting for writers to add/remove line-breaks, add/remove punchlines, or add/remove connections in order to make a text more appealing to particular outlets. For some styles (those using surrealism, perhaps) I don't think any artistic integrity is lost by doing this. Given the variation in word-count limits it's also worth having more than one version of stories. When  texts are adapted, more genre decision might be necessary. When line-breaks are removed from poetry, one of two effects are likely - 
\begin{itemize}
\item More narrative might be added (i.e. more prose features added) 
\item The text might appear rather flat, so to compensate the content may become more surreal/imagistic, less linear (i.e. more poetry features added).
\end{itemize}

When shortening prose, several things can happen. The result might be
\begin{itemize}
 \item A sketch - same proportions as a story
 \item A slice (just the sounds, maybe, or a moment in time without back-story)
 \item A fable (a genre that allows omissions)
 \item Selected extracts - an interesting set-up followed quickly by a punchline rather than by character development.
 \item A prose form - a shopping list, an application form, a questionnaire, etc.
\item More obscurity (on the grounds that readers can re-read)
\item More intensity or extremes
\end{itemize}
I think that a piece in a form is often printed in poetry sections of magazines even if its content isn't poetic - "forms" and "short texts" both tend to be associated with poetry. So shortened prose can end up in a poetry venue.

\subsubsection*{Series}
The distinction between short and long texts is challenged by series. In \url{http://georgeszirtes.blogspot.co.uk/2013/03/time-lines-few-lines-and-fewer.html} ``Time Lines: few lines and fewer'' George Szirtes looks at Twitter, pointing out that 
Jennifer Egan's novella, ``Black Box'', is told all in tweets. Similarly, microfictions can be strung into a sequence.



\newpage\subsection{Poetry about Science in the UK }
\begin{itemize}
\item "The purpose of Art is to impart the sensation of things as they are perceived, and not as they are known" - Shklovsky
\item     "Art works because perception goes beyond the evidence and it ignores much counter-evidence" - Richard Gregory
\item     "Art evokes while science explains" - quoted by Richard Gregory
 \item    "Science is a system of statements based on direct experience and controlled by experimental verification." - Carnap
 \item    "Science, far from destroying the beauty and romance of the world as seen by artists, musicians and writers, enhances it by revealing the underlying reasons and purposes" - McConnell
\item     "Science is for those who learn; poetry for those who know"
\end{itemize}
Newton received a mixed reception from poets. As the scientific revolution that he spurred took hold, the Romantics protested against its mechanistic abstractions. The protest was, in hindsight, valid though it wasn't until the 20th century that mainstream scientists realised the truth of the accusations. The protest was also ill-informed - the later Tennyson was much better read in the sciences and subsequent generations of poets have always included a few professional scientists. Apart from the debate sparked off by C.P. Snow's "Two Cultures" and sporadic fascination with polymaths like da Vinci, 20th century pure science and art continue in splendid isolation, though "the rise of science has had something to do with displacing [poetry] as a publicly important vehicle for those truths that people accept as being centrally important.". The challenge of Positivism caused the Arts to retreat from making great claims about `truth'. Poetry, for example, was cut down to size by I.A. Richards and isolated/protected from other disciplines by its claim to be a unique mode of discourse. Meanwhile literary theory (as opposed to literary criticism) prides itself on being ever more scientific. Technology, with its invitation to control and change, has entered into all aspects of our everyday lives, making artists more aware of contemporary science than scientists are of modern art. This new openness from artists presents an opportunity to build greater understanding between the `cultures'. Connections and analogies have been advanced.


Our world has been transformed by technology so one would expect
cars, TVs and test-tube babies to appear in poetry. And as the
percentage of scientists in the population grows, the lives and
preoccupations of their profession feature more often in poems.

But the world we \textit{can't} see directly has changed too. Relativity
and (even more so) Quantum Physics have shown us that we live in a 
world beyond what common sense can cope with. And Mathematicians,
dealing with topics like (and here I quote at random from the first maths 
journal at hand)
'Examples of tunnel number one knots which have the property 1+1=3' live in a stranger world still.

One might have thought that adventurous poets would have rushed into
these newly opened territories, but it seems to me that poets over the last 
few centuries have withdrawn
from trying to tackle the big questions about the Nature of the Universe. 
They tend not to deal with the moral complications that new technology
engenders, and it's rare to see a Blakean anti-materialist piece attacking
scientists.

In England the poetry of Prynne, Dorothy Lehane, etc sometimes includes a liberal sprinkling of science vocabulary. The risk with using science terms is that the poem is going to come over differently depending on how much science the reader knows. This risk applies to many types of allusions of course, but in the science case the reading communities are easier to define, and the material may more easily become out-dated. 

Using science words is easy. Less frequently, poets deal with science (and maths) concepts. We are used to philosophical or religious poetry, poetry that presents an argument. Quantum theory and Relativity are common themes for those wanting to express scientific ideas poetically. More mainstream than Prynne and Lehane are Heidi Williamson and Lavinia Greenlaw who use science or (more often) scientists as subject matter. I think poets' attempts to write science poems fall into these 
main categories

\begin{itemize}
\item  \textit{Awe and Wonder} - like children given a microscope for
Christmas. Science is the new Exotic to some (mysterious trinkets from another land), to others it's the new Theology - deep truths masked by code. With its cornucopia of new ("X-ray") and re-used ("charm") vocabulary  it's tempting to raid its word kitty. If the result sounds clever, the reader might think the poet's clever too. Few readers are going to be in a position to challenge from a scientific position, and in any case, what would it prove? A poem's not a thesis. However, I suspect that if poets appropriated the vocabulary of Art in similarly cavalier fashion, they wouldn't get away with it.
\item  \textit{Effects on relationships} - e-mail and the mobile phone
\item  \textit{Science Fiction Poetry} - there's a book of StarTrek
poetry
\item  \textit{Biographical pieces about famous scientists}
\end{itemize}

These are topics that can easily be tackled by people who know little
about science. And yet, more poetry is written by scientists nowadays than in 
any previous era. So why don't we have more poetry from the frontiers of science? On the international front, the internationally acclaimed
Czech poet Miroslav Holub was also a serious immunologist. It is perhaps
significant that he had doubts about twinning poetry and science. In  "The Dimension of the Present Moment" (Faber and Faber, 1990) he wrote
\begin{itemize}
\item "At first glimpse one might suspect that literature would be closer to the sciences than other art forms, because sciences also use words and depend on syntax for expressing their findings and formulating ideas. ... [but] There is no common language and there is no common network of relations and references. Actually, modern painting has in some ways come closer to the new scientific notions and paradigms, precisely because a painter's vocabulary, colours, shapes and dimensions are not congruent to the scientist's vocabulary." (p.130) 
\item "In the use of words, poetry is the reverse of the sciences. Sciences bar all secondary factors associated with writing or speaking; ... poetry tries for as many possibilities as it can." (p.132)
\end{itemize}

In New Scientist (24 July 1999) Graham Farmelo (Science Museum, London)
wrote "Be sceptical of any science-art initiative and you are liable to find 
yourself marked down as a narrow-minded reactionary. If a new work of art is 
based on a theme related to science, most critics will give it an easy ride...
It seems that this flavour of political correctness encourages intellectual 
laziness, allowing shallow and sentimental nonsense about the relationship
to pass for serious thought". So perhaps we should be wary of some recent
initiatives.

In the UK ex physics/maths professors appear in small magazines, and
Mario Petrucci, who's active in many areas of UK poetry, has a physics
Ph.D. London's Science Museum sometimes has a poet-in-residence. The
best known holder of that post is Lavinia Greenlaw. In 2000, she was awarded 
a three-year fellowship by the National Endowment for Science, Technology 
and the Arts. She said she'd use the 67,000 pounds to "\textit{undertake formal study}" in science, and journey "\textit{to places with extreme
   perspectives - precarious and changing landscapes, or those which
   experience the natural phenomena of eclipses and equinoxes}". 
For television, she has written a sequence of poems about the 
meaning of numbers for an Equinox documentary. Her WWW site is
\url{http://www.laviniagreenlaw.co.uk/}
Also worth a look is \url{http://www.poetryandscience.co.uk/} LUPAS (Liverpool University Centre for Poetry and Science).

Heidi Williamson was poet-in-residence for the London Science Museum’s Dana Centre in 2008 and 2009. On the back of her book it says that her "\textit{fascination with science leads her to explore less usual territories for poetry, including mathematics, chemistry, and computer programming, as well as space travel, electricity, and evolution}". I think that more poets who write about science go into it nowadays with their eyes more pragmatically open than that -

\begin{itemize}
\item "\textit{The main difficulty with 'Night Photograph' has been the “poetry about science” tag. I grew up in a family of scientists and have long been fascinated by time and space, so this is a natural source of metaphor for me. I only became conscious of how much science there is in the book when it was pointed out. Since then, I have resisted science like hell. It is mostly too seductive, incomprehensible and exciting to be anything other than borrowed}" -  Lavinia Greenlaw, interview in Thumbscrew (1997)

\item "\textit{I married an astronomer!  ... I think initially I was trying to write metaphors for the science, based on human experience but that wasn’t working out so well, the science was present, but the poetry seemed dry. I didn’t think I was achieving anything more than representing the original idea, theory, or astrophotography I was looking at, or the paper I was researching. So I tried to do more than represent the original by using the science as a launch pad but moving away from it, by keeping a dialogue with human concerns at the same time}" - Dorothy Lehane, interview in Annex (2013)

\end{itemize}

Andrew Duncan reviewing Lehane's "Ephemeris" in Litter magazine writes that "\textit{Lehane’s project has to do with combining poetry and science. The two are intertwined in a very specific way here: objective knowledge separates projections of feelings and wishes from the information provided by the eyes, but here the idea is to interfuse them. Her poems are intensely personal and highly coded: everything profound loves a mask}". 

He goes on to suggest that "\textit{The poetry-science project is likely to draw a great deal of attention in the next twenty years or so. It is quite hard to define what the purpose is; I think the core is the sense of opportunity, that there is a wilderness here, and that if you buy creative people time they will wander around that wilderness and bring back things never before seen. Part of the impetus is the wish of museum staff to make their holdings presented anew in visible or audible form.}"




If you want to read science poems, 2 UK anthologies of note are
\begin{itemize}
\item  "Poems of Science", eds J. Heath-Stubbs and P. Salman, Penguin, 1984.

\item  "A quark for Mister Mark", eds Maurice Roirdan, Jon Turney: Faber, 2000.
\end{itemize}


To finish, here's a list of UK poets with science-related degrees
\begin{itemize}
\item Michael Bartholomew-Biggs (was a computing professor)
\item Tania Hershman (M.Sc, M.Phil)  
\item Peter Howard (science degree)
\item Valerie Laws (Maths/Theoretical physics)
\item Kona MacPhee (computing degree)
\item David Morley (post-degree)
\item Stephen Payne (Professor of Human-Centric Systems)
\item Mario Petrucci (Ph.D)
\item Colin Will ('majored' in earth sciences and chemistry; a Ph.D in
information science)
\item Rachel McCarthy (Chemistry and Physics)
\item Jemma Borg (Ph.D in evolutionary genetics)
\end{itemize}


\newpage
\section{Psychology}

Here I delve deeper into the psychology of reading, and wonder what types of people might be drawn to poetry.


\subsection{Poetry, voice, and discourse analysis}
''Language and Creativity: the art of common talk'' by Ronald Carter (Routledge, 2004) analyses fragments of dialogue from various contexts to show how conversationalists are creative at a linguistic level - one example provided is
\begin{verbatim}
A: Yes, he must have a bob or two.
B: Whatever he does he makes money out of it, just like that.
C: Bob's your uncle.
B: He's quite a lot of money erm tied up in property and things. He's got a 
finger in all kinds of pies and houses and stuff
\end{verbatim} (p.2)

Carter points out that ``the most frequent forms of linguistic creativity include: speaker displacement of fixedness, particularly of idioms and formulaic phrases; metaphor extension; morphological inventiveness; verbal play, punning and parody through overlapping forms and meanings; 'echoing' by repetition, including echoing by means of allusion and phonological echoes'' (p.109).

This creativity performs many functions, amongst them ``to give pleasure, to establish both harmony and convergence as well as disruption and critique, to express identities and to evoke alternative fictional worlds which are recreational and which recreate the familiar world in new ways'' (p.82). It can be performative, competitive, figurative, space-filling, or for fun. Situations which are less concerned with information transfer (e.g. banter) give more scope for creativity. Often on ``the surface and to the outsider (though not to the participants) there is much divergence, disconnection and incoherence. Beneath the surface there is, however, much convergence and coherence marked in a distinctive range of pattern-reinforcing linguistic features, especially repetition'' (p.101), and that ``how what is said is as significant, if not more so, than what is talked about'' (p.105).


This ``Common talk has continuities with and exists along clines [aka gradients] with forms that are valued by societies as art. The values which are attached to the art of common talk will vary according to context, time and place'' (p.210). The author suggests that ``Speakers also often wish to give a more affective contour to what they or others are saying. It is hypothesised here that there are three essential expressive options open to them: the expression of intimacy, the expression of intensity and the expression of evaluation'' (p.117). All of these features tend to be increased in informal situations. Shifts along these dimensions are significant and often signposted - e.g. ``proverbs appear at a discourse boundary, as if functioning to close down a conversation by summarising an attitude or by indicating a particular stance towards what has been said or to allow a smooth transition from one topic to another'' (p.134).


I think we're alert (often subconsciously) to these nuances of register change - to how they're signalled and what their purpose is. At an appointment between a GP and a patient for example, a patient will react to the doctor's invitation to informality, seeing it perhaps as an indication that there's nothing seriously wrong. The GP on the other hand might be trying to extract a less inhibited description of the symptoms from the patient. Friends in discussion are also sensitive to the significance of these switches, or at least they sense the undercurrents that these shifts and switches create.


At a 2014 poetry workshop run with Emily Berry,  Jack Underwood said he thought that poetry nowadays was more about voice and less about comparisons. He suggested that participants could try suppressing explicit narrative, using juxtaposition (of registers, tone, etc) to create distances for the reader to travel. The sample poems were mostly by Americans. Here are extracts from 2 of them

\begin{itemize}
\item 
\begin{verbatim}
There are about 35,000 elk. 
But I should be studying for my exam. 
I wonder if Dean will celebrate with me tonight, 
assuming I pass. Finnish Literature
really came alive in the 1860s. 
Here, in Cambridge Massachusetts, 
no one cares that I am a Finn.
They've never even heard of Frans Eeemil Sillanaa, 
winner of the 1939 Nobel Prize in Literature.
As a Finn, this infuriates me.
\end{verbatim}

(``I Am a Finn'' by James Tate)
\item 
\begin{verbatim} 
It's a travesty of hand-stitching, a decapitation.
Whose cotton limb? It dangles from my thumb
and forefinger. The universe slackens in its shadow
\end{verbatim}
``Ruminations on 25mm of Cotton'' by  Heather Phillipson)
\end{itemize}

Bakhtin's definition of 'hybridization: the mixture of two social languages within the confines of one utterance ... poetry marked by heteroglossia, which perhaps what this is. There are rapid changes of register (changes of intimacy, intensity and evaluation are evident). Both are presented as if from a single persona, but in general the distinction between this and polyphony isn't clear - polyphony can be flattened into monologue. In Eliot's notes for ``The Wasteland'' he writes ``Tiresias, although a mere spectator and not indeed a 'character,' is yet the most important personage in the poem, uniting all the rest. Just as the one-eyed merchant, seller of currants, melts into the Phoenician Sailor, and the latter is not wholly distinct from Ferdinand Prince of Naples, so all the women are one woman, and the two sexes meet in Tiresias. What Tiresias sees, in fact, is the substance of the poem''. Sections of the Wasteland read as dialogue even if they're not, and vice versa - ''Do/ You know nothing? Do you see nothing? Do you remember/ Nothing?''/ I remember/  
Those are pearls that were his eyes./ ``Are you alive, or not? Is there nothing in your head?''/ But/  
O O O O that Shakesperian Rag—'' or
\begin{verbatim} 
Bin gar keine Russin, stamm’ aus Litauen, echt deutsch.
And when we were children, staying at the archduke’s, 
My cousin’s, he took me out on a sled,  
And I was frightened. He said, Marie,
Marie, hold on tight. And down we went.  
In the mountains, there you feel free.  
I read, much of the night, and go south in the winter.  

What are the roots that clutch, what branches grow
Out of this stony rubbish? 
\end{verbatim}

There's little simile and narrative in these extracts. But need the interplay of registers be at the expense of narrative and form? In the past I think there are many type of poetry capable of exhibiting contrasting registers - the cubist idea of fusing different viewpoints and the use of collage are early 20th century discoveries, but the use of a Fool, a mad person or a cacophonous crowd to depict polyphony go back much further. Many have retained some kind of plot. And what about sound?  Poets of many types have suggested that the form and sonics of their poems create a constantly changing, parallel effect to the meaning, that an even-handed dialogue is possible between sound and ``meaning''. 
\begin{itemize} 
\item ''The chief use of the 'meaning' of a poem, in the ordinary sense, may be ... to satisfy one habit of the reader, to keep his mind diverted and quiet, while the poem does its work upon him.'', T.S. Eliot
\item ''The sounds, acting together with the measure, are a kind of extended onomatopoeia - i.e., they imitate, not the sounds of an experience ... but the feeling of an experience, its emotional tone, its texture'', Denise Levertov
\item ''sound enacts meaning as much as designates something meant'', Charles Bernstein
\item ''sound in its due place is as much true as knowledge (and all that mere claptrap about information and learning). Rhyme is the public truth of language, sound paced out in the shared place'', J.H. Prynne

\item ''Bunting would say that you should hear the 'meaning' of the poetry purely in the sound ... Word patterns which may at first appear dense and complicated on the page become articulated and clarified, resonating across the poems' structure. The subtleties and echoes of language which hold a poem together are revealed by the process of sounding it'', Richard Caddel

\item ''The ear is satisfied when the metre is balanced and the rhyme struck, but the sentence is incomplete and the mind seeks its satisfaction in resolution of the sense... By the counterpoint - a kind of suspense - created between the arrangement of sounds and the construing of sense, a pace builds and a drama develops'', Michael Schmidt

\item ''The classic prejudice persists, however, that sound is secondary to meaning. The prejudice has been challenged by John Hollander, who, seeking to show the relation between sound and poetic meaning, discovers that sound pattern can play the role of an allegory or metaphor of the poem's content the role of sound in language becomes clear only when expression becomes artistic, so that language exceeds its purely representational function'', Anca Rosu
\end{itemize}



Because discourse-based poems emulate speech, they tend not to use sound effects (regular ones, at least). These new-style poems exploit readers' conversational skills, using their reactions as the pivots that articulate the movement within a poem in preference to using their ear for music. The tasks performed in the past by sound can be simulated by register changes -
\begin{itemize}
\item Particular sounds were thought to be invested with particular meaning (within the context of a particular poem, at least). For example an ``oo'' sound might signify sadness. In discourse-based poems there's a corresponding way to trigger emotion - for instance the switch to an impersonal standard phrase might denote cold-shouldering or distancing.
\item An earlier phrase can be alluded to by use of rhyme. In discourse-based poems, sudden formality might remind the reader of the previous formality, or use can be made of stand-up comedians' callbacks.

\end{itemize}


As Bryan Walpert points out in "Resistance to science in contemporary American poetry'', ``The 'meaning' ... lies not in an expression of the individual author or speaker but in the collision of languages or discourses'' (p.128), going on to write that ``it is in language that we construct what appear to us to be unified central selves and so it is language that poetry must scrutinize'' (p.182).

Older poems might be analysed by counting beats and identifying constellations of sounds. W.G. Shepherd for example, quoting ``Bellflowers, seldom seen now, stellar, trim. by Peter Dale, remarked ``Note the triple statement of the el(l) sound counterpointed against the duple m; the narrowing of el(l)'s vowel to ee and i - boldly interrupted by recapitulation of ow; and the modulation of s through st to t''. I've seen descriptions of how discourse-based poems work using terms like ``shifting planes'', ``tectonic plates colliding'', ``slippery'', ``tonal juxtapositions'', ``shimmering surfaces'', etc. Narrative is compared to melody, and discourse-based poems to atonality. I think that one can be less impressionistic than this. The poems might be studied by 
\begin{itemize}
\item re-casting as a multi-voice poems.
\item identifying the direction and magnitude of each jump.
\end{itemize}

\newpage
\subsection{Linguistic psychopathology of poets and strangers}

Poets are a clan, a secret society. One poet reading another will as likely trust them. But to prose readers, poets are strangers. An atmosphere of trust needs to be fostered by the poet otherwise the reader will just shut the door in the poet's face. Readers need to be convinced that the effort of reading will be repaid. A collection of poetry offers the poet an opportunity to develop a trusting relationship. The reader might come to assume that some poems share a persona, so assumptions can be made about the persona's gender, era, location, sexual orientation, etc - factors that complicate interpretation of poems when they're read in isolation. As readers continue reading, they might find shared understandings and shed suspicions of there being any Emperor's New Clothes. They might feel that they're getting to know the poet. Some poets (Sharon Olds maybe, or some Confessionalists) encourage or even require such an engagement by the reader, a commitment for richer and for poorer. 


But suppose the poet seems untrustworthy? In real life when introduced to new people we sometimes encounter similar situations, meeting people who seem to be bad judges of character, whose opinions fly against evidence, who contradict themselves, who talk too much, who exaggerate, who are pretentious or incoherent. None of these traits of course is sufficient to sever all contact with the person - they may be interesting and likeable nonetheless. Even if they're not, you may have to interact with them - they may be new colleagues or in-laws; they may be sitting next to you on a long flight. You make allowances, even if the people are a little strange. Indeed, strange people are often rather interesting to people who read poetry, especially if their use of language is strange too.



Is all non-standard language poetic? Of course not. Indeed it's barely a meaningful question. And yet, within each domain of language use, a poetic element can be brought out. Perhaps it's easier to see the poetry when it's in an unusual context. Away from the familiar poetry patterns, rhymes, metaphors, and emotions, there's potential in recipes, adverts, specialist jargon, shopping lists and rants of the mad. This relocation (often removing the purpose from usually purposeful language) draws attention to the strangeness of language.


 Any breaking of the word=world equivalence, any doubts raised on the transparency of language, can be considered poetic. The disruption may be minimal. For example, in one of her poems Jo Bell writes about ``disappearing her toes in the sand''. Making an intransitive verb into a transitive one in this way doesn't make the text harder to understand, but does make the text more likely to be considered a poem.


The most poetic non-standard language comes perhaps from people with mental disorders. In his 1911 book, Bleuler (who coined the term schitzophrenia) quoted this much-quoted passage that exhibits some non-standard traits - ``I always liked geography. My last teacher in that subject was Professor August A. He was a man with black eyes. I also like black eyes. There are also blue eyes and grey eyes and other sorts, too. I have heard it said that snakes have green eyes. All people have eyes.''


When people are in a manic phase their language is affected. There's likely to be
\begin{itemize}
\item an increased use of pronouns and verbs at the expense of adjectives and prepositions. Speech is likely to be I-oriented
\item more circumstantial, anecdotal or random links between phrases
\item more discursive and verbose discourse. More loose ends.
\end{itemize}

These features can make a manic person more tedious to be with. The resulting isolation can worsen their problems. With high-functioning manics these features are no less tedious, but at least there are more compensations. Other non-standard personality features that affect language may also be present -
\begin{itemize}
\item Quirks and idiosyncrasies
\item Disinhibition
\end{itemize}


This disinhibition (which makes it possible to say things that one wouldn't normally), coupled with the greater variety of links between ideas, aids creativity, so it's not surprising that there's a similarity between manic language and some styles of poetry. Those with borderline symptoms may be encouraged to be conventional, or (if writers) try to write a normal piece with a mad person as the main character. Alternatively, rather than change the writer to fit society, new surroundings can be found to suit the writer. The poetry world is one such world, confessionalism being a tempting style. The world of poetry offers opportunities to legitimize behaviour that would otherwise be considered anti-social or rude. If a person with mental problems adopts an existing role, other people to know how to interact with them, thus helping to socialise the troubled person. It's well known that poets have poetic license, so people expect non-PC or unconventional behaviour from people who identify as poets.


Because manics are more likely to be isolated or drugged nowadays, for many people it's something of a novelty to listen to a manic. Their words can superficially sound creative. However, you don't need to be with patients long to know that the illusion soon wears off.


There are of course huge differences between a reader-manicpoet relationship engendered when reading a collection, and the kind of relationship one might develop with (say) a manic stranger on a long plane flight -
\begin{itemize}
\item In real life, stuck on a plane, one politely makes allowances. Quirks and idiosyncrasies (e.g. beginning each contribution with ``Er ...'' or ``Well ...'', for example) are filtered out in a way that we're not used to doing when we read (Christopher Smart's ‘My Cat Jeoffrey‘ being an exception)
\item One tends to become tolerant of exaggeration during casual conversation. One might be expecting Truth and Illumination from a poet.
\item Though it's tempting in a book to assume that the persona is the person, that's far from always true.

\item On a plane, one can't easily walk away. Once one's bought a book one tries to justify the time and money spent, especially if the book's been recommended, but one can always stop reading
\item The stranger may not realise that they're coming over as egotistical and demanding. If the poet doesn't realise, then presumably the publisher should. Gratuitously unconventional language and subject matter means that the listener/reader will need to work harder. There's an added elitism in the poem situation - only  poetry readers with editing skills will be able to benefit comfortably from the text.
 
\end{itemize}


When I was in my 20s a friend I'd known since school began acting strangely. For a while he was institutionalised, then he went into shared accommodation - ``care in the community''. I often visited. I got to know his new friends. Having had a rather sheltered, scientific upbringing I was interested and stimulated by the company. The subject matter was new to me (some of it eventually got into a prizewinning story of mine) but it was more the uninhibited mix of subject matter (and of reality/fantasy, public/private) that struck me. I transcribed the odd monologue. A mutual friend made video recordings that (heavily sampled) have become minor cult hits on YouTube. At that time I hadn't met any people from the creative arts, and was impressed by the imagination exhibited by my friend's friends, free from society's pressure to conform. That said, I grew increasingly bored of their rambling monologues and the repetition. I see the same free association and lack of inhibition in some poetry. I wonder sometimes if people enjoy it for the same reasons I liked those monologues all those years ago - for the novelty, the unacademic unstuffiness, the escapism. 


I have bursts of writing between quiet periods during which the excesses of my writing phases are pruned. The writing's not manic, but I'm aware that the re-writing may remove the spontaneity and freedom of association that made the original interesting. So I can sympathize with manic depressive writers who don't want to change a word of their first draft, even those writers who during their depressive phases feel like destroying their work. But on the whole I'm wary of literature that has too many of the traits of manic writing. Poetry and social discourse are very different contexts with different expectations and norms of behaviour. But it's this very difference that readers should remind themselves of. Readers don't have the duties and responsibilities that carers bear. I suggest that they  
\begin{itemize}
\item be cautious about the impact of novelty, disorganisation and disinhibition - text that's sexually explicit or non-PC might appear striking and original in a poetry context, but in a wider context it might be common
\item consider what's lacking as well as what's in surplus
\item wonder why they're being made to work harder than is necessary, and wonder whether it's fair
\item remember that they can cut their losses and stop reading. Nobody's present whose feelings will be hurt
\end{itemize}

\newpage
\subsection{Psychology, Psychiatry and Writers Groups}
%\begin{figure}[htbp]
\begin{wrapfigure}{r}{0.2\textwidth}
\centering
\includegraphics{head.jpg}
\end{wrapfigure}
%\end{figure}
As Professor Philip Thomas (Lancaster University) says ``There have always been people in societies and cultures who have different experiences of reality compared with the majority, and there's always been an overlap between people who have those gifts, or insights, and people who are identified as suffering from mental illnesses.'' ... ``it's the strangeness of people's experience, and what they try to communicate about it, that's dangerous, threatening, anxiety-provoking to those of us who have conventional rationalities''

Society's opinion about madness and how to deal with it varies - isolationism, integration, or normalisation have been tried. Foucault argues that with the gradual disappearance of leprosy, madness came to occupy this excluded position. The ship of fools in the 15th century is a literary version of one such exclusionary practice, the practice of sending mad people away in ships.

The nature of creativity is another socially defined variable. In the 19th Century, Sass writes, the tradition of the romantic poet was the paradigm of a creative human. Eccentrics and outsiders had more trouble in some other times.

The relationship between creativity and madness has long been studied, and creative writing is frequently used for therapeutic purposes. Both institutional and open-access writers groups offer further opportunities for patients and for researchers. An awareness of literary fashions and the current role of writers groups will help maximise the benefits.

\subsubsection*{Writers and the Self}

 Subconsciously or otherwise, people may read literature for psychological benefits. Zunshine (2006), Mar, Oatley, Djikic, etc suggest that reading fiction is a kind of simulation of social interactions. ``After being given either fiction or non-fiction from the New Yorker, those who read the fiction piece scored higher on a test of social reasoning'' (``The Psychologist'', V21 No12, p.1030-1). Compared to the general reading public and even to other creative people, writers might have more need of these benefits

\begin{itemize} 
\item  ``Nancy Andreasen has tracked 30 students from the University of Iowa Writer's Workshop. 80\% had mood disorders (30\% is average amongst similar people who are non-writers). 43\% had some degree of manic-depressive illness (10\% is average). 2 committed suicide over the 15 years of the study'' (``Psychology Today'', April 1987)

\item  ``A great many writers find relating both painfully difficult and beside the point. The same qualities that make them writers - self-direction, independence, intelligence, skepticism, a love of solitude - also incline them in the direction of isolation, alienation and a carelessness about relating.'' (Maisel, 1999)

\end{itemize}


Writing may be a more useful form of simulation than passive reading, helping practitioners to

\begin{itemize} 
\item  analyse, prepare and anticipate human responses
\item  see other points of view 
\item  create retrospective autobiographical
narratives to analyse their past and plan behavioural reprogramming. Some writers create online personas, especially in the ``fanfic'' world - Fan fiction and its communities have long been of interest to academic researchers - see Henry Jenkins (1992)
\end{itemize}



Schizophrenia and depression are the mental illnesses most linked to creativity in the historical context (Schuldberg, 2001). He suggests that most often,

\begin{itemize}
         \item  artists who focus on emotions and feelings in their work are manic-depressive. Sass writes that poets like William Blake, Lord Byron, Shelley and Keats all suffered from manic-depressive illness
         \item  artists who remove themselves from the world are more often associated with schizophrenia. Creative people with schizophrenia often experience a sense of alienation from the self, from their bodies and from the world. They become hyper-self-conscious but are able to step outside themselves, allowing a more cerebral form of creativity.
\end{itemize}


Some  symptomatic mental traits could be seen as useful to writers, though the benefits tend to be specialised.

\begin{itemize}
\item        \textit{Obsession} - writers need to be determined and focussed
         \item \textit{Detachment; asocial distancing} - writers need to be observers (``a   poet even as falling down the stairs, will observe his fall''  - Holub). A lack of empathy may assist observation. Also, staying away from people frees up more time for writing. Awareness of a lack of social empathy may result in compensating strategies - increased observation, etc.
         \item  \textit{Asocial self-revelation/freedom} - freed from the constraints of politeness and political correctness, writers might produce more interesting (or at least provocative) work. This could be another compensating strategy, hoping to encourage a response.
         \item  \textit{Decontextualised (field-independent) thinking} - randomness, chaotic/original
thinking ``outside the box'', and finding unusual connections between things may help with creativity.
         \item  \textit{Multi-level thinking} - a characteristic of some schizoid thinking is the ability to see the underlying media without inferring meaning, seeing pattern as well as plot; noticing fonts, wordplay.
         \item  \textit{Sensitivity} - Highly Sensitive Persons and ``neurotic'' people might see things that others miss.
         \item  \textit{Inhibition} - excessive control may lead to an interest in Formalism and Oulipo
         \item  \textit{Private language} - Unusual forms of expression may result in interesting (albeit intractable) work.
\end{itemize}

\subsubsection*{The Writer in Society}

 ``There have always been people in societies and cultures who have different experiences of reality compared with the majority, and there's always been an overlap between people who have those gifts, or insights, and people who are identified as suffering from mental illnesses'' (Thomas, 2007). Social acceptance depends on the literary trends of the period. Some literary styles align with particular mental problems.

\begin{itemize}
    \item  \textit{Romanticism} - trying to be at one with nature presupposes a split between the mind and the world
         \item  \textit{Modernism} - reading Sass's ``Madness and Modernism'' one might easily believe that modernism is dominated by schizoids
        \item \textit{Nouveau Roman} - might suit the mind-blind
         \item  \textit{Confessionalism} - a school of poetry that merges well with therapy - easier to do if you don't care what others think of you, or the effects on family and friends
         \item  \textit{Surrealism/Dada} - these schools are based on random or subconscious images
         \item  \textit{Ermetismo (Hermeticism)}  - a school whose poems were characterized by unorthodox structure, illogical sequences, and highly subjective language.
   \item  \textit{Elliptical poetry} (a term coined in 1998 by Stephen Burt) - ``Full of illogic, of associative leaps, their poems resembled dreams, performances, speeches, or pieces of music''
\end{itemize}


Patients might consider themselves lucky if they're born into a era whose literary style matches
their symptoms. Currently there's no dominant literary mode. The Web has helped like-minded people
keep in touch, leading to a more fragmented literary scene where minority genres more easily survive.

\subsubsection*{Social Integration}

If writers are going to support themselves by writing nowadays, they will need to teach, so social adaption is useful. It may also improve the chances of wider publication. The risk from the writer's viewpoint is that if (as Freud believed) their writing's a symptom, then it might disappear as they become more ``healthy''. Several normalisation options are on offer
\begin{itemize}
         \item  \textit{Asylum} - rather than change the writer to fit society, new surroundings can be found to suit the writer. Some art colonies (and academic settings) are big enough to be self-contained worlds within which eccentricities are tolerated, even encouraged, creating their own norms that visitors adapt to.
         \item  \textit{ Borderline cases} - those with borderline symptoms may be more strongly encouraged to tweak their style rather than adopt wholesale changes - e.g. writing a narratively normal piece with a mad person safely compartmentalised as the main character; normalising the appearance of their manuscripts and cover letters.
         \item   \textit{Drugs} - These may be offered to make life easier, but they may dull writing. Schuldberg suggests that drugs blunt the creativity of patients with manic-depressive illness more than that of schizophrenic patients.
         \item   \textit{CBT} - behaviour change (e.g. being encouraged to meet people) may take the edge off writing or use up time.
\end{itemize}

Some neurotic people have a high tolerance for loneliness, and may find  writing a useful way to gain esteem while being alone. Socialising may not  ``cure'' writing, it may merely take away the opportunities for writing, leading  eventually to lowered self-esteem. However, socialising with writers 
is less problematic.

\subsubsection*{Writers Groups}

 ``Seizing on a traditional trope of the poet as exceptional individual, certain individuals receiving health-care who feel themselves to be exceptional apparently adopt poetic discourse as part of that role'', (Fiona Sampson, in ``Kicking Daffodils'', 1997). Some writers groups exist solely for people with diagnosed mental problems (see Survivors Poetry, etc). These and private study can help a writer's inner development, but before the writer emerges fully into society they can join a halfway house - a more public writers' group. The semi-structured discourse within a restricted domain coupled with tolerance of quirks makes such groups a welcoming environment. They range from one-off events to Master's degrees.

\begin{itemize}
\item \textit{Amateur} - Local writers groups are as popular as ever. Some focus more on literary appreciation than production. Their repetitive, undemanding format offers newcomers the chance to hone social and literary skills prepared in isolation. They usually have annual membership fees. In the bigger cities there are performance venues too, with open-mike sessions. Weekend and week-long residential courses are increasingly popular.
 
\item \textit{Academic} - There are creative writing evening classes in most towns.  UK universities are slowly catching up with their US counterparts (over 800 degree programs in creative writing exist in the States). Some are part-time with low-residency options.
 
\end{itemize}


 Technical competence, commercial success, emotional authority and educational status all contribute to a complex web of interaction in a group. In an educational context the tutors have a potential conflict between academic assessment and encouraging self-discovery. Autobiographical
writing is an exploitable grey area, especially when authenticity is considered a positive literary feature.  ``Creative Non-fiction'' (encompassing autobiography and personal essay) is on the increase.

\subsubsection*{How writers can use a public writers group as a support group}


 At one writers group I attended in the 80s, a subset huddled at the tea-break who seemed to have little in common. I later found out that they'd all been to the same local mental hospital. Therapy professionals who recommend patients to go to particular groups should perhaps attend one first - their atmosphere can vary a lot. Some groups offer tea, biscuits and companionship with a stable membership, others are competitive hot-houses. Some benefits of using writers groups are that they're cheaper than evening classes, casual, less committal, and (as opposed to self-help therapy groups) the person is not stigmatised as a patient. However, the writers need to be self-analytical enough to exploit the benefits, and meetings can be rather unstructured with difficult members. Advice includes
 
\begin{itemize}
\item trying an online group first. 
\item being prepared to face robust criticism that judges commercial potential more than depth of insight. Nervousness is common when presenting work, and crying's not unknown. 
\item looking upon all expression (comments as well as explicit self-analysis pieces) as revelation. The more gestures, the more that's said and   written, the more there is to analyse. 
\item monitoring how much you speak in proportion to how much others speak, and checking the proportion of positive comments you make compared to negative ones.
\item playing the game of wanting to be a writer, wanting to be published. 
\item studying theory in order to defend idiosyncrasies on theoretical grounds. Finding an appropriate style. Finding role models. Mixing theory with person-centred comments. 
\item going to the pub with members afterwards to get more varied feedback.
\end{itemize}

\subsubsection*{How psychologists can use writers groups}

 Less well covered than links between creativity and mental health is the group dynamics of writers groups. It's a mutually beneficial topic. A university creative writing course might
welcome multi-disciplinary interest from a psychology department, hoping to improve students' ability to benefit from (and run) workshops. Topics could include studying 

\begin{itemize}
\item  how a writer's style influences the type of criticism they offer, and how
  influential their comments are.
\item  how a person's psychological type affects their chosen genre - is fragmented work a reflection of, or reaction to, personality? 
\item  how genre affect happiness/survival statistics - are confessional poets
happier than similar people who don't write? 
\item  workshop dynamics (the different types of leadership and dissent) and how groups reach consensus  on a text
\end{itemize}


 The morality of using a writers group in this way needs to be addressed. There are precedents (in
literature as well as psychology - in the US particularly, writing tutors use teaching contexts as subject matter for their poems and stories nowadays), and the group organisers are often willing to volunteer. The organisers of writers group who I've met are used to one-off visits and bouts of strange behaviour.

\subsubsection*{References}
\begin{itemize}
         \item  Richard M. Berlin (ed) (2008), ``Poets on Prozac: Mental Illness, Treatment and the Creative Process'', John Hopkins University Press
         \item  Vicki Betram (ed) (1997), ``Kicking Daffodils'', Edinburgh University Press p.261
         \item  Stephen Burt (2009),
    \url{http://bostonreview.net/BR34.3/burt.php} ''The Boston Review'' May/June 2009
         \item  Kay Redfield Jamison (1993), ``Touched with Fire: Manic-Depressive Illness and the Artistic Temperament'', Simon and Schuster Adult Publishing Group
        \item Henry Jenkins (1992), ``Textual Poachers'', Routledge
\item J.C. Kaufman and S.B. Kaufman (2009), ``The Psychology of Creative Writing'', Cambridge University Press
        \item Eric Maisel (1999), ``Living the Writer's Life'', Watson-Guptill, p.125
        \item Louis A. Sass (1992), ``Madness and Modernism'', Harvard University Press

        \item David Schuldberg (2001) \url{http://www.lcmedia.com/mind195.htm} Infinite Mind: Art and Madness
         \item  Philip Thomas (2007) \url{http://www.independent.co.uk/life-style/health-and-families/health-news/is-there-a-link-between-madness-and-creativity-440374.html} The Independent, Sunday, 18 March 2007
\item Lisa Zunshine (2006) ``Why We Read Fiction: Theory of Mind and the Novel'', Ohio State University Press 

         \item  \url{http://www.poetrytherapy.org/} The National Association for Poetry Therapy and the \url{http://www.nfbpt.com/} National Federation for
Biblio/Poetry Therapy have information on training
         \item  The UK's Poetry Society were/are involved with various projects - see their
    \url{http://www.poetrysociety.org.uk/content/archives/healthcare/} healthcare page.
         \item  \url{http://www.survivorspoetry.org/} Survivor's poetry
\item \url{http://www.insidethebelljar.com/} Inside the BellJar (a literary journal dedicated to providing a raw and honest insight into the complexity of mental illness)
\item \url{http://www.thereader.org.uk/media/72227/therapeutic_benefits_of_reading_final_report_march_2011.pdf} An investigation into the therapeutic benefits of reading in relation to depression and well-being
\end{itemize}


\end{document}
